\chapter{Sentential Logic}
\markright{Chap. \ref{chap:SL}: Sentential Logic}
\label{chap:SL}
\setlength{\parindent}{1em}

\iflabelexists{part:cat_logic} %There are two versions of the preamble for this chapter, one for books that include the chapters on categorical logic, and a generic one
{In Part \ref{part:cat_logic}, we introduced a system of logic that dealt with categorical statements, statements like ``All people are mortal'' or ``Some dogs have  fleas.'' The system developed there was somewhat formal, because it replaced some of the contents of ordinary English sentences with abstract symbols. In Part \ref{part:sent_logic}, we go the rest of the way, and replace all of ordinary English with abstract symbols, thus creating a fully artificial language. In the previous system, capital letters like $S$ and $P$ stood for categories, like ``dogs'' or ``things that have fleas.'' In the new system individual letters will stand for whole sentences, like ``Tom wants to go to the bookstore'' or ``The sky is blue.'' Because individual letters stand for sentences this kind of system is known as \textit{sentential logic}, a term which we will be able to precisely define on page \pageref{def:sentential_logic}. We will call the specific version of sentential logic we will be developing SL.}%this is the preamble for texts that include categorical logic
{This chapter introduces a logical language called SL. It is a version of \emph{sentential logic}, because the basic units of the language will represent statements, and a statement is usually given by a complete sentence in English.} %this  is the generic preamble







% ******************************************
%  * Section 6.1  Sentence Letters                          *
% ******************************************

\section{Sentence Letters}


\newglossaryentry{sentence letter}
{
name=sentence letter,
description={A single capital letter, used in SL to represent a statement.}
}

\newglossaryentry{symbolization key}
{
name=symbolization key,
description={A list that shows which English sentences are represented by which sentence letters in SL. It is also called a dictionary.}
}

The most basic unit in our formal language SL is an individual capital letter---$A, B, C, D$, etc. These letters, called \textsc{\glspl{sentence letter}}, \label{def:sentence_letter} are used to represent individual statements. Remember in section \ref{def:statement}, we defined a statement as some bit of language that can be true or false, and listed all kinds of things that count as statements in English, from ``\emph{Tyrannosaurus rex} went extinct 65 million years ago'' to ``Lady Gaga is pretty.'' In SL, all these statements are reduced to single capital letters.

Considered only as a symbol of SL, the letter $A$ could mean any statement. So when translating from English into SL, it is important to provide a symbolization key, or dictionary. The \textsc{\gls{symbolization key}} \label{def:symbolization_key} provides an English language sentence for each sentence letter used in the symbolization. Consider this argument (recall that the portion of the passage in italics establishes the context, and is not part of the passage):

\begin{quotation}
\noindent \textit{A teacher is looking to see who has come to class} There is an apple on the desk. If there is an apple on the desk, then Jenny made it to class. Therefore, Jenny made it to class.
\end{quotation}

In canonical form, the argument would look like this:

\begin{earg}
\item[1.] There is an apple on the desk.
\item[2.] If there is an apple on the desk, then Jenny made it to class.
\item[] \textcolor{white}{.}\sout{\hspace{.8\linewidth}}\textcolor{white}{.} 
\item[$\therefore$] Jenny made it to class.
\end{earg}

A good symbolization key for this passage would look like this:

\begin{ekey}
\item[A:]There is an apple on the desk.
\item[B:]Jenny made it to class.
\end{ekey}

Why do the symbolization key this way? The argument we are looking at is obviously valid in English. In symbolizing it, we want to preserve the structure of the argument that makes it valid. We could have made each sentence in the original argument into its own letter. Then the symbolization key would look like this: 

\begin{ekey}
\item[A:]There is an apple on the desk.
\item[B:]If there is an apple on the desk, then Jenny made it to class.
\item[C:]Jenny made it to class.
\end{ekey}
But that would mean the argument would look like this:
\begin{earg}
\item[1.] $A$
\item[2.] $B$
\item[] \textcolor{white}{.}\sout{\hspace{.05\linewidth}}\textcolor{white}{.} 
\item[$\therefore$] $C$
\end{earg}
There is no necessary connection between some sentence $A$, which could be any statement, and some other sentences $B$ and $C$, which could also be anything. The structure of the argument has been completely lost in this translation.

The important thing about the argument is that the second premise is not merely \emph{any} statement, logically divorced from the other statement in the argument. The second premise contains the first premise and the conclusion \emph{as parts}. Our original symbolization key allows us to write the argument like this.

\begin{earg}
\item[1.] $A$
\item[2.] If $A$, then $B$.
\item[] \textcolor{white}{.}\sout{\hspace{.2\linewidth}}\textcolor{white}{.} 
\item[$\therefore$] $B$
\end{earg}
This preserves the structure of the argument that makes it valid, but it still makes use of the English expression ``If$\ldots$ then$\ldots$.'' Although we ultimately want to replace all of the English expressions with logical notation, this is a good start.

\newglossaryentry{atomic sentence}
{
name=atomic sentence,
description={A sentence that does not have any sentences as proper parts.}
}

The individual sentence letters in SL are called atomic sentences, because they are the basic building blocks out of which more complex sentences can be built. We can identify atomic sentences in English as well. An \textsc{\gls{atomic sentence}} \label{def:atomic_sentence} is one that cannot be broken into parts that are themselves sentences. ``There is an apple on the desk'' is an atomic sentence in English, because you can't find any proper part of it that forms a complete sentence. For instance ``an apple on the desk'' is a noun phrase, not a complete sentence. Similarly ``on the desk'' is a prepositional phrase, and not a sentence, and ``is an'' is not any kind of phrase at all. This is what you will find no matter how you divide ``There is an apple on the desk.'' On the other hand you can find two proper parts of ``If there is an apple on the desk, then Jenny made it to class'' that are complete sentences: ``There is an apple on the desk'' and ``Jenny made it to class.'' As a general rule, we will want to use atomic sentences in SL (that is, the sentence letters) to represent atomic sentences in English. Otherwise, we will lose some of the logical structure of the English sentence, as we have just seen. 

There are only 26 letters of the alphabet, but there is no logical limit to the number of atomic sentences. We can use the same letter to symbolize different atomic sentences by adding a subscript, a small number written after the letter. We could have a symbolization key that looks like this:
\begin{ekey}
\item[A$_1$:] The apple is under the armoire.
\item[A$_2$:] Arguments in SL always contain atomic sentences.
\item[A$_3$:] Adam Ant is taking an airplane from Anchorage to Albany.
\item[$\vdots$]
\item[A$_{294}$:] Alliteration angers otherwise affable astronauts.
\end{ekey}
Keep in mind that each of these is a different sentence letter. When there are subscripts in the symbolization key, it is important to keep track of them.


% ******************************************
%  * Sentential Connectives		                          *
% ******************************************

\section{Sentential Connectives}


The previous section introduced the basic elements of SL, the sentence letters. But when we were looking at the argument involving Jenny and the apple, we saw that the best way to write a dictionary for the argument left the words ``if'' and ``then'' in English. In this section we will introduce ways to connect the sentence letters together that will allow us to form a complete artificial language.   

\newglossaryentry{sentential connective}
{
name=sentential connective,
description={A logical operator in SL used to combine sentence letters into larger sentences.}
}

\newglossaryentry{logical constant}
{
name=logical symbol,
description={A symbol whose meaning is fixed by a formal language. Sometimes these are just called ``logical symbols.'' They are contrasted with \textsc{non-logical symbols}. (See below.)}
}

\newglossaryentry{nonlogical symbol}
{
name=nonlogical symbol,
description={A symbol whose meaning is not fixed by a formal language.}
}

The symbols used to connect sentence letters are called \textsc{\glspl{sentential connective}} \label{def:sentential_connective}, naturally enough. SL uses five sentential connectives: \eand, \eor, \enot, \eif, and \eiff. To write the sentence about Jenny and the apple we use the symbol ``\eif.'' Using the dictionary above, ``If there is an apple on the desk, then Jenny made it to class'' becomes $A \eif B$. Table \ref{table:sentential_connectives} summarizes the meaning of the five sentential connectives.

The sentential connectives are a kind of \textsc{\gls{logical constant}}, because their meaning is fixed by the formal language that we have chosen. The sentence letters, by contrast, are \textsc{\glspl{nonlogical symbol}}, \label{def:nonlogical_symbol} because their meaning can change as we change the symbolization key. 

The subsections below describe each connective in more detail.

\begin{table}
\begin{mdframed}[style=mytablebox]
\begin{tabu}{p{.1\linewidth}p{.3\linewidth}p{.3\linewidth}}
\underline{Symbol}&\underline{What it is called}&\underline{What it means}\\
\enot&negation&``It is not the case that$\ldots$''\\
\eand&conjunction&``Both $\ldots$\ and $\ldots$''\\
\eor&disjunction&``Either $\ldots$\ or $\ldots$''\\
\eif&conditional&``If $\ldots$\ then $\ldots$''\\
\eiff&biconditional&``$\ldots$ if and only if $\ldots$''\\
\end{tabu}
\end{mdframed}
\caption{The Sentential Connectives.}
\label{table:sentential_connectives}
\end{table}

%%%%%%%%%%%%%%%%%% 2.2.1 Negation

\subsection{Negation}
Consider how we might symbolize these sentences:
\begin{earg}
\item[\ex{not1}] Mary is in Barcelona.
\item[\ex{not2}] Mary is not in Barcelona.
\item[\ex{not3}] Mary is somewhere other than Barcelona.
\end{earg}

In order to symbolize sentence \ref{not1}, we will need one sentence letter. We can provide a symbolization key:

\begin{ekey}
\item[B:]Mary is in Barcelona.
\end{ekey}

Note that here we are giving $B$ a different interpretation than we did in the previous section. The symbolization key only specifies what $B$ means \emph{in a specific context}. It is vital that we continue to use this meaning of $B$ so long as we are talking about Mary and Barcelona. Later, when we are symbolizing different sentences, we can write a new symbolization key and use $B$ to mean something else.

\newglossaryentry{negation}
{
name=negation,
description={The symbol \enot, used to represent words and phrases that function like the English word ``not''.}
}

Now, sentence \ref{not1} is simply $B$. Sentence \ref{not2} is obviously related to sentence \ref{not1}: it is basically \ref{not1} with a ``not'' added. We could put the sentence partly our symbolic language by writing ``Not $B$.'' This means we do not want to introduce a different sentence letter for \ref{not2}. We just need a new symbol for the ``not'' part. Let's use the symbol `\enot,' which we will call \textsc{\gls{negation}}. \label{def:negation} Now we can translate `Not $B$' to $\enot B$. 

Sentence \ref{not3} is about whether or not Mary is in Barcelona, but it does not contain the word ``not.'' Nevertheless, it is obviously logically equivalent to sentence \ref{not2}. They both say that if you are looking for Mary, you shouldn't look in Barcelona. Remember that in section \ref{def:logical_equivalence}, we said that two sentences in English are logically equivalent if they always have the same truth value. For our purposes, this means that they basically say the same thing. It is clear then that \ref{not2} and \ref{not3} are logically equivalent, so we can translate them both as $\enot B$.



Consider these further examples:
\begin{earg}
\item[\ex{not4}] The widget can be replaced if it breaks.
\item[\ex{not5}] The widget is irreplaceable.
\item[\ex{not5b}] The widget is not irreplaceable.
\end{earg}


If we let $R$ mean ``The widget is replaceable'', then sentence \ref{not4} can be translated as $R$. Sentence \ref{not5} means the opposite of sentence \ref{not4}, so we can translate it $\enot R$. Sentence \ref{enot5b} adds another negation to sentence \ref{not5}. We know, as competent English speakers, that the two negations cancel each other out, so that sentence \ref{enot5b} is equivalent to sentence \ref{not4}. But the fact that two negations cancel each other out is a part of the logic of English that we actually want to capture with our formal language SL. So we will represent the two negations in sentence \ref{not5b} as two negations in SL: $\enot \enot R$. We will now have to be sure that in SL the sentences $R$ and $\enot \enot R$ mean the same thing. 

As the above examples begin to indicate, English has all kinds of ways to negate a sentence.  Sometimes we use an explicit ``not.'' Sometimes we use a prefix like the ``ir-'' in ``irreplaceable.'' SL has just one way to form a negation: slap a \enot in front of the sentence. There is an English expression, however, that always occurs in the same place in an English sentence as the \enot occurs in the sentence SL. The English phrase is ``It is not the case that.'' Although this phrase sounds awkward, it always occurs in front of the sentence it is negating, just as the symbol \enot does. This makes it useful in translating sentences from SL back into English. $\enot R$ can be translated ``it is not the case that this widget is replaceable.'' In the earlier example, $\enot B$ can be translated ``It is not the case that Mary is in Barcelona.'' 

\factoidbox{
A sentence can be symbolized as $\enot\script{A}$ can always be paraphrased in English as ``It is not the case that \script{A}.''
}

Sometimes negations in English do not function as neatly as the \enot does in SL, because two things aren't perfect opposites. Consider these sentences:

\begin{earg}
\item[\ex{not6}] Elliott is happy.
\item[\ex{not7}] Elliott is unhappy.
\end{earg}


If we let $H$ mean ``Elliot is happy'', then we can symbolize sentence \ref{not6} as $H$, but does \ref{not7} really mean the same thing as $\enot H$? Saying ``Elliott is unhappy'' indicates that Elliott is actively sad. But $\enot H$ can be paraphrase as simply ``It is not the case that Elliott is happy,'' which might merely mean that Elliott is just feeling neutral. As we saw on page \pageref{def:bivalent}, the logics we discuss in this textbook are \emph{bivalent}. Statements are only either true or false. Everything is in black and white, and issues like Elliott's fine gradations in mood cannot be directly represented in our system. So in SL, sentences \ref{not6} and  \ref{not7} would generally be represented by separate sentence letters.

One way of capturing the meaning of a sentential connective is to make a table which shows how the connective changes the meaning of the sentences it is applied to. The negation simply reverses the truth value of any sentence it is put in front of. For any sentence \script{A}: If \script{A} is true, then \enot\script{A} is false. If \enot\script{A} is true, then \script{A} is false. Using T for true and F for false, we can summarize this in a \emph{characteristic truth table} for negation:

\begin{center}
\begin{tabular}{c|c}
\script{A} & \enot\script{A}\\
\hline
T & F\\
F & T 
\end{tabular}
\end{center}
We will discuss truth tables at greater length in the next chapter.

%%%%%%%%%%%%%%%%%% 2.2.2 Conjunction

\subsection{Conjunction}
Consider these sentences:
\begin{earg}
\item[\ex{and1}]Adam is athletic.
\item[\ex{and2}]Barbara is athletic.
\item[\ex{and3}]Adam is athletic, and Barbara is also athletic.
\end{earg}

We will need separate sentence letters for \ref{and1} and \ref{and2}, so we define this symbolization key:
\begin{ekey}
\item[A:] Adam is athletic.
\item[B:] Barbara is athletic.
\end{ekey}


\newglossaryentry{conjunction}
{
name=conjunction,
description={The symbol \eand, used to represent words and phrases that function like the English word ``and.''}
}

\newglossaryentry{conjunct}
{
name=conjunct,
description={A sentences joined to another by a conjunction.}
}

Sentence \ref{and1} can be symbolized as $A$. Sentence \ref{and2} can be symbolized as $B$. Sentence \ref{and3} can be paraphrased as ``$A$ and $B$.'' In order to fully symbolize this sentence, we need another symbol. We will use \eand. We translate ``$A$ and $B$'' as $A\eand B$. The logical connective \eand is called the \textsc{\gls{conjunction}}, \label{def:conjunction} and $A$ and $B$ are each called \textsc{\glspl{conjunct}}. \label{def:conjunct}

Notice that we make no attempt to symbolize ``also'' in sentence \ref{and3}. Words like ``both'' and ``also'' function to draw our attention to the fact that two things are being conjoined. They are not doing any further logical work, so we do not need to represent them in SL.

Some more examples:
\begin{earg}
\item[\ex{and4}]Barbara is athletic and energetic.
\item[\ex{and5}]Barbara and Adam are both athletic.
\item[\ex{and6}]Although Barbara is energetic, she is not athletic.
\item[\ex{and7}]Barbara is athletic, but Adam is more athletic than she is.
\end{earg}

Sentence \ref{and4} is obviously a conjunction. The sentence says two things about Barbara, so in English it is permissible to refer to Barbara only once. It might be tempting to try this when translating the argument: Since $B$ means ``Barbara is athletic'', one might paraphrase the sentences as ``$B$ and energetic.'' This would be a mistake. Once we translate part of a sentence as $B$, any further structure is lost. $B$ is an atomic sentence; it is nothing more than true or false. Conversely, ``energetic'' is not a sentence; on its own it is neither true nor false. We should instead paraphrase the sentence as ``$B$ and Barbara is energetic.'' Now we need to add a sentence letter to the symbolization key. Let $E$ mean ``Barbara is energetic.'' Now the sentence can be translated as $B \eand E$.

\factoidbox{
A sentence can be symbolized as $\script{A} \eand \script{B}$ if it can be paraphrased in English as `Both \script{A}, and \script{B}.' Each of the conjuncts must be a sentence.
}

Sentence \ref{and5} says one thing about two different subjects. It says of both Barbara and Adam that they are athletic, and in English we use the word ``athletic'' only once. In translating to SL, it is important to realize that the sentence can be paraphrased as, ``Barbara is athletic, and Adam is athletic.'' This translates as $B \eand A$.

Sentence \ref{and6} is a bit more complicated. The word ``although'' sets up a contrast between the first part of the sentence and the second part. Nevertheless, the sentence says both that Barbara is energetic and that she is not athletic. In order to make each of the conjuncts an atomic sentence, we need to replace ``she'' with ``Barbara.''

So we can paraphrase sentence \ref{and6} as, ``\emph{Both} Barbara is energetic, \emph{and} Barbara is not athletic.'' The second conjunct contains a negation, so we paraphrase further: ``\emph{Both} Barbara is energetic \emph{and} \emph{it is not the case that} Barbara is athletic.'' This translates as $E \eand \enot B$.

Sentence \ref{and7} contains a similar contrastive structure. It is irrelevant for the purpose of translating to SL, so we can paraphrase the sentence as ``\emph{Both} Barbara is athletic, \emph{and} Adam is more athletic than Barbara.'' (Notice that we once again replace the pronoun ``she'' with her name.) How should we translate the second conjunct? We already have the sentence letter $A$ which is about Adam's being athletic and $B$ which is about Barbara's being athletic, but neither is about one of them being more athletic than the other. We need a new sentence letter. Let $R$ mean ``Adam is more athletic than Barbara.'' Now the sentence translates as $B \eand R$.

\factoidbox{Sentences that can be paraphrased ``\script{A}, but \script{B}'' or ``Although \script{A}, \script{B}'' are best symbolized using conjunction  \script{A} \eand \script{B}.}

It is important to keep in mind that the sentence letters $A$, $B$, and $R$ are atomic sentences. Considered as symbols of SL, they have no meaning beyond being true or false. We have used them to symbolize different English language sentences that are all about people being athletic, but this similarity is completely lost when we translate to SL. No formal language can capture all the structure of the English language, but as long as this structure is not important to the argument there is nothing lost by leaving it out.

As with the negation, we can understand the meaning of the conjunction by making a table that shows how the conjunction affects the truth value of the  sentences it is bringing together. 
For any sentences \script{A} and \script{B}, \script{A} \eand \script{B} is true if and only if both \script{A} and \script{B} are true. We can summarize this in the {characteristic truth table} for conjunction:
\begin{center}
\begin{tabular}{c|c|c}
\script{A} & \script{B} & \script{A} \eand \script{B}\\
\hline
T & T & T\\
T & F & F\\
F & T & F\\
F & F & F
\end{tabular}
\end{center}

Conjunction is symmetrical because we can swap the conjuncts without changing the truth value of the sentence. Regardless of what \script{A} and \script{B} are, \script{A}\eand\script{B} is logically equivalent to \script{B} \eand \script{A}.


%%%%%%%%%%%%%%%%%%%% 2.2.3 disjunction

\subsection{Disjunction}
Consider these sentences:
\begin{earg}
\item[\ex{or1}]Either Denison will play golf with me, or he will watch movies.
\item[\ex{or2}]Either Denison or Ellery will play golf with me. 
\end{earg}

For these sentences we can use this symbolization key:

\begin{ekey}
\item[D:] Denison will play golf with me.
\item[E:] Ellery will play golf with me.
\item[M:] Denison will watch movies.
\end{ekey}

\newglossaryentry{disjunction}
{
name=disjunction,
description={The symbol \eor, used to represent words and phrases that function like the English word ``or'' in its inclusive sense.}
}

\newglossaryentry{disjunct}
{
name=disjunct,
description={A sentences joined to another by a disjunction.}
}



Sentence \ref{or1} is ``Either $D$ or $M$.'' To fully symbolize this, we introduce a new symbol. The sentence becomes $D \eor M$. The $\eor$ connective is called \textsc{\gls{disjunction}}, \label{def:disjunction} and $D$ and $M$ are called \textsc{\glspl{disjunct}}. \label{def:disjunct}

Sentence \ref{or2} is only slightly more complicated. There are two subjects, but the English sentence only gives the verb once. In translating, we can paraphrase it as ``Either Denison will play golf with me, or Ellery will play golf with me.'' Now it obviously translates as $D \eor E$.


\factoidbox{
A sentence can be symbolized as $\script{A}\eor\script{B}$ if it can be paraphrased in English as ``Either \script{A} or \script{B}.'' Each of the disjuncts must be a sentence.
}


\newglossaryentry{exclusive or}
{
name=exclusive or,
description={A kind of disjunction that excludes the possibility that both disjuncts are true. The exclusive or says ``This or that, but not both.''}
}

\newglossaryentry{inclusive or}
{
name=inclusive or,
description={A kind of disjunction that allows for the possibility that both disjuncts are true. The inclusive or says ``This or that, or both.''}
}


The English word ``or'' is somewhat ambiguous. Sometimes in English, when we say ``this or that,'' we mean that either option is possible, but not both. For instance, if a  restaurant menu says, ``Entr\'ees come with either soup or salad'' we naturally assume you can have soup, or you can have salad; but, if you want \emph{both} soup \emph{and} salad, then you will have to pay extra. This kind of disjunction is called an \textsc{\gls{exclusive or}} \label{def:exclusive_or}, because it excludes the possibility that both disjuncts are true. 
 
At other times, the word ``or'' allows for the possibility that both disjuncts might be true. This is probably the case with sentence \ref{or2}, above. I might play with Denison, with Ellery, or with both Denison and Ellery. Sentence \ref{or2} merely says that I will play with \emph{at least} one of them. The \textsc{\gls{inclusive or}}\label{def:inclusive _or} is the kind of disjunction that allows for the possibility that both disjuncts are true. The inclusive or says ``This or that, or both.''

To goal of a formal language is to remove ambiguity, so we need to pick one of these ors. SL follows tradition and uses the symbol $\eor$ to represent an \emph{inclusive or}. This winds up being reflected in the characteristic truth table for the $\eor$. The sentence $D \eor E$ is true if $D$ is true, if $E$ is true, or if both $D$ and $E$ are true. It is false only if both $D$ and $E$ are false. The truth table looks like this:

\begin{center}
\begin{tabular}{c|c|c}
\script{A} & \script{B} & \script{A} \eor \script{B} \\
\hline
T & T & T\\
T & F & T\\
F & T & T\\
F & F & F
\end{tabular}
\end{center}

Like conjunction, disjunction is symmetrical. \script{A} \eor \script{B} is logically equivalent to \script{B} \eor \script{A}.


%%%%%%%%%%%%%%%%% 2.2.4 conditional

\subsection{Conditional}

\newglossaryentry{conditional}
{
name=conditional,
description={The symbol \eif, used to represent words and phrases that function like the English phrase ``if \ldots then.''}
}

\newglossaryentry{antecedent}
{
name=antecedent,
description={The sentence to the left of a conditional..}
}

\newglossaryentry{consequent}
{
name=consequent,
description={The sentence to the right of a conditional.}
}

We already met the conditional at the start of this section, when we were discussing the sentence ``If there is an apple on the table, Jenny made it to class,'' which became $A \eif B$. The symbol $\eif$ is called a \textsc{\gls{conditional}}. \label{def:conditional} The sentence on the left-hand side of the conditional ($R$ in this example) is called the \textsc{\gls{antecedent}}. \label{def:antecedent}.  The sentence on the right-hand side ($B$) is called the \textsc{\gls{consequent}}. \label{def:consequent} 
	
Like the English word ``or,'' the English phrase ``if\ldots then\ldots'' has some ambiguity. Consider our original example, ``If there is an apple on the table, Jenny made it to class.'' The statements tells us what we should infer if there is an apple on the table, but what if there \emph{isn't} an apple on the table. Does that guarantee that Jenny did not make it to class? It could be that an apple on the table is a clear sign that Jenny made it to class, because no one else would put an apple on the table, but nevertheless Jenny sometimes comes to class without putting an apple on the table. 

We can get a good sense of the decision we face if we try to write up the characteristic truth table for the conditional. The first two lines are easy. The sentence``If \script{A}, then \script{B}'' means that if \script{A} is true, then so is \script{B}. This would be confirmed by the situation where both \script{A} and \script{B} are true, but falsified by the situation where \script{A} is true and \script{B} is false. In terms of our example, if we came to class and found the apple there, but Jenny absent, we would know that the statement ``If there is an apple on the table, Jenny made it to class'' is false. But if we came to class and found both Jenny and the apple present, we could say that the statement ``If there is an apple on the table, Jenny made it to class'' is true. That gives us this much of a truth table.


\begin{center}
\begin{tabular}{c|c|c}
\script{A} & \script{B} & \script{A}\eif\script{B}\\
\hline
T & T & T\\
T & F & F\\
F & T & ?\\
F & F & ?
\end{tabular}
\end{center}

How do we fill in the question marks in the last two lines?  In real life, we would generally make judgments on a case by case basis, relying heavily on the context we are in. But for a formal language we just want to lay down a simple rule. The traditional solution for sentential logic is to say that the conditional is what logicians call a ``material conditional.'' If the antecedent of a material conditional is false, then the whole statement is automatically true, regardless of the truth value of \script{B}. In short, \script{A} \eif \script{B} is false if and only if \script{A} is true and \script{B} is false. We can summarize this with a characteristic truth table for the conditional.

\begin{center}
\begin{tabular}{c|c|c}
\script{A} & \script{B} & \script{A}\eif\script{B}\\
\hline
T & T & T\\
T & F & F\\
F & T & T\\
F & F & T
\end{tabular}
\end{center}

The conditional is asymmetrical. You cannot swap the antecedent and consequent without changing the meaning of the sentence, because \script{A} \eif \script{B} and \script{B} \eif \script{A} are not logically equivalent.

%\begin{earg}
%\item[\ex{if3}] Everytime a bell rings, an angel earns its wings.
%\item[\ex{if4}] Bombs always explode when you cut the red wire.
%\end{earg}

Not all sentences of the form ``If$\ldots$, then$\ldots$'' are conditionals. Consider this sentence:

\begin{earg}
\item[\ex{if5}] If anyone wants to see me, then I will be on the porch.
\end{earg}

When I say this, it means that I will be on the porch, regardless of whether anyone wants to see me or not---but if someone did want to see me, then they should look for me there. If we let $P$ mean ``I will be on the porch,'' then sentence \ref{if5} can be translated simply as $P$.

%%%%%%%%%%%%%%%% 6.2.5 Biconditional

\subsection{Biconditional}

\newglossaryentry{biconditional}
{
name=biconditional,
description={A sentential connective, written as double headed arrow, $\eiff$, used to represent a situation where $A$ implies  $B$ and $B$ implies $A$. This is also the situation where $A$ and $B$ are logically equivalent. This is often expressed by the English phrase ``if and only if.''}
}

The conditional was an asymmetric connective. The sentence $A \eif B$ does not mean the same thing as the sentence $B \eif A$. It is convenient to have a single symbol that combines the meaning of these two sentences. The \textsc{\gls{biconditional}}\label{def:bicondional}---written as double headed arrow, $\eiff$---is a sentential connective used to represent a situation where $A$ implies $B$ and $B$ implies $A$. 

To draw up the characteristic truth table for the biconditional, we need to think about the situations where $A \eif B$ and $B \eif A$ are false. The sentence $A \eif B$ is only false when $A$ is true and $B$ is false. For $B \eif A$ the reverse is true. It is false when $B$ is true and $A$ is false. Our biconditional $A \eiff B$ needs to avoid both of these situations to be true, because it is only true when $A \eif B$ and $B \eif A$ are true. This, then, is the characteristic truth table for the biconditional. It says that the biconditional is true when the truth values of the two sides match.

\begin{center}
\begin{tabular}{c|c|c}
\script{A} & \script{B} & \script{A} \eiff \script{B}\\
\hline
T & T & T\\
T & F & F\\
F & T & F\\
F & F & T
\end{tabular}
\end{center}

If the bioconditional holds between two sentences, we can that the two sentences are logically equivalent. Back on page \pageref{def:logical_equivalence}, we said that two sentences were logically equivalent if they always  have the same truth value. That is exactly what is happening here. 


% ******************************************
%  * 		More Complicated Translations                     *
% ******************************************

\section{More Complicated Translations}

\iflabelexists{part:cat_logic} %There are two versions of this passage, one for books that include the chapters on categorical logic, and a generic one
{Back in section \ref{sec:transformation}, we saw that the system of categorical logic we were studying at the time could actually represent a large range of sentences in ordinary English, even though it only had the quantifiers ``All'' and ``some'' plus negation. In this section, we will see that something similar happens with SL. There is actually a lot we can cover, even though we only have five connectives. }%this is the preamble for texts that include categorical logic
{The previous section introduced the five sentential connectives. Now we will look at some trickier translations involving those connectives}%this is the generic preamble


\subsection{Combining connectives}

\iflabelexists{part:cat_logic} %There are two versions of this passage, one for books that include the chapters on categorical logic, and a generic one
{In our system of categorical logic, we just had four kinds of sentences---A, E, I, and O---and if we wanted to combine them, the only way to do that would be to form a syllogism. In SL, we can combine an unlimited number of connectives together into a single sentence to express complicated ideas that couldn't be represented by Aristotelean logic. }%this is the preamble for texts that include categorical logic
{A single sentence in SL can use multiple connectives.} %this is the generic preamble

Consider the English sentence ``If it is not raining, we will have a picnic.'' There are two aspects of this sentence we will want to represent with sentential connectives in SL, the ``if\ldots then\ldots'' structure and the negation in the first part of the sentence. The rest of the sentence can be represented by these sentence letters

\begin{ekey}
\item[$A$:] It is raining.
\item[$M$:] We will have a picnic.
\end{ekey}

We can then translate the whole sentence into SL like this: $\enot A \eif B$. We can make sentences as complicated as we want this way, even to the point where the equivalent English sentence would be impossible to follow. The sentence $\enot (P \eand Q) \eif  [(R \eor S) \eiff \enot (T \eand U)]$ is perfectly acceptable in SL, even if any English sentence it translates into would be a monster. This is part of the power of a complete formal language like SL, but it is also why arguments in SL begin to resemble the ob/ob mouse more than they resemble any argument you might encounter in the wild. (See page \pageref{fig:ob_ob_mouse}) 

Although sentences in SL can be as long as you like, you can't just combine symbols any old way. There is a specific set of rules you have to follow. These are outlined in section \ref{recursive_syntax_for_SL}, below.

The fact that we can write these more complicated sentences means we can actually do without some of the connectives we have given ourselves in SL. For instance, we don't really need the biconditional. Any sentence of the form $\script{A} \eiff \script{B}$ is going to be equivalent to the sentence $(\script{A} \eif \script{B}) \eand (\script{B} \eif \script{A})$. This just follows from the way we defined the biconditional earlier. Nevertheless, tradition and convenience mandate that we give the biconditional a separate symbol.

\subsection{Unless}

Because our connectives can be put together in different ways, some English sentences can be represented equally well by multiple sentences in SL. English sentences involving the word ``unless'' are a case in point. 

\begin{earg}
\item[\ex{unless1}] Unless you wear a jacket, you will catch cold. 
\item[\ex{unless2}] You will catch cold unless you wear a jacket. 
\end{earg}

These are basically two different version of the same English sentence. The only difference is that in one case, the ``unless'' clause comes first, and in the other it comes second. Let $J$ mean ``You will wear a jacket'' and let $C$ mean ``You will catch a cold.'' We can paraphrase sentence \ref{unless1} as ``Unless $J$, $C$.'' This means that if you do not wear a jacket, then you will catch cold. With this in mind, we might translate it as $\enot J \eif C$. It also means that if you do not catch a cold, then you must have worn a jacket; with this in mind, we might translate it as $\enot C \eif J$.

Which of these is the correct translation of sentence \ref{unless1}? Both translations are correct, because the two translations are logically equivalent in SL. Sentence \ref{unless2}, in English, is logically equivalent to sentence \ref{unless1}. So, it also can be translated as either $\enot J \eif D$ or $\enot D \eif J$.

When symbolizing sentences like sentence \ref{unless1} and sentence \ref{unless2}, it is easy to get turned around. We have two different versions of the English sentence and two different versions of the sentence in SL. The important thing to see here is that none of these sentences are equivalent to $J \eif \enot D$. The negated statement must be the antecedent to the conditional. 

If this is too many options to keep track of, there is a simpler alternative. It turns out that any ``unless'' statement is actually equivalent to an ``or'' statement. Both statements \ref{unless1} and  \ref{unless2} mean that you will wear a jacket or---if you do not wear a jacket---then you will catch a cold. So we can translate them as $J \eor D$. (You might worry that the ``or'' here should be an \emph{exclusive or}. However, the sentences do not exclude the possibility that you might \emph{both} wear a jacket \emph{and} catch a cold; jackets do not protect you from all the possible ways that you might catch a cold.)


\factoidbox{
If a sentence can be paraphrased as ``Unless \script{A}, \script{B},'' then it can be symbolized as $\script{A}\eor\script{B}$.
}

\subsection{Only}

\iflabelexists{part:cat_logic} %There are two versions of this passage, one for books that include the chapters on categorical logic, and a generic one
{In section \ref{sec:transformation}, we saw that the word ``only'' could reverse the meaning of a statement in Mood A. ``All dogs are mammals'' means something different than ``Only dogs are mammals,'' the first one is true but the second one is false. Something similar happens with conditional statements in SL.}%this is the preamble for texts that include categorical logic
{[The word ``only'' can reverse the meaning of a conditional sentence in SL.]}%this is the generic preamble
For the following sentences, let $R$ mean ``You will cut the red wire'' and $B$ mean ``The bomb will explode.''

\begin{earg}
\item[\ex{if1}] If you cut the red wire, then the bomb will explode.
\item[\ex{if2}] The bomb will explode only if you cut the red wire.
\end{earg}

Sentence \ref{if1} can be translated partially as ``If $R$, then $B$.'' Sentence \ref{if2} is also a conditional. Since the word ``if'' appears in the second half of the sentence, it might be tempting to symbolize this in the same way as sentence \ref{if1}. That would be a mistake.

The conditional $R\eif B$ says that \emph{if} $R$ were true, \emph{then} $B$ would also be true. It does not say that you cutting the red wire is the \emph{only} way that the bomb could explode. Someone else might cut the wire, or the bomb might be on a timer. The sentence $R\eif B$ does not say anything about what to expect if $R$ is false. Sentence \ref{if2} is different. It says that the only conditions under which the bomb will explode involve you having cut the red wire; i.e., if the bomb explodes, then you must have cut the wire. As such, sentence \ref{if2} should be symbolized as $B \eif R$.

It is important to remember that the connective $\eif$ says only that, if the antecedent is true, then the consequent is true. It says nothing about the \emph{causal} connection between the two events. Translating sentence \ref{if2} as $B \eif R$ does not mean that the bomb exploding would somehow have caused you cutting the wire. Both sentence \ref{if1} and \ref{if2} suggest that, if you cut the red wire, you cutting the red wire would be the cause of the bomb exploding. They differ on the \emph{logical} connection. If sentence \ref{if2} were true, then an explosion would tell us---those of us safely away from the bomb---that you had cut the red wire. Without an explosion, sentence \ref{if2} tells us nothing.

\factoidbox{
The paraphrased sentence ``\script{A} only if \script{B}'' is logically equivalent to ``If \script{A}, then \script{B}.''
}

Things can get a bit more complicated, because English also allows you to reverse the order of the clauses. Think about this sentence

\begin{earg}
\item[\ex{if3}] The bomb will explode, if you cut the red wire
\end{earg}

This is just sentence \ref{if1} with the order of the clauses reversed, so it still means $R \eif B$. Changing the order of the English clauses does not change the sentence in SL, but adding the word ``only'' does.

If this gets confusing, just remember this rule: 

\factoidbox{
``If\ldots'' introduces the antecedent. ``Only if\ldots'' introduces the consequent. 
}

Because ``if'' and ``only if'' have opposite meanings, when we put them together, we get the biconditional. Consider these sentences:
\begin{earg}
\item[\ex{iff1}] The figure on the board is a triangle only if it has exactly three sides.
\item[\ex{iff2}] The figure on the board is a triangle if it has exactly three sides.
\item[\ex{iff3}] The figure on the board is a triangle if and only if it has exactly three sides.
\end{earg}

Let $T$ mean ``The figure is a triangle'' and $S$ mean ``The figure has three sides.'' Sentence \ref{iff1}, for reasons discussed above, can be translated as $T\eif S$. Sentence \ref{iff2} is importantly different. It can be paraphrased as ``If the figure has three sides, then it is a triangle.'' So it can be translated as $S\eif T$.

Sentence \ref{iff3} says that $T$ is true \emph{if and only if} $S$ is true; we can infer $S$ from $T$, and we can infer $T$ from $S$.  In other words, \ref{iff3} is equivalent to $T\eif S$ and $S\eif T$, which is the same as $T \eiff S$

A final way to think about the way ``only'' effects a conditional sentence is to think about the  difference between necessary and sufficient conditions. In a way, the terms are pretty much self explanatory. Nevertheless, it is really easy to get them confused, to the extent that even professional logicians and trained philosophers can get them mixed up. 


\newglossaryentry{necessary condition}
{
name=necessary condition,
description={A condition that must be true in order for something else to be, generally contrasted with a \textit{sufficient condition}.}
}

A \textsc{\gls{necessary condition}}\label{def:necessary_condition} is one that is needed for something else to be true, just like the name says. Having gas in the tank is a \textit{necessary} condition for the car to move. It just doesn't go anywhere without gas. However, having gas in the tank isn't \textit{all you need} to get the car moving. You also have to put the key in the  ignition and turn it. 

\newglossaryentry{sufficient condition}
{
name=sufficient condition,
description={A condition that is all you need for something to be true, generally contrasted with a \textit{necessary condition}.}
}

A \textsc{\gls{sufficient condition}}\label{def:sufficient_condition}, on the other hand, is \textit{all you need} for something else to be true. If something is a dog, that is a \textit{sufficient} condition for it to be a mammal. Once you know Cupcake (Fig. \ref{fig:cupcake}) is a dog, you have enough information to infer that she is a mammal. Being a dog is not a necessary condition for being a mammal however. You can also be a mammal being being a cat, or a human, or a wombat. 

\begin{figure}
\begin{mdframed}[style=mytableclearbox]
\begin{center}
\includegraphics*{img/cupcake}
\end{center}
\end{mdframed}
\caption{This is Cupcake. The fact that she is a dog is a \textit{sufficient} condition for her to be a mammal. She also likes socks.}
\label{fig:cupcake}
\end{figure}

The conditional symbol in SL represents a sufficient condition, at least when read forward. That is, the antecedent is a sufficient condition for the consequent. If you have the antecedent, that is all you need to know to infer the consequent. So if $D$ is ``Cupcake is a dog'' and $M$ is ``Cupcake is a mammal, then $D \eif C$ is true. Being a dog is sufficient for being a mammal. As it turns out, if the relationship is sufficient going one direction, it is necessary going the other. So being a mammal is a necessary condition for being a mammal. If cupcake weren't a mammal, there would be no way for her to be a dog. Figure \ref{fig:necessary_and_sufficient} shows this relationship.

\begin{figure}
\begin{mdframed}[style=mytableclearbox, userdefinedwidth=.5\textwidth]
\begin{center}
\includegraphics*{img/necessaryandsufficient.png}
\end{center}
\end{mdframed}
\caption{The antecedent of a material conditional is a sufficient condition for the consequent, while the consequent is a necessary condition for the antecedent.}
\label{fig:necessary_and_sufficient}
\end{figure}


\subsection{Combining negation with conjunction and disjunction}

Tricky things happen when you combine a negation with a conjunction or disjunction, so it is worth taking a closer look here. Consider these sentences

\begin{earg}
\item[\ex{or3}] Either you will not have soup, or you will not have salad.
\item[\ex{or4}] You will have neither soup nor salad.
\end{earg}

We let $S_1$ mean that you get soup and $S_2$ mean that you get salad. Sentence \ref{or3} can be paraphrased in this way: ``Either \emph{it is not the case that} you get soup, or \emph{it is not the case that} you get salad.'' Translating this requires both disjunction and negation. It becomes $\enot S_1 \eor \enot S_2$.

Sentence \ref{or4} also requires negation. It can be paraphrased as, ``\emph{It is not the case that} either you get soup or you get salad.'' We need some way of indicating that the negation does not just negate the right or left disjunct, but rather negates the entire disjunction. In order to do this, we put parentheses around the disjunction: ``It is not the case that $(S_1 \eor S_2)$.'' This becomes simply $\enot (S_1 \eor S_2)$. Notice that the parentheses are doing important work here. The sentence $\enot S_1 \eor S_2$ would mean ``Either you will not have soup, or you will have salad.''

Something similar happens with negation and conjunction. Consider these sentences

\begin{earg}
\item[\ex{notand1}] You can't have soup and you can't have salad.
\item[\ex{notand2}] You can't have both soup and salad. 
\end{earg}

In sentence \ref{notand1}, the two parts of the sentence are negated individually. We would translate it into SL like this: $\enot S_1 \eand \enot S_2$. In sentence \ref{notand2}, the negation applies to soup and salad taken together. You are allowed to have soup only, or salad only. You just can't have both together. We would translate sentence \ref{notand2} like this: $\enot(S_1 \eand S_2)$. 

You can combine disjunction, conjunction, and negation to represent the exclusive or, as in this sentence. 

\begin{earg}
\item[\ex{or.xor}] You get either soup or salad, but not both.
\end{earg}

Remember on page \pageref{def:inclusive_or}, we said that the $\eor$ in SL represented an inclusive or. It said ``this or that or both.'' If we want to represent an exclusive or, we need to combine disjunction, conjuction and negation. We can break the sentence into two parts. The first part says that you get one or the other. We translate this as $(S_1 \eor S_2)$. The second part says that you do not get both. We can paraphrase this as ``It is not the case both that you get soup and that you get salad.'' Using both negation and conjunction, we translate this as $\enot(S_1 \eand S_2)$. Now we just need to put the two parts together. As we saw above, ``but'' can usually be translated as a conjunction. Sentence \ref{or.xor} can thus be translated as $(S_1 \eor S_2) \eand \enot(S_1 \eand S_2)$.


% ******************************************
%  * 		Recursive Syntax for  SL                		    *
% ******************************************
\label{recursive_syntax_for_SL}

\section{Recursive Syntax for SL} % I reworked this section to focus on the idea of recursive syntax

The previous two sections gave you a rough, informal sense of how to create sentences in SL. If I give you an English sentence like ``Grass is either green or brown,'' you should be able to write a corresponding sentence in SL: ``$A \eor B$.'' In this section we want to give a more precise definition of a sentence in SL.  When we defined sentences in English, we did so using the concept of truth: Sentences were units of language that can be true or false. (See page \pageref{def:statement}.) In SL, it is possible to define what counts as a sentence without talking about truth. Instead, we can just talk about the structure of the sentence. This is one respect in which a formal language like SL is more precise than a natural language like English.

\newglossaryentry{syntax}
{
name=syntax,
description={The structure of a bit of language, considered without reference to truth, falsity, or meaning.}
}

\newglossaryentry{semantics}
{
name=semantics,
description={The meaning of a bit of language is its meaning, including truth and falsity.}
}

The structure of a sentence in SL considered without reference to truth or falsity is called its syntax. More generally \textsc{\gls{syntax}} \label{def:syntax} refers to the study of the properties of language that are there even when you don't consider meaning. Whether a sentence is true or false is considered part of its meaning. In this chapter, we will be giving a purely syntactical definition of a sentence in SL.  The contrasting term is \textsc{\gls{semantics}} \label{def:semantics} the study of aspects of language that relate to meaning, including truth and falsity. (The word ``semantics'' comes from the Greek word for ``mark'')

\newglossaryentry{object language}
{
name=object language,
description={A language that is constructed and studied by logicians. In this textbook, the object \iflabelexists{part:quant_logic}{languages are SL and QL.}{language is SL.}}
}


\newglossaryentry{metalanguage}
{
name=metalanguage,
description={The language logicians use to talk about the object language. In this textbook, the metalanguage is English, supplemented by certain symbols like metavariables and technical terms like ``valid.''}
}

If we are going to define a sentence in SL just using syntax, we will need to carefully distinguish SL from the language that we use to talk about SL. When you create an artificial language like SL, the language that you are creating is called the \textsc{\gls{object language}}. \label{def:object_language} The language that we use to talk about the object language is called the \textsc{\gls{metalanguage}}. \label{def:metalanguage} Imagine building a house. The object language is like the house itself. It is the thing we are building. While you are building a house, you might put up scaffolding around it. The scaffolding isn't part of the the house. You just use it to build the house. The metalanguage is like the scaffolding. 

The object language in this chapter is SL. For the most part, we can build this language just by talking about it in ordinary English. However we will also have to build some special scaffolding that is not a part of SL, but will help us build SL. Our metalanguage will thus be ordinary English plus this scaffolding.

\newglossaryentry{metavariables}
{
name=metavariables,
description={A variable in the metalanguage that can represent any sentence in the object language.}
}



%rob: Paragraph on metavariables added.
An important part of the scaffolding are the \textsc{\gls{metavariables}} \label{def:metavariables} These are the fancy script letters we have been using in the characteristic truth tables for the connectives: \script{A}, \script{B}, \script{C}, etc. These are letters that can refer to any sentence in SL. They can represent sentences like $P$ or $Q$, or they can represent longer sentences, like $(((A \eor B) \eand G) \eif (P \eiff Q))$. Just as the sentence letters $A$, $B$, etc. are variables that range over any English sentence, the metavariables \script{A}, \script{B}, etc. are variables that range over any sentence in SL, including the sentence letters $A$, $B$, etc. 

As we said, in this chapter we will give a syntactic definition for ``sentence of SL.'' The definition itself will be given in mathematical English, the metalanguage. Table \ref{tab:basic_elements_of_SL} gives the basic elements of SL.


\begin{table}
\begin{mdframed}[style=mytablebox, userdefinedwidth=.75\textwidth]
\begin{tabu}{p{.3\linewidth}p{.4\linewidth}}
\underline{Element}& \underline{Symbols} \\ 
sentence letters & $A,B,C,\ldots,Z$ $A_1, B_1,Z_1,A_2,A_{25},J_{375},\ldots$\\
connectives & \enot,\eand,\eor,\eif,\eiff\\
parentheses&( , )\\\end{tabu}
\end{mdframed}
\caption{The basic elements of SL} \label{tab:basic_elements_of_SL}
\end{table}


Most random combinations of these symbols will not count as sentences in SL. Any random connection of these symbols will just be called a ``string'' or ``expression'' Random strings only become meaningful sentences when the are structured according to the rules of syntax. We saw from the earlier two sections that individual sentence letters,  like $A$ and $G_{13}$ counted as sentences. We also saw that we can put these sentences together using connectives so that  $\enot A$ and $\enot G_{13}$ is a sentence.  The problem is, we can't simply list all the different sentences we can put together this way, because there are infinitely many of them. Instead, we will define a sentence in SL by specifying the process by which they are constructed.

Consider negation: Given any sentence \script{A} of SL, $\enot\script{A}$ is a sentence of SL. It is important here that \script{A} is not the sentence letter $A$. Rather, it is a metavariable: part of the metalanguage, not the object language. Since \script{A} is not a symbol of SL, $\enot\script{A}$ is not an expression of SL. Instead, it is an expression of the metalanguage that allows us to talk about infinitely many expressions of SL: all of the expressions that start with the negation symbol. 


\newglossaryentry{sentence of SL}
{
name=sentence of SL,
description={A string of symbols in SL that can be built up using according to the recursive rules given on page \pageref{def:sentence_of_SL}.} % }
}



We can say similar things for each of the other connectives. For instance, if \script{A} and \script{B} are sentences of SL, then $(\script{A}\eand\script{B})$ is a sentence of SL. Providing clauses like this for all of the connectives, we arrive at the following formal definition for a \textsc{\gls{sentence of SL}}: \label{def:sentence_of_SL}

\begin{enumerate}
\item Every atomic sentence is a sentence.
\item If \script{A} is a sentence, then $\enot\script{A}$ is a sentence of SL.
\item If \script{A} and \script{B} are sentences, then $(\script{A}\eand\script{B})$ is a sentence.
\item If \script{A} and \script{B} are sentences, then $(\script{A}\eor\script{B})$ is a sentence.
\item If \script{A} and \script{B} are sentences, then $(\script{A}\eif\script{B})$ is a sentence.
\item If \script{A} and \script{B} are sentences, then $(\script{A}\eiff\script{B})$ is a sentence.
\item All and only sentences of SL can be generated by applications of these rules.
\end{enumerate}

We can apply this definition to see whether an arbitrary string is a sentence. Suppose we want to know whether or not $\enot \enot \enot D$ is a sentence of SL. Looking at the second clause of the definition, we know that $\enot \enot \enot D$ is a sentence \emph{if} $\enot \enot D$ is a sentence. So now we need to ask whether or not $\enot \enot D$ is a sentence. Again looking at the second clause of the definition, $\enot \enot D$ is a sentence \emph{if} $\enot D$ is. Again, $\enot D$ is a sentence \emph{if} $D$ is a sentence. Now $D$ is a sentence letter, an atomic sentence of SL, so we know that $D$ is a sentence by the first clause of the definition. So for a compound formula like $\enot \enot \enot D$, we must apply the definition repeatedly. Eventually we arrive at the atomic sentences from which the sentence is built up.

\newglossaryentry{recursive definition}
{
name=recursive definition,
description={A definition that defines a term by identifying base class and rules for extending that class. Also called an ``inductive definition.''}
}

Definitions like this are called recursive. \textsc{\Glspl{recursive definition}}\label{def:recursive_definition} begin with some specifiable base elements and define ways to indefinitely compound the base elements. Just as the recursive definition allows complex sentences to be built up from simple parts, you can use it to decompose sentences into their simpler parts. To determine whether or not something meets the definition, you may have to refer back to the definition many times. Recursive definitions are also sometimes called ``inductive definitions.''

\newglossaryentry{sentential logic}
{
name=sentential logic,
description={A system of logic in which statements can be defined using a recursive definition with only sentences in the base class.}
}


We are now in a position to define what it means for a system of logic to be a system of sentential logic. A \textsc{\gls{sentential logic}} \label{def:sentential_logic} is a system of logic in which statements can be defined using a recursive definition with only sentences in the base class. This book defines on system of sentential logic, which we call SL. Other books use other systems.


\newglossaryentry{scope}
{
name=scope,
description={The sentences that are joined by a connective. These are the sentences the connective was applied to when the sentence was assembled using a recursive definition.}
}

When you use a connective to build a longer sentence from shorter ones, the shorter sentences are said to be in the \textsc{\gls{scope}} \label{def:scope} of the connective. So in the sentence $(A \eand B) \eif C$, the scope of the connective $\eif$ includes $(A \eand B)$ and C. In the sentence $\enot(A \eand B)$ the scope of the $\enot$ is $(A \eand B)$. On the other hand, in the sentence $\enot A \eand B$ the scope of the $\enot$ is just A.

\newglossaryentry{main connective}
{
name=main connective,
description={The last connective that you add when you assemble a sentence using the recursive definition.}
}

The last connective that you add when you assemble a sentence using the recursive definition is the \textsc{\gls{main connective}} \label{def:main_connective} of that sentence. For example: The main logical operator of $\enot (E \eor (F \eif G))$ is negation, \enot. The main logical operator of $(\enot E \eor (F \eif G))$ is disjunction, \eor. The main connective of any sentence will have all the rest of the sentence in its scope.

\newglossaryentry{unique readability}
{
name=unique readability,
description={A property of formal languages which is present when each \iflabelexists{part:quant_logic}{well formed formula}{statement} is the product of a unique process of recursive construction.}
}

Because statement in our language is defined recursively, we can say it is ``uniquely readable'' \textsc{\Gls{term}}\label{def:term} is a property of formal languages which is present when each \iflabelexists{part:quant_logic}{well formed formula}{statement} can only be constructed in a single way. Every process of building up a sentence recursively yields a unique sentence, and every sentence is the product of a unique process of recursive definitions. This means that in an important sense our language SL is free of ambiguity, which is a key goal in the construction of any formal language. Every sentence in SL will have a unambiguous main connective and every connective in a sentence will have an unambiguous scope. This makes logicians happy.


%The recursive structure of sentences in SL will be important when we consider the circumstances under which a particular sentence would be true or false. The sentence $\enot \enot \enot D$ is true if and only if the sentence $\enot \enot D$ is false, and so on through the structure of the sentence until we arrive at the atomic components: $\enot \enot \enot D$ is true if and only if the atomic sentence $D$ is false. We will return to this point in the next chapter.
%restore when you restore the recursive part of chap. 3.

\subsection{Notational conventions}
\label{SLconventions}
A sentence like $(Q \eand R)$ must be surrounded by parentheses, because we might apply the definition again to use this as part of a more complicated sentence. If we negate $(Q \eand R)$, we get $\enot(Q \eand R)$. If we just had $Q \eand R$ without the parentheses and put a negation in front of it, we would have $\enot Q \eand R$. It is most natural to read this as meaning the same thing as $(\enot Q \eand R)$, something very different than $\enot(Q\eand R)$. The sentence $\enot(Q \eand R)$ means that it is not the case that both $Q$ and $R$ are true; $Q$ might be false or $R$ might be false, but the sentence does not tell us which. The sentence $(\enot Q \eand R)$ means specifically that $Q$ is false and that $R$ is true. As such, parentheses are crucial to the meaning of the sentence.

So, strictly speaking, $Q \eand R$ without parentheses is \emph{not} a sentence of SL. When using SL, however, we will often be able to relax the precise definition so as to make things easier for ourselves. We will do this in several ways.

First,  we understand that $Q \eand R$ means the same thing as $(Q \eand R)$. As a matter of convention, we can leave off parentheses that occur \emph{around the entire sentence}.

Second, it can sometimes be confusing to look at long sentences with many nested pairs of parentheses. We adopt the convention of using square brackets [ and ] in place of parentheses. There is no logical difference between $(P\eor Q)$ and $[P\eor Q]$, for example. The unwieldy sentence
$$(((H \eif I) \eor (I \eif H)) \eand (J \eor K))$$
could be written in this way:
$$\bigl[(H \eif I) \eor (I \eif H)\bigr] \eand (J \eor K)$$


Third, we will sometimes want to translate the conjunction of three or more sentences. For the sentence ``Alice, Bob, and Candice all went to the party,'' suppose we let $A$ mean ``Alice went,'' $B$ mean ``Bob went,'' and $C$ mean ``Candice went.'' The definition only allows us to form a conjunction out of two sentences, so we can translate it as $(A \eand B) \eand C$ or as $A \eand (B \eand C)$. There is no reason to distinguish between these, since the two translations are logically equivalent. There is no logical difference between the first, in which $(A \eand B)$ is conjoined with $C$, and the second, in which $A$ is conjoined with $(B \eand C)$.  So we might as well just write $A \eand B \eand C$. As a matter of convention, we can leave out parentheses when we conjoin three or more sentences.

Fourth, a similar situation arises with multiple disjunctions. ``Either Alice, Bob, or Candice went to the party'' can be translated as $(A \eor B) \eor C$ or as $A \eor (B \eor C)$. Since these two translations are logically equivalent, we may write $A \eor B \eor C$.

These latter two conventions only apply to multiple conjunctions or multiple  disjunctions. If a series of connectives includes both disjunctions and conjunctions, then the parentheses are essential; as with $(A \eand B) \eor C$ and $A \eand (B \eor C)$. The parentheses are also required if there is a series of conditionals or biconditionals; as with $(A \eif B) \eif C$ and $A \eiff (B \eiff C)$.

We have adopted these four rules as notational conventions, not as changes to the definition of a sentence. Strictly speaking, $A \eor B \eor C$ is still not a sentence. Instead, it is a kind of shorthand. We write it for the sake of convenience, but we really mean the sentence $(A \eor (B \eor C))$.

If we had given a different definition for a sentence, then these could count as sentences. We might have written rule 3 in this way: ``If \script{A}, \script{B}, $\ldots$ \script{Z} are sentences, then $(\script{A}\eand\script{B}\eand\ldots\eand\script{Z})$, is a sentence .'' This would make it easier to translate some English sentences, but would have the cost of making our formal language more complicated. We would have to keep the complex definition in mind when we develop truth tables and a proof system. We want a logical language that is expressively simple and allows us to translate easily from English, but we also want a formally simple language. Adopting notational conventions is a compromise between these two desires.


\practiceproblems
\setlength{\parindent}{0em}

\problempart Using the symbolization key given, translate each English-language sentence into SL.
\label{pr.monkeysuits}
\begin{ekey}
\item[M:] Those creatures are men in suits. 
\item[C:] Those creatures are chimpanzees. 
\item[G:] Those creatures are gorillas.
\end{ekey}
\begin{earg}
\item Those creatures are not men in suits. \answer{$\enot M$} 
\item Those creatures are men in suits, or they are not. \answer{$M \eor \enot M$} 
\item Those creatures are either gorillas or chimpanzees. \answer{$G \eor C$} 
\item Those creatures are not gorillas, but they are not chimpanzees either. \answer{$\enot G \eand \enot C$} 
\item Those creatures cannot be both gorillas and men in suits. \answer{$\enot(G \eand M)$} 
\item If those creatures are not gorillas, then they are men in suits \answer{$\enot G \eif M$} 
\item Those creatures are men in suits only if they are not gorillas. \answer{$M \eif \enot G$} 
\item Those creatures are chimpanzees if and only if they are not gorillas. \answer{$C \eiff G$} 
\item Those creatures are neither gorillas nor chimpanzees. \answer{$~(G \eor C).$} %See p.34, sentence 19, and p. 156
\item Unless those creatures are men in suits, they are either chimpanzees or they are gorillas. \answer{$M \eor (C \eor G)$} 
\end{earg}

%If					X
%only if				X
%if and only if		X
%but				X
%unless				X
%not both      		X
%neither nor			X

%added and changed problems to get a better distribution of kinds of problems. 

\problempart Using the symbolization key given, translate each English-language sentence into SL.
\begin{ekey}
\item[A:] Mister Ace was murdered.
\item[B:] The butler did it.
\item[C:] The cook did it.
\item[D:] The Duchess is lying.
\item[E:] Mister Edge was murdered.
\item[F:] The murder weapon was a frying pan.
\end{ekey}
\begin{earg}
\item Either Mister Ace or Mister Edge was murdered. % {\color{red} $A \eor E$}  \vspace{1ex}
\item If Mister Ace was murdered, then the cook did it. % {\color{red} $A \eif C$} \vspace{1ex}
\item If Mister Edge was murdered, then the cook did not do it. % {\color{red} $E \eif \enot C} \vspace{1ex}
\item Either the butler did it, or the Duchess is lying. % {\color{red} $B \eor D$} \vspace{1ex}
\item The cook did it only if the Duchess is lying. % {\color{red} $C \eif D$} \vspace{1ex}
\item If the murder weapon was a frying pan, then the culprit must have been the cook. % {\color{red} $F \eif C$} \vspace{1ex}
\item If the murder weapon was not a frying pan, then the culprit was neither the cook nor the butler. % {\color{red} $\enot F \eif \enot(C \or B) \vspace{1ex}
\item Mister Ace was murdered if and only if Mister Edge was not murdered. % {\color{red} $A \eiff \enot E$} \vspace{1ex}
\item The Duchess is lying, unless it was Mister Edge who was murdered. % {\color{red} $D \eor A$} \vspace{1ex}
\item Mister Ace was murdered, but not with a frying pan. % {\color{red} $A \eand \enot F$} \vspace{1ex}
\item The butler and the cook did not both do it. % {\color{red} $\enot(B \enad C)$} \vspace{1ex}
\item Of course the Duchess is lying! % {\color{red}$D$} \vspace{1ex}
\end{earg}

%If  			x
%only if             x
%if and only if   x
%but			x
%unless			x
%not both		x
%neither nor		x

%changed problems to get a better distribution of kinds of problems. 


\problempart Using the symbolization key given, translate each English-language sentence into SL.
\label{pr.avacareer}
\begin{ekey}
\item[E$_1$:] Ava is an electrician.
\item[E$_2$:] Harrison is an electrician.
\item[F$_1$:] Ava is a firefighter.
\item[F$_2$:] Harrison is a firefighter.
\item[S$_1$:] Ava is satisfied with her career.
\item[S$_2$:] Harrison is satisfied with his career.
\end{ekey}
\begin{earg}
\item Ava and Harrison are both electricians. %{\color{red} $E_1 \eand E_2$} \vspace{1ex}
\item If Ava is a firefighter, then she is satisfied with her career. %{\color{red} $F_1 \eif S_1$}  \vspace{1ex}
\item Ava is a firefighter, unless she is an electrician. %{\color{red} $F_1 \eor E_1$  \vspace{1ex}
\item Harrison is an unsatisfied electrician. %{\color{red} $E_2 \eand \enot S_2$}  \vspace{1ex}
\item Neither Ava nor Harrison is an electrician. %{\color{red} $\enot(E_1 \eor E_2)$}  \vspace{1ex}
\item Both Ava and Harrison are electricians, but neither of them find it satisfying. %{\color{red} $(E_1 \eand E_2) \eand \enot (S_1 \eor S_2)$} \vspace{1ex}
\item Harrison is satisfied only if he is a firefighter. %{\color{red} $S_2 \eif F_2$} \vspace{1ex}
\item If Ava is not an electrician, then neither is Harrison, but if she is, then he is too. %{\color{red} $(\enot E_1 \eif \enot E_2) \eand (E_1 \eif E_2)$} \vspace{1ex}
\item Ava is satisfied with her career if and only if Harrison is not satisfied with his. %{\color{red} $S_1 \eiff \enot S_2$} \vspace{1ex}
\item If Harrison is both an electrician and a firefighter, then he must be satisfied with his work. %{\color{red} $(E_2 \eand F_2) \eif S_2$} \vspace{1ex}
\item It cannot be that Harrison is both an electrician and a firefighter. %{\color{red} $\enot (E_2 \eand F_2)$} \vspace{1ex}
\item Harrison and Ava are both firefighters if and only if neither of them is an electrician. %{\color{red} $(F_1 \eand F_2) \eiff \enot (E_1 \eor E_2)$} \vspace{1ex}
\end{earg}

%If					x	
%only if				x
%if and only if		x
%but				x
%unless				x
%not both			x
%neither nor			x

\problempart Using the symbolization key given, translate each English-language sentence into SL.
\label{pr.jazzinstruments}
\begin{ekey}
\item[J$_1$:] John Coltrane played tenor sax.
\item[J$_2$:] John Coltrane played soprano sax.
\item[J$_3$:] John Coltrane played tuba
\item[M$_1$:] Miles Davis played trumpet
\item[M$_2$:]Miles Davis played tuba
\end{ekey}

\begin{earg}
\item John Coltrane played tenor and soprano sax. %{\color{red} $J_1 \eand J_2$} \vspace{1ex}
\item Neither Miles Davis nor John Coltrane played tuba. %{\color{red} $\enot(M_2 \eor J_3)$ or $\enot M_2 \eand \enot J_3$} \vspace{1ex}
\item John Coltrane did not play both tenor sax and tuba.  %{\color{red} $\enot(J_1 \eand J_3)$ or $\enot J_1 \eor \enotJ_3$} \vspace{1ex}
\item John Coltrane did not play tenor sax unless he also played soprano sax. %{\color{red} $\enot J_1 \eor J_2$} \vspace{1ex}
\item John Coltrane did not play tuba, but Miles Davis did. %{\color{red} $\enotJ_3 \eand M_2$} \vspace{1ex}
\item Miles Davis played trumpet only if he also played tuba. %{\color{red} $M_1 \eiff M_2$} \vspace{1ex}
\item If Miles Davis played trumpet, then John Coltrane played at least one of these three instruments: tenor sax, soprano sax, or tuba. %{\color{red} $M_1 \eif (J_1 \eor (J_2 \eor J_3))&} \vspace{1ex}
\item If John Coltrane played tuba then Miles Davis played neither trumpet nor tuba. %{\color{red} $J_3 \eif \enot(M_1 \eor M_2)$ or $J_3 \eif (\enot M_1 \eand \enot M_2)$  } \vspace{1ex}
\item Miles Davis and John Coltrane both played tuba if and only if Coltrane did not play tenor sax and Miles Davis did not play trumpet. %{\color{red} $(J_3 \eand M_2) \eiff \enotJ_1 & \enot M_1)$ or $(J_3 \eand M_2) \eiff \enot (J_1 \eor M_1)$} \vspace{1ex}
\end{earg}
%If					x					
%only if				x		
%if and only if		x
%but				x
%unless				x
%not both			x
%neither nor			x


\problempart
\label{pr.spies}
Give a symbolization key and symbolize the following sentences in SL. \\
%\begin{ekey}
%\item[A:] Alice is a spy
%\item [B:] Bob is a spy
%\item [C:] The code has been broken
%\item [D:] The German embassy is in an uproar
%\end{ekey}
\begin{earg}
\item Alice and Bob are both spies. % {\color{red}$A \eand B$ \vspace{1ex}}
\item If either Alice or Bob is a spy, then the code has been broken. %{\color{red} $(A \eor B) \eif C$ \vspace{1ex}}
\item If neither Alice nor Bob is a spy, then the code remains unbroken.%{\color{red}$\enot(A \eor B) \eif \enot C$ \vspace{1ex}}
\item The German embassy will be in an uproar, unless someone has broken the code.% {\color{red}$D \eor C$ \vspace{1ex}}
\item Either the code has been broken or it has not, but the German embassy will be in an uproar regardless. %{\color{red}$(C \eor \enot C) \eand D$ \vspace{1ex}}
\item Either Alice or Bob is a spy, but not both. %{\color{red}$(A \eor B) \eand \enot (A \eand B)$}
\end{earg}

%If
%only if
%if and only if
%but
%unless
%not both
%neither nor

\problempart Give a symbolization key and symbolize the following sentences in SL.
%\begin{ekey}
%\item[A:] Gregor plays first base
%\item[B:] The team will lose
%\item[C:] There is a miracle
%\item[D:] Gregor's mom will bake cookies.
%\end{ekey}

\begin{earg}
\item If Gregor plays first base, then the team will lose. %{\color{red} $A \eif B$ \vspace{1ex}}
\item The team will lose unless there is a miracle. %{\color{red}$B \eor C$ \vspace{1ex}}
\item The team will either lose or it won't, but Gregor will play first base regardless. % {\color{red}$(B \eor \enot B) \eand A$ \vspace{1ex}}
\item Gregor's mom will bake cookies if and only if Gregor plays first base.% {\color{red}$C \eiff A$ \vspace{1ex}}
\item If there is a miracle, then Gregor's mom will not bake cookies. %{\color{red} $C \eif \enot D$}
\end{earg}


\problempart
For each argument, write a symbolization key and translate the argument as well as possible into SL.
\begin{earg}
\item If Dorothy plays the piano in the morning, then Roger wakes up cranky. Dorothy plays piano in the morning unless she is distracted. So if Roger does not wake up cranky, then Dorothy must be distracted.
%{\color{red}
%\begin{ekey}
%\item[A:] Dorothy plays the piano in the morning
%\item[B:] Roger wakes up cranky
%\item[C:] Dorothy is distracted
%\end{ekey}

%begin{\earg}
%\item[1.] $A \eif B$
%\item[2.] $A \eor C$
%\item[$$\therefore$$] $\enot B \eif C$
%}
\item It will either rain or snow on Tuesday. If it rains on Tuesday, Neville will be sad. If it snows on Tuesday, Neville will be cold. Therefore, Neville will either be sad or cold on Tuesday.

%{\color{red}
%\begin{ekey}
%\item[A:]  It will rain on Tuesday
%\item[B:]  It will snow on Tuesday
%\item[C:]  Neville will be sad
%\item[D:]  Neville will be cold
%\end{ekey}

%\begin{earg}
%\item[1.]  $A \eor B$
%\item[2.]  $A \eif C$
%\item[3.] $B \eif D$
%\item[$$\therefore$$]  $C \eor D$
%\end{earg}
%}

\item If Zoog remembered to do his chores, then things are clean but not neat. If he forgot, then things are neat but not clean. Therefore, things are either neat or clean---but not both.
\end{earg}

%{\color{red}
%\begin{ekey}
%\item[A:] Zoog remembered to do his chores. 
%\item[B:] Things are clean 
%\item[C:] Things are neat %\end{ekey}

%\begin{earg}
%\item[1.]  $A \eif (B \eand \enot C)$
%\item[2.]  $\enot A \eif (\enot B \eand C)$
%\item[$\therefore$]  $(B \eor C) \eand \enot (B \eand C)$ 
%\end{earg}
%}

\problempart
For each argument, write a symbolization key and translate the argument as well as possible into SL. The part of the passage in italics is there to provide context for the argument, and doesn't need to be symbolized.
\begin{earg}
\item It is going to rain soon. I know because my leg is hurting, and my leg hurts if it’s going to rain. 

%{\color{red}
%\begin{ekey}
%\item[A:]  
%\item[B:]  
%\item[C:]  %\end{ekey}

%begin{\earg}
%\item[1.]  
%\item[2.]  
%\item[$\therefore$]  
%}

\item  \emph{Spider-man tries to figure out the bad guy’s plan.} If Doctor Octopus gets the uranium, he will blackmail the city. I am certain of this because if Doctor Octopus gets the uranium, he can make a dirty bomb, and if he can make a dirty bomb, he will blackmail the city.

%{\color{red}
%\begin{ekey}
%\item[A:]  
%\item[B:]  
%\item[C:]  %\end{ekey}

%begin{\earg}
%\item[1.]  
%\item[2.]  
%\item[$\therefore$]  
%}

\item \emph{A westerner tries to predict the policies of the Chinese government.} If the Chinese government cannot solve the water shortages in Beijing, they will have to move the capital. They don’t want to move the capital. Therefore they must solve the water shortage. But the only way to solve the water shortage is to divert almost all the water from the Yangzi river northward. Therefore the Chinese government will go with the project to divert water from the south to the north.       



%{\color{red}
%\begin{ekey}
%\item[A:]  
%\item[B:]  
%\item[C:]  %\end{ekey}

%begin{\earg}
%\item[1.]  
%\item[2.]  
%\item[$\therefore$]  
%}

\end{earg}




\problempart
\begin{earg}
\item Are there any sentences of SL that contain no sentence letters? Why or why not? % \vspace{1ex} \\ {\color{red} No, because the rules for creating sentences begin with sentence letters and then apply connectives and more sentence letters. There is no way to remove the sentence letters that you start with.} \vspace{2ex} 
\item In the chapter, we symbolized an \emph{exclusive or} using \eor, \eand, and \enot. How could you translate an \emph{exclusive or} using only two connectives? Is there any way to translate an \emph{exclusive or} using only one kind of connective? %\vspace{1ex} \\ {\color{red} The exclusive or (sometimes written xor) is true whenever the two sides of it have opposite truth values. This is the reverse of what the biconditional does. Thus you can represent the xor like this: \enot(A \eiff B). You can't get rid of any more connectives, though. If you had a single connective in the sentence, it would have to be equivalent on its own to the exclusive or, and none of our connectives work like that. Some systems do introduce a separate symbol for the exclusive or, often a plus sign: +.}
\end{earg}




%\solutions
%\problempart
%\label{pr.wiffSL}
%For each of the following: (a) Is it a wff of SL? (b) Is it a sentence of SL, allowing for notational conventions?
%\begin{earg}
%\item $(A)$
%\item $J_{374} \eor \enot J_{374}$
%\item $\enot \enot \enot \enot F$
%\item $\enot \eand S$
%\item $(G \eand \enot G)$
%\item $\script{A} \eif \script{A}$
%\item $(A \eif (A \eand \enot F)) \eor (D \eiff E)$
%\item $[(Z \eiff S) \eif W] \eand [J \eor X]$
%\item $(F \eiff \enot D \eif J) \eor (C \eand D)$
%\end{earg}


%%%%    Key term list
\section*{Key Terms}
\begin{multicols}{2}
\begin{sortedlist}
\sortitem{Sentence letter}{} 	
\sortitem{Symbolization key}{} 	
\sortitem{Atomic sentence}{}
\sortitem{Sentential connective}{}
\sortitem{Negation}{}
\sortitem{Conjunction}{}
\sortitem{Conjunct}{}
\sortitem{Disjunction}{}
\sortitem{Disjunct}{}
\sortitem{Conditional}{}
\sortitem{Antecedent}{}
\sortitem{Consequent}{}
\sortitem{Biconditional}{}
\sortitem{Syntax}{}
\sortitem{Semantics}{}
\sortitem{Object language}{}
\sortitem{Metalanguage}{}
\sortitem{Metavariables}{}
\sortitem{Sentence of SL}{}
\sortitem{Main connective}{}
\sortitem{Recursive definition}{}
\sortitem{Scope}{}
\sortitem{Nonlogical symbol}{}
\sortitem{Logical constant}{}
\sortitem{Exclusive or}{}
\sortitem{Inclusive or}{}
\sortitem{Necessary condition}{}
\sortitem{Sufficient condition}{}
\end{sortedlist}
\end{multicols}





