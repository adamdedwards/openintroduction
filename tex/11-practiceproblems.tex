% Practice Problems %%%%%%%%%%%%%%%

\section{Practice Problems}

\noindent  For each passage, (i) put the argument in canonical form and (ii) say whether it is valid or invalid.

\begin{longtabu}{X[1,l,p]X[.15,l,p]X[8.5,l,p]}

\textbf{Example}: & \multicolumn{2}{p{.9\linewidth}}{\textit{Monica is looking for her coworker} Jack is in his office. Jack's office is on the second floor. So, Jack is on the second floor.} \\
\\
\textbf{Answer}: & (i) & {\color{white}.} \vspace{-22pt} \begin{kormanize}
\premise{ Jack is in his office.}
\premise{ Jack's office is on the second floor.}
\conclusion{ Jack is on the second floor.}
\end{kormanize} \\
& (ii) & Valid
\end{longtabu}

\begin{itemize}
\item All dinosaurs are people, and all people are fruit. Therefore all dinosaurs are fruit.

%\answer{
\begin{enumerate}[label=(\roman*)]
\item {\color{white}.} \vspace{-13pt} \begin{kormanize}
\premise{ All dinosaurs are people}
\premise{ All people are fruit.}
\conclusion{[.4] All dinosaurs are fruit.}
\end{kormanize}
\item Valid
\end{enumerate}
}
%% F, F, conclusion last


\item All people are mortal. Socrates is mortal. Therefore all people are Socrates.
%\answer{
\begin{enumerate}[label=(\roman*)]
\item {\color{white}.} \vspace{-13pt} \begin{kormanize}
\premise{  All people are mortal.}
\premise{ Socrates is mortal.}
\conclusion{[.4]  All people are Socrates.}
\end{kormanize}
\item Invalid
\end{enumerate}}
%%Formal fallacy



\item All dogs are mammals. Therefore, Fido is a mammal, because Fido is a dog.

%\answer{
\begin{enumerate}[label=(\roman*)]
\item {\color{white}.} \vspace{-13pt} \begin{kormanize}
\premise{ All dogs are mammals.}
\premise{ Fido is a dog.}
\conclusion{[.4] Fido is a mammal.}
\end{kormanize}
\item Valid
\end{enumerate}
}
%%Made up, conclusion middle

\item Abe Lincoln must have been from France, because he was either from France or from Luxemborg, and we know was not from Luxemborg.
%\answer{
\begin{enumerate}[label=(\roman*)]
\item {\color{white}.} \vspace{-13pt} \begin{kormanize}
\premise{ Abe Lincoln was either from France or from Luxemborg.}
\premise{ Abe Lincoln was not from Luxemborg.}
\conclusion{[.4] Abe Lincoln was from France.}
\end{kormanize}
\item Valid
\end{enumerate}
}
%%F, F, conclusion first


\item If the world were to end today, then I would not need to get up tomorrow morning. I will need to get up tomorrow morning. Therefore, the world will not end today.

%\answer{
\begin{enumerate}[label=(\roman*)]
\item {\color{white}.} \vspace{-13pt} \begin{kormanize}
\premise{ If the world were to end today, then I would not need to get up tomorrow morning.}
\premise{ I will need to get up tomorrow morning.}
\conclusion{[.4]  The world will not end today.}
\end{kormanize}
\item Valid
\end{enumerate}
}
%%Made up

\item If the triceratops were a dinosaur, it would be extinct. Therefore, the triceratops is extinct, because the triceratops was a dinosaur.
%\answer{
\begin{enumerate}[label=(\roman*)]
\item {\color{white}.} \vspace{-13pt} \begin{kormanize}
\premise{  If the triceratops were a dinosaur, it would be extinct.}
\premise{ The triceratops was a dinosaur.}
\conclusion{[.4] The triceratops is extinct.}
\end{kormanize}
\item Valid
\end{enumerate}
}
%%T, T, conclusion middle

\item If George Washington was assassinated, he is dead. George Washington is dead. Therefore George Washington was assassinated.
%\answer{
\begin{enumerate}[label=(\roman*)]
\item {\color{white}.} \vspace{-13pt} \begin{kormanize}
\premise{  If George Washington was assassinated, he is dead.}
\premise{ George Washington is dead.}
\conclusion{[.4] George Washington was assassinated.}
\end{kormanize}
\item Invalid
\end{enumerate}
}
%% Formal fallacy
%
\item Jack prefers Pepsi to Coke. After all, about 52\% of people prefer Pepsi to Coke, and Jack is a person.

%\answer{
\begin{enumerate}[label=(\roman*)]
\item {\color{white}.} \vspace{-13pt} \begin{kormanize}
\premise{  About 52\% of people prefer Pepsi to Coke.}
\premise{ Jack is a person.}
\conclusion{[.4] Jack prefers Pepsi to Coke.}
\end{kormanize}
\item invalid
\end{enumerate}
}
%%Inductive, conclusion first

\item \textit{Steve thinks about the consequences of laziness.}  If I don't mow the lawn, it will become a haven for all kinds of exotic insect species. If the lawn becomes a haven for all kinds of exotic insect species, I will be protecting biodiversity. Therefore, if I don't mow the lawn, I'll be protecting biodiversity.

%\answer{
\begin{enumerate}[label=(\roman*)]
\item {\color{white}.} \vspace{-15pt} \begin{kormanize}
\premise{  If I don't mow the lawn, it will become a haven for insects.}
\premise{  If the lawn becomes a haven for insects, I will be protecting biodiversity}
\conclusion{[.6]  If I don't mow the lawn, I will be protecting biodiversity.}
\end{kormanize}
\item Valid. If the premises were true, the conclusion would have to be true.

In general, the argument has the form,

\begin{kormanize}
\premise{ If $A$, then $B$}
\premise{ If $B$, then $C$}
\conclusion{[.2] If $A$ then $C$.}
\end{kormanize}
which is valid.

\end{enumerate}
}

%conclusion last

\item \textit{A forest ranger is surveying the park} I can tell that bears have been down by the river, because there are tracks in the mud. Tracks like these are made by bears in almost every case.

%\answer{
\begin{enumerate}[label=(\roman*)]
\item {\color{white}.} \vspace{-13pt} \begin{kormanize}
\premise{  There are tracks in the mud.}
\premise{ Tracks like these are made by bears in almost every case.}
\conclusion{[.4]  Bears have been down by the river.}
\end{kormanize}
\item Invalid
\end{enumerate}
}
%%Inductive, conclusion first

\end{itemize}

%part B
\noindent For each passage, (i) put the argument in canonical form and (ii) say whether it is valid or invalid.
%\answer{Answers by Ben Sheredos}
\begin{itemize}
\item Cindy Lou Who lives in Whoville. You can tell because Cindy Lou Who is a Who, and all Whos live in Whoville.

%\answer{
	\begin{kormanize}
		\premise{ Cindy Lou Who is a Who.}
		\premise{ All Whos live in Whoville.}
		\conclusion{ Cindy Lou Who lives in Whoville.}
	\end{kormanize}
Valid}
%Made up, conclusion first

\item If Frog and Toad like each other, they are friends. Frog and Toad like each other. Therefore, Frog and Toad are friends.
%\answer{
	\begin{kormanize}
		\premise{ If Frog and Toad like each other, they are friends.}
		\premise{ Frog and Toad like each other.}
		\conclusion{ Frog and Toad are friends.}
	\end{kormanize}
Valid}
%Made up

\item If Cindy Lou Who is no more than two, then she is not five years old. Cindy Lou Who is not five. Therefore Cindy Lou Who is two or more.
%\answer{
	\begin{kormanize}
		\premise{ If Cindy Lou Who is no more than two, then she is not five years old.}
		\premise{ Cindy Lou Who is not five.}
		\conclusion{ Cindy Lou Who is two or more.}
	\end{kormanize}
Invalid. This starts out as a formal fallacy, affirming the consequent. Then it goes even further wrong by swapping ``no more than two'' for ``two or more."}
% Formal fallacy

\item \textit{Jack's suspicious house mate is in the kitchen} Jack has moved my leftover slice of pizza. Jack must have moved it, because Jack is the only person who has been in the house, and the pizza is no longer in the fridge.
%\answer{
	\begin{kormanize}
		\premise{ Jack is the only person who has been in the house, and the pizza is no longer in the fridge.}
		\conclusion{ Jack has moved my leftover slice of pizza.}
	\end{kormanize}
Alternatively:
	\begin{kormanize}
	\premise{ Jack is the only person who has been in the house.}
	\premise{ The pizza is no longer in the fridge.}
	\conclusion{ Jack has moved my leftover slice of pizza.}
	\end{kormanize}
Invalid. Maybe pets or robots can open the fridge? Or possibly, Jack opened the fridge, but did so with his cell phone is his hand and right at that moment received a funny video from a friend and left the fridge door open while he watched it, allowing the dog to steal the pizza. Not likely, admittedly, but if it could happen, the argument is \underline{not valid}.}
% Inductive, conclusion first

\item Jack is Smith's work colleague. So, Jack and Smith are friends.
%\answer{
	\begin{kormanize}
		\premise{ Jack is Smith's work colleague.}
		\conclusion{ Jack and Smith are friends.}
	\end{kormanize}
Invalid. Not all coworkers are friends.}

%Inductive

\item Abe Lincoln was either born in Illinois or he was once president. Therefore Abe Lincoln was born in Illinois, because he was never president.
%\answer{
	\begin{kormanize}
		\premise{ Lincoln was either born in Illinois or he was once president.}
		\premise{ Lincoln was never president.}
		\conclusion{ Lincoln was born in Illinois.}
	\end{kormanize}
Valid. Probably every statement here is false, but what matters is that IF the premises were true, the conclusion would have to be true.}
%F, T, conclusion middle

\item Politicians get a generous allowance for transportation costs. Enda Kenny is a politician. Therefore Kenny gets a generous transportation allowance.
%\answer{
	\begin{kormanize}
		\premise{ Politicians get a generous allowance for transportation costs.}
		\premise{ Enda Kenny is a politician.}
		\conclusion{ Kenny gets a generous transportation allowance.}
	\end{kormanize}
Valid. If the plural ``politicians'' is understood to mean ``all politicians'' rather than ``most", the inference is valid. If you wrote ``invalid'' and explained that you thought ``politicians'' only meant ``most politicians,'' that would be OK, as long as you made it clear. English is ambiguous like that.}

% Valid. If the plural ``politicians'' is understood to mean ``all politicians'' rather than ``most", the inference is valid.
%might as well be made up

\item Jones is taller than Bill, because Smith is taller than Jones and Bill is shorter than Smith.
%\answer{
	\begin{kormanize}
		\premise{ Smith is taller than Jones.}
		\premise{ Bill is shorter than Smith.}
		\conclusion{ Jones is taller than Bill.}
	\end{kormanize}
Invalid. Jones and Bill could be the same height.}

%Formal fallacy, conclusion first

\item If grass is green, then I am the pope. Grass is green. So, I am the pope.
%\answer{
	\begin{kormanize}
		\premise{ If grass is green, then I am the pope.}
		\premise{ Grass is green.}
		\conclusion{ I am the pope.}
	\end{kormanize}
Valid. IF premises are true, conclusion has to be.}

%F, F

\item Smith is paid more than Jack. They are both paid weekly. So, Smith has more money than Jack.
%\answer{
	\begin{kormanize}
		\premise{ Smith is paid more than Jack.}
		\premise{ Both Smith and Jones are paid weekly.}
		\conclusion{ Smith has more money than Jack.}
	\end{kormanize}
Invalid. There are sources of wealth other than what one is paid.}% Weak
\end{itemize}

%Part C
\noindent For each passage, (i) put the argument in canonical form and (ii) say whether it is valid or invalid.

\begin{itemize}
\item Jack is close to the pond. The pond is close to the playground. So, Jack is close to the playground.
%\answer{
\begin{enumerate}[label=(\roman*)]
\item {\color{white}.} \vspace{-13pt} \begin{kormanize}
\premise{  Jack is close to the pond.}
\premise{ The pond is close to the playground.}
\conclusion{[.4] Jack is close to the playground.}
\end{kormanize}
\item Invalid
\end{enumerate}
}
%% General fallacy

\item \textit{Jack is at work, and is unable to leave early} I have up to half an hour to get to the bank, because work ends at 5:00 and the bank closes at 5:30.
%\answer{
\begin{enumerate}[label=(\roman*)]
\item {\color{white}.} \vspace{-13pt} \begin{kormanize}
\premise{  Work ends at 5:00.}
\premise{ The bank closes at 5:30.}
\conclusion{[.4] I have up to half an hour to get to the bank.}
\end{kormanize}
\item Valid
\end{enumerate}
}
%% Made up, conclusion first.

\item Jack and Gill ate at Guadalajara restaurant earlier and both of them feel nauseated now. So, something they ate there is making them sick.
%\answer{
\begin{enumerate}[label=(\roman*)]
\item {\color{white}.} \vspace{-13pt} \begin{kormanize}
\premise{  Jack and Gill ate at Guadalajara restaurant earlier.}
\premise{  Jack and Gill feel nauseated now.}
\conclusion{[.4] Something they ate there is making them sick.}
\end{kormanize}
\item Invalid
\end{enumerate}
}
%% Inductive

\item Zhaoquing must be west of Huizhou, because Zhaoquing is west of Guangzhou, which is west of Huizhou.
%\answer{
\begin{enumerate}[label=(\roman*)]
\item {\color{white}.} \vspace{-13pt} \begin{kormanize}
\premise{   Zhaoquing is west of Guangzhou.}
\premise{  Guangzhou is west of Huizhou.}
\conclusion{[.4] Zhaoquing is west of Huizhou.}
\end{kormanize}
\item Valid
\end{enumerate}
}
%%T, T (?), conclusion first
%
\item \textit{Henry can't find his glasses. }I remember I had them when I came in from the car. So, they are in the house somewhere.
%\answer{
\begin{enumerate}[label=(\roman*)]
\item {\color{white}.} \vspace{-13pt} \begin{kormanize}
\premise{   Henry had his glasses when he came in from the car.}
\conclusion{[.4]  His glasses are in the house somewhere.}
\end{kormanize}
\item Invalid
\end{enumerate}}
% False dilemma
%% General fallacy

\item I was talking about tall John---the one who is over 6'4''---but Jack was talking about short John, who is at most 5'2''. So, we were talking about two different Johns.
%\answer{
\begin{enumerate}[label=(\roman*)]
\item {\color{white}.} \vspace{-13pt} \begin{kormanize}
\premise{ I was talking about tall John.}
\premise{   Jack was talking about short John.}
\conclusion{[.4]  We were talking about two different Johns.}
\end{kormanize}\item Valid
\end{enumerate}
}
%% Made up

\item Tomorrow's trip to Ensenada will take about 10 hours, because the last time I drove there from here it took 10 hours.
%\answer{
\begin{enumerate}[label=(\roman*)]
\item {\color{white}.} \vspace{-13pt} \begin{kormanize}
\premise{   The last time I drove to Ensenada from here it took 10 hours.}
\conclusion{[.4]  Tomorrow's trip to Ensenada will take about 10 hours.}
\end{kormanize}\item Invalid
\end{enumerate}
}
%%Induction, conclusion first.

\end{itemize}

\noindent For each passage, (i) put the argument in canonical form and (ii) say whether it is valid or invalid.
%\answer{Answers by Ben Sheredos}
\begin{itemize}
\item \textit{Monica is surveying the crowd that showed up for her talk} There must be at least 150 people here. That's how many people the auditorium holds, and every seat is full and people are beginning to sit on the stairs at the side.
%\answer{
	\begin{kormanize}
		\premise{ The auditorium holds 150 people.}
		\premise{ Every seat in the auditorium is full and people are beginning to sit on the stairs at the side.}
		\conclusion{ There are at least 150 people here.}
	\end{kormanize}
Valid}
% Made up, conclusion first

\item The fire bell in the building is ringing. There is sometimes a fire in the building when the alarm goes off. So, there is a fire.
%\answer{
	\begin{kormanize}
		\premise{ The fire bell in the building is ringing.}
		\premise{ There is sometimes a fire in the building when the alarm goes off.}
		\conclusion{ There is a fire (in the building).}
	\end{kormanize}
Invalid. ``Sometimes'' isn't ``always,'' so the conclusion is not necessarily true.}
% Inductive

\item I cannot drive on the motorways yet, because I just passed my driving test and anyone who passes can drive on the roads but not on the motorway for six months.
%\answer{
	\begin{kormanize}
		\premise{ I just passed my driving test.}
		\premise{ Anyone who passes can drive on the roads but not on the motorway for 6 months.}
		\conclusion{ I cannot drive on the motorways yet.}
	\end{kormanize}
Valid; it's implied pretty strongly that ``just passing'' means ``passed within the past 6 months.'' There is room for equivocation here, but it looks pretty solid.}

% Made up

\item Yesterday's the temperature reached 91 degrees Fahrenheit. Today it is 94. So, today is warmer than yesterday.
%\answer{
	\begin{kormanize}
		\premise{ Yesterday the temp. reached 91F.}
		\premise{ Today the temp. is 94F.}
		\conclusion{ Today is warmer than yesterday.}
	\end{kormanize}
Valid, unless the speaker inexplicably changes to a Celsius scale or something, but more likely the idea is that they just told you what scale they were using, and so they don't repeat it.}
% Made up

\item  My car is functioning well at the moment. So, all of the parts in my car are functioning well.
%\answer{
	\begin{kormanize}
		\premise{ My car is functioning well at the moment.}
		\conclusion{ All the parts of my car are functioning well.}
	\end{kormanize}
Probably invalid -- probably a fallacy of ``Composition and Division.'' Suppose the speaker added, as P2: ''I mean, the antenna fell off, so I can't listen to Jazz 98.3, but I'll fix that later'' We wouldn't jump on them and say ''\textit{A-HA!} so your car \textit{isn't} functioning well!''}
\item It has been sunny every day for the last five days. So, it will be sunny today.

%\answer{
	\begin{kormanize}
		\premise{ It has been sunny every day for the past 5 days.}
		\conclusion{ It will be sunny today.}
	\end{kormanize}
Not valid in the logical sense defined here. Five days in a row is no guarantee that the sixth day will be the same.}

% Inductive

\item Jack is in front of Gill. So, Gill is behind Jack.
%\answer{
	\begin{kormanize}
		\premise{ Jack is front of Gill.}
		\conclusion{ Gill is behind Jack.}
	\end{kormanize}
Valid}

% made up

\item \textit{Gill is returning home}: The door to my house is still locked. So, my possessions are still inside.
%\answer{
	\begin{kormanize}
		\premise{ The door to my house is still locked.}
		\conclusion{ My possessions are still inside.}
	\end{kormanize}
Invalid. Oh simple, naive Gil. A pro would definitely pick the lock, rob you blind, and lock the door on the way out so as not to arouse suspicions. By now your possessions have been pawned, and the thief is halfway to Vegas.}

%Inductive.

\end{itemize}

\section{Practice Problems}

\noindent For each inference, (i) say whether it is valid, strong, or weak and (ii) explain your answer.

\begin{longtabu}{p{.1\linewidth}p{.8\linewidth}}
\textbf{Example}: & The patient has a red rash covering the extremities and head, but not the torso. The only cause of such a rash is a deficiency in vitamin K. So, the patient must have a vitamin K deficiency. \\
\textbf{Answer}: & \noindent (i) Valid. \newline
\noindent (ii) The word ``only'' means it must be vitamin K deficiency.
\\
\end{longtabu}

\begin{itemize}

\item On 2003-06-19 in Norfolk, VA, a violent storm blew through and the power went out over much of the city. So, the storm caused the power to go out.

%\answer{\begin{enumerate}[label=(\roman*)]
\item Strong
\item Storms often cause power outages, but other things can cause them, too. In general, causal inferences are part of inductive reasoning, and are therefore at best strong.
\end{enumerate}
}

\item  All human beings are things with purple hair, and all things with purple hair have nine legs. Therefore, all human beings have nine legs.

%\answer{\begin{enumerate}[label=(\roman*)]
\item Valid.
\item You can see this by substituting in different things for ``purple hair'' and ``has nine legs.'' For instance, you could use ``mammals'' and ``has hair.'' There is no way to do this that will make the premises true and the conclusion false. So the argument is valid.
\end{enumerate}
}


\item  Elvis Presley was known as The King. Elvis had 18 songs reach \#1 in the Billboard charts. So, The King had 18 \#1 hits.

%\answer{\begin{enumerate}[label=(\roman*)]
\item Valid
\item This is an example of substituting different names for the same thing to create a valid argument.
\end{enumerate}}

\item Most philosophers are right-handed. Terence Irwin is a philosopher. So, he is right-handed.

%\answer{\begin{enumerate}[label=(\roman*)]
\item Either strong or weak depending on how much confidence you need. Definitely not valid.
\item The conclusion isn't necessarily true, so the inference is not valid. Is it strong or weak? If you read ``most'' as ``very many'' or something like that, it would be strong; if you read ``most'' as ``a majority'' (in the sense of 'somewhere between 51\% and 99\%' ), it would probably be weak. Advertisers sometimes use the vagueness of ``most'' to get you to feel that lots of people are buying a certain product or service, when in fact only a small majority is.
\end{enumerate}}

\item  Jack has purple hair, and purple toe nails. Hence, he has toe nails.

%\answer{\begin{enumerate}[label=(\roman*)]
\item Valid
\item If the color exists, the colored object has to exist
\end{enumerate}
}

\item  The Ohio State football team beat the Miami football team on 2003-01-03 for the college national championship. So, the Ohio State football team was the best team in college football in the 2002-2003 season.

%\answer{\begin{enumerate}[label=(\roman*)]
\item Strong or weak, depending on your background beliefs.
\item What are the chances that the non-best team would win the championship? If you think that a non-best time wins somewhat frequently, this inference would be weak. If, on the other hand, you think winning the championship is  a good way to judge the best team, you think the inference is strong.
\end{enumerate}
}

\item Willie Mosconi made almost all of the pool shots he took from 1940-1945. He took a bunch of shots in 1941. So, he made almost every shot he took in 1941.

%\answer{\begin{enumerate}[label=(\roman*)]
\item Strong
\item ``Almost all'' is fairly vague. So we are not sure how many missed shots we are talking about in the five year period. If there were all clustered in 1941, it is possible that the success rate for 1941 would no longer qualify as ``almost all,'' but this is unlikely.
\end{enumerate}
}

\item Some philosophers are people who are right-handed. Therefore, some people who are right-handed are philosophers.

%\answer{\begin{enumerate}[label=(\roman*)]
\item Valid.
\item The conclusion can't be false if the premise is true.
\end{enumerate}
}

\item U.S. President Obama firmly believed that Iran is planning a nuclear attack against Israel. We can conclude that Iran is planning a nuclear attack on Israel.

%\answer{\begin{enumerate}[label=(\roman*)]
\item Weak
\item Even if you think Obama's judgment is generally reliable here, a nuclear attack on Israel is an incredibly unlikely event. After all, this land is holy to Muslims, too. Extraordinary claims require extraordinary evidence, and I don't think any one's person's judgment is enough to go on here.
\end{enumerate}
}

\item Since the Spanish American War occurred before the American Civil War, and since the American Civil War occurred after the Korean War, it follows that the Spanish American War occurred before the Korean War.

%\answer{\begin{enumerate}[label=(\roman*)]
\item Weak.
\item If A is before B and C is before B, we know nothing about the relationship between B and C. A and C could be at the same time or either one before the other.
\end{enumerate}
}

\item There are exactly 10 humans in Carnegie Hall right now. Every human in Carnegie Hall right now has exactly ten legs. And, of course, no human in Carnegie Hall shares any legs with another human. Thus, there are at least 100 legs in Carnegie Hall right now.

%\answer{\begin{enumerate}[label=(\roman*)]
\item Valid
\item The conclusion follows from the fact that $10 \times 10 = 100$
\end{enumerate}
}

\item Amy Bishop is an evolutionary biologist (who shot a number of her colleagues to death in 2010). Evolutionary biology is incompatible with [Christian] scriptural teaching. Scriptural teaching is the only grounding for morality. Thus, evolutionary biologists are immoral.

%\answer{\begin{enumerate}[label=(\roman*)]
\item Weak
\item Many beliefs are incompatible with scriptural teaching on some point or other, but it's not clear that the people who hold those beliefs are immoral, unless morality is defined as following every single scriptural edict.
\end{enumerate}
}

%\item Corrupt people do harm to those around them, and no one intentionally wants to be done harm. Therefore, I [Socrates] did not corrupt my associates intentionally.
%
%%\answer{\underline{Valid} \\ The argument works if you think the premises are true always and everywhere. They seem like natural enough statements to make, but are they really perfectly and unequivocally true?}

\item Taxation means paying some of your earned income to the government. Some of this income is distributed to others. Paying so that someone else can benefit is slavery. Therefore, taxation is slavery.

%\answer{\begin{enumerate}[label=(\roman*)]
\item Valid.
\item Chain argument. However, the definition of slavery here is contentious, to say the least.
\end{enumerate}}

%\item Attempts have been made recently to carry bombs or bomb-making materials onto planes in the underwear and in other personal areas. These types of procedure provide a large measure of security against such attempts. Thus, flyers are required to submit to either a full-body scan or a thorough pat-down.
%
%%{\color{red}This problem shouldn't have been here, because it is really an explanation and not an argument.}
\end{itemize}


\noindent For each inference, (i) say whether it is valid, strong, or weak and (ii) explain your answer.
%\answer{Answers by Ben Sheredos}
\begin{itemize}
\item The sun has come up in the east every day in the past. So, the sun will come up in the east tomorrow.

%\answer{Invalid, but strong. That's a huge number of cases to generalize from, so the conclusion is very likely to be true, even if it is not \textit{certain}}

\item Jack's dog Jim will die before the age of 73 (in human years). After all, you are familiar with lots of dogs, and lots of different kinds of dogs, and any dog that is now dead died before the age of 73 (in human years).

%\answer{Invalid, but strong. That's a huge number of cases to generalize from, so the conclusion is very likely to be true. Don't get confused because you yourself are \textit{absolutely certain} that no dog will live to 73 in human years. The question is how well \textit{this argument} supports that claim.}

\item Any time the public receives a tax rebate, consumer spending increases, and the economy is stimulated. Since the public just received a tax rebate, consumer spending will increase.

%\answer{Valid. One could continue on to infer that the economy will be stimulated. The key is that the premise is that \textit{every time} there is a tax rebate, spending increases. This might be false, but \textit{if} it is true, the conclusion follows.}

\item  90\% of the marbles in the box are blue. So, about 90\% of the 20 I pick at random will be blue.

%\answer{Invalid, and pretty weak. This is a common error in statistical reasoning. The 20 marbles you pick out are not connected in any way, so if you pick out a non-blue marble, that doesn't increase the odds of you picking out a blue one next time. You might happen to pick nothing but marbles from the 10\% of marbles that are not blue.}

\item  According to the world-renowned physicist Stephen Hawking, quarks are one of the fundamental particles of matter. So, quarks are one of the fundamental particles of matter.

%\answer{This is a simple appeal to authority, which is a fallacy. The argument is weak. It might be supplemented to be made stronger (''Hawking is \textit{the world's foremost authority} on this topic, and he has put forth a convincing argument that quarks are a fundamental particle''). But as it is stated here, it's garbage. }

\item Sean Penn, Susan Sarandon and Tim Robbins are actors, and Democrats. So, most actors are Democrats.

%\answer{Invalid and weak. This is a very hasty generalization.}


\item  The President's approval rating has now fallen to 53\%, employment is at a 10 year high, and he is in charge of two foreign wars. He would not win another term in two years' time, if he were to run.

%\answer{Weak. There are some suppressed premises here, concerning how voters are likely to respond to the claims presumed in the premises. Only by filling them in could the argument be made strong.}


\item If Bill Gates owns a lot of gold then Bill Gates is rich, and Bill Gates doesn't own a lot of gold. So, Bill Gates isn't rich.

%\answer{Weak, since (\textit{a}) owning gold is not necessary for being rich, and (\textit{b}) Bill Gates is demonstrably rich even though (suppose) he owns little gold.}

\item All birds have wings, and all vertebrates have wings. So, all birds are vertebrates.

%\answer{ Weaksauce, even if the premises are true. Compare: ''All students in PHIL 10 are enrolled at UCSD, and all students in PHIL 163 are enrolled at UCSD. So all students in Phil 10 are in PHIL 163.'' Clearly wrongheaded.}

\item U.S. President Obama gave a speech in Berlin shortly after his inauguration. Berlin, of course, is where Hitler gave many speeches. Thus, Obama intends to establish a socialist system in the U.S.

%\answer{ Obviously Weak. Doing something as general as ``giving a speech'' in a place where Hitler gave a speech does not make one relevantly like Hitler to draw this conclusion.}

\item Einstein said that he believed in a god only in the sense of a pantheistic god equivalent with nature. Thus, there is no god in the Judeo-Christian sense.

%\answer{Weak. Appeal to authority. The argument concludes that something is true just because one person believed it; why trust Einstein on this? No support is provided. What, are we just supposed to be impressed because it was Einstein?}

\item The United States Congress has more members than there are days in the year. Thus, at least two members of the United States Congress celebrate their birthdays on the same day of the year.

%\answer{Valid. There are 365 days in a year. In any group of 366 people, at least 2 people have to share birthdays. (And don't try weaseling in that leap-year nonsense.)}

\item The base at Guantanamo ought to be closed. The continued incarceration of prisoners without any move to try or release them provides terrorist organizations with an effective recruiting tool, perhaps leading to attacks against Americans overseas.

%\answer{Pretty strong? The first sentence is the conclusion, and reasons are provided for thinking it is true. Maybe there are countervailing reasons that tell against...? Informal reasoning is tricky.}

\item Smith and Jones surveyed teenagers (13-19 years old) at a local mall and found that 94\% of this group owned a mobile phone. Therefore, they concluded, about 94\% of all teenagers own mobile phone.

%\answer{Weak. Why suppose an un-specified number of teenagers in one place are representative of that entire group of people, worldwide?}

\item Janice Brooks is an unfit mother. Her Facebook and Twitter records show that in the hour prior to the youngest son's accident she had sent 50 messages --- any parent who spends this much time on social media when they have kids is not giving them proper attention.

%\answer{Weak. Does Janice Brooks even live with her youngest son? Was there any reason she should've known his accident was impending? What, are people with children never allowed to have a day chatting with friends?}

\end{itemize}
