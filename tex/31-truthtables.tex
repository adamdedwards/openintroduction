\chapter{Truth Tables}\label{ch:truth_tables}
\markright{Chapter \ref{ch:truth_tables}: Truth Tables}

This chapter introduces a way of evaluating sentences and arguments of SL called the truth table method. As we shall see, the truth table method is \emph{semantic} because it involves one aspect of the meaning of sentences, whether those sentences are true or false. As we saw on page \pageref{def:semantics}, semantics is the study of aspects of language related to meaning, including truth and falsity. Although it can be laborious, the truth table method is a purely mechanical procedure that requires no intuition or special insight. When we get to Chapter \ref{chap:semantics_for_ql},we will provide a parallel semantic method for QL; however, this method will not be purely mechanical.

\section{Basic Concepts}

\newglossaryentry{logical constant}
{
name=logical constant,
description={A symbol whose meaning is fixed by a formal language. Sometimes these are just called ``logical symbols.'' They are contrasted with \textsc{non-logical symbols}.}
}

\newglossaryentry{nonlogical symbol}
{
name=nonlogical symbol,
description={A symbol whose meaning is not fixed by a formal language.}
}



In the previous chapter, we said that a formal language is built from two kinds of elements: logical constants and nonlogical symbols. The \textsc{\glspl{logical constant}}\label{def:logical_constant} have their meaning fixed by the formal language, while the \textsc{\glspl{nonlogical symbol}} \label{def:nonlogical_symbol} get their meaning in the symbolization key. The logical constants in SL are the sentential connectives and the parentheses, while the nonlogical symbols are the sentence letters.

\newglossaryentry{interpretation}
{
name=interpretation,
description={A correspondence between nonlogical symbols of the object language and elements of some other language or logical structure.}
}

When we assign meaning to the nonlogical symbols of a language using a dictionary, we say we are giving an ``interpretation'' of the language. More formally an \textsc{\gls{interpretation}\label{def:interpretation}} of a language is a correspondence between elements of the object language and elements of some other language or logical structure. The symbolization keys we defined in Chapter \ref{chap:SL} (p. \pageref{def:translation_key}) are one sort of interpretation. Fancier languages will have more complicated kinds of interpretations.

\newglossaryentry{truth value}
{
  name=truth value,
  description={The status of a statement with relationship to truth. For  this textbook, this means the status of a statement as true or false}
}

The truth table method will also involve giving an interpretation of sentences, but they will be much simpler than the translation keys we used in Chapter \ref{chap:SL}. We will not be concerned with what the individual sentence letters mean. We will only care whether they are true or false. In other words, our interpretations will assign \glspl{truth value} to the sentence letters. (See page \pageref{def:Truth_value}.)

\newglossaryentry{truth-functional connective}
{
name=truth-functional connective,
description={an operator that builds larger sentences out of smaller ones and fixes the truth value of the resulting sentence based only on the truth value of the component sentences.}
}

We can get away with only worrying about the truth values of sentence letters because of the way that the meaning of larger sentences is generated by the meaning of their parts. Any larger sentence of SL is composed of atomic sentences with sentential connectives. The truth value of the compound sentence depends only on the truth value of the atomic sentences that it comprises. In order to know the truth value of $D\iff E$, for instance, you only need to know the truth value of $D$ and the truth value of $E$. Connectives that work in this way are called truth functional. More technically, we define a \textsc{\gls{truth-functional connective}} \label{def:truth-functional_connective}as an operator that builds larger sentences out of smaller ones, and fixes the truth value of the resulting sentence based only on the truth value of the component sentences.

\newglossaryentry{truth assignment}
{
name=truth assignment,
description={A function that maps the sentence letters in SL onto truth values.}
}

Because all of the logical symbols in SL are truth functional, we can study the the semantics of SL looking only at truth and falsity. If we want to know about the truth of the sentence $A \land B$, the only thing we need to know is whether $A$ and $B$ are true. It doesn't actually matter what else they mean. So if $A$ is false, then $A \land B$ is false no matter what false sentence $A$ is used to represent. It could be ``I am the Pope'' or ``Pi is equal to 3.19.'' The larger sentence $A \land B$ is still false. So to give an interpretation of sentences in SL, all we need to do is create a truth assignment. A \textsc{\gls{truth assignment}} \label{def:truth_assignment} is a function that maps the sentence letters in SL onto our two truth values. In other words, we just need to assign Ts and Fs to all our sentence letters.

It is worth knowing that most languages are not built only out of truth functional connectives. In English, it is possible to form a new sentence from any simpler sentence $\mathcal{X}$ by saying ``It is possible that $\mathcal{X}$.'' The truth value of this new sentence does not depend directly on the truth value of $\mathcal{X}$. Even if $\mathcal{X}$ is false, perhaps in some sense $\mathcal{X}$ \emph{could} have been true---then the new sentence would be true. Some formal languages, called \emph{modal logics}, have an operator for possibility. In a modal logic, we could translate ``It is possible that $\mathcal{X}$'' as {\large $\diamond$}$\mathcal{X}$. However, the ability to translate sentences like these comes at a cost: The {\large $\diamond$} operator is not truth-functional, and so modal logics are not amenable to truth tables.

\section{Complete Truth Tables}

In the last chapter we introduced the characteristic truth tables for the different connectives. To put them all in one place, the truth tables for the connectives of SL are repeated in Table \ref{tab:CharacteristicTTs}. On the left is the truth table for negation, and on the right is the truth table for the other four connectives. Notice that the truth table for the negation is shorter than the other table. This is because there is only one metavariable here, $\mathcal{A}$, which can either be true or false. The other connectives involve two metavariables, which give us four possibilities of true and false. The columns to the left of the double line in these tables are called the reference columns. They just specify the truth values of the individual sentence letters. Each row of the table assigns truth values to all the variables. Each row is thus a truth assignment---a kind of interpretation---for that sentence. Because the full table gives all the possible truth assignments for the sentence, it gives all the possible interpretations of it.


\begin{table}
\begin{center}
\begin{longtabu}{cccc|c||c|c|c|c}
\multicolumn{1}{r||}{$\mathcal{A}$}&$\lnot\mathcal{A}$ & & $\mathcal{A}$ & $\mathcal{B}$ & $\mathcal{A}\land\mathcal{B}$ & $\mathcal{A}\lor\mathcal{B}$ & $\mathcal{A}\onlyif\mathcal{B}$ & $\mathcal{A}\iff\mathcal{B}$\\
\cline{1-2} \cline{4-9}
\multicolumn{1}{r||}{T}	&	F	&	& T & T & T & T & T & T \\
\multicolumn{1}{r||}{F}	&	T	&	& T & F & F & T & F & F \\
	                    &		&	& F & T & F & T & T & F \\
	                    &		&	& F & F & F & F & T & T \\
\end{longtabu}
\end{center}
\caption{The characteristic truth tables for the connectives of SL.}
\label{tab:CharacteristicTTs}
\end{table}

The truth table of sentences that contain only one connective is given by the characteristic truth table for that connective. So the truth table for the sentence $P \land Q$ looks just like the characteristic truth table for $\land$, with the sentence letters $P$ and $Q$ substituted in. The truth tables for more complicated sentences can simply be built up out of the truth tables for these basic sentences. Consider the sentence $(H\land I)\onlyif H$. This sentence has two sentence letters, so we can represent all the possible truth assignments using a four line truth table. We can start by writing out all the possible combinations of true and false for $H$ and $I$ in the reference columns. We then copy the truth values for the sentence letters and write them underneath the letters in the sentence.

\begin{center}
\begin{tabu}{c|c||@{\TTon}*{5}{c}@{\TToff}}
$H$ & $I$ & $(H$ & $\land$ & $I)$ & $\onlyif$ & $H$ \\
\hline
 T & T & T & & T & & T\\
 T & F & T & & F & & T\\
 F & T & F & & T & & F\\
 F & F & F & & F & & F
\end{tabu}
\end{center}

Now consider just one part of the sentence above, the subsentence $H\land I$. This is a conjunction $\mathcal{A}\land\mathcal{B}$ with $H$ as $\mathcal{A}$ and with $I$ as $\mathcal{B}$. $H$ and $I$ are both true on the first row. Since a conjunction is true when both conjuncts are true, we write a T underneath the conjunction symbol. We continue for the other three rows and get this:

\begin{center}
\begin{tabu}{c|c||ccccc}%{c|c||@{\TTon}*{5}{c}@{\TToff}}
\multicolumn{1}{r}{} &\multicolumn{1}{r}{} & \multicolumn{3}{c}{$(\mathcal{A}\land\mathcal{B})$} & & \\
\multicolumn{1}{r}{} &\multicolumn{1}{r}{} & \multicolumn{3}{c}{\downbracefill} & & \\
$H$	&	$I$	&	$(H$	&$\land$	&	$I)$	&	$\onlyif$	&	$H$\\
\hline
 T & T & T & \TTbf{T} & T & & T\\
 T & F & T & \TTbf{F} & F & & T\\
 F & T & F & \TTbf{F} & T & & F\\
 F & F & F & \TTbf{F} & F & & F\\
\end{tabu}
\end{center}

Next we need to fill in the final column under the conditional. The conditional is the main connective of the sentence, so the whole sentence is of the form $\mathcal{A}\onlyif\mathcal{B}$ with $(H \land I)$ as $\mathcal{A}$ and with $H$ as $\mathcal{B}$. So to fill the final column, we just need to look at the characteristic truth table for the conditional. For the first row, the sentence $(H \land I)$ is true and the sentence $H$ is also true. The truth table for he conditional tells us this means that the whole sentence is true. Filling out the rest of the column gives us this:

\begin{center}
\begin{tabu}{c|c||ccccc}%{c|c||@{\TTon}*{5}{c}@{\TToff}}
\multicolumn{1}{r}{} &\multicolumn{1}{r}{} & \multicolumn{3}{c}{\mathcal{A}} & \onlyif & \mathcal{B} \\
\multicolumn{1}{r}{} &\multicolumn{1}{r}{} & \multicolumn{3}{c}{\downbracefill}	& \downbracefill & \downbracefill \\
$H$ & $I$ & $(H$ & $\land$ & $I)$ & $\onlyif$ & $H$\\
\hline
 T & T & T & {T} & T &\TTbf{T} & T\\
 T & F & T & {F} & F &\TTbf{T} & T\\
 F & T & F & {F} & T &\TTbf{T} & F\\
 F & F & F & {F} & F &\TTbf{T} & F\\
\end{tabu}
\end{center}

The column of Ts underneath the conditional tells us that the sentence $(H \land I)\onlyif H$ is true regardless of the truth values of $H$ and $I$. They can be true or false in any combination, and the compound sentence still comes out true. It is crucial that we have considered all of the possible combinations. If we only had a two-line truth table, we could not be sure that the sentence was not false for some other combination of truth values.

In this example, the script letters over the table have just been there to indicate how the columns get filled in. We won't need them in the final product. Also, the reference columns are redundant with the columns under the individual sentence letters, so we can eliminate those as well. Most of the time, when you see truth tables, we will just write them out this way:
\begin{center}
\begin{tabu}{ccccc}
$(H$	&	\land	&	$I)$	& \onlyif	\tikz[overlay, shift={(0ex,-27pt)}, gray] \draw (0pt,0pt) ellipse (2ex and 44pt);			&$H$\\
\hline
T 		& 	{T} 	& 	T 		& T 	& T\\
T 		& 	{F} 	& 	F 		& T 	& T\\
F 		& 	{F} 	&	T 		& T 	& F\\
F 		& 	{F} 	& 	F 		& T 	& F
\end{tabu}
\end{center}
\label{tautology3.1}

The truth value of the sentence on each row is just the column underneath the \emph{main connective} (see p. \pageref{def:main_connective}) of the sentence, in this case, the column underneath the conditional.

\newglossaryentry{complete truth table}
{
name=complete truth table,
description={A table that gives all the possible interpretations for a sentence or set of sentences in SL.}
}

A \textsc{\gls{complete truth table}} \label{def:complete_truth_table} is a table that gives all the possible interpretations for a sentence or set of sentences in SL. It has a row for each possible assignment of T and F to all of the sentence letters. The size of the complete truth table depends on the number of different sentence letters in the table. A sentence that contains only one sentence letter requires only two rows, as in the characteristic truth table for negation. This is true even if the same letter is repeated many times, as in this sentence: $$[(C\iff C) \onlyif C] \land \lnot(C \onlyif C).$$ The complete truth table requires only two lines because there are only two possibilities: $C$ can be true, or it can be false. A single sentence letter can never be marked both T and F on the same row. The truth table for this sentence looks like this:
\begin{center}
\begin{tabu}{cccccccccc}%{c@{\TTon}*{13}{c}@{\TToff}}
[($C$	&\iff	&	$C)$	&	\onlyif	&	$C]$	&	\land	\tikz[overlay, shift={(-1ex,-12pt)}, gray] \draw (0pt,0pt) ellipse (2ex and 27pt);		&\lnot	&	$(C$	&	\onlyif	&	$C)$\\
\hline
	T 	&  T  	& 	T 		&  T  		& 	T 		&	F	&  F		& T 		&  T 		& T \\
	F 	&  T  	& 	F		&  F  		&	F 		&	F	&  F		& F 		&  T  		& F \\
\end{tabu}
\end{center}
\label{contradiction3.1}
Looking at the column underneath the main connective, we see that the sentence is false on both rows of the table; i.e., it is false regardless of whether $C$ is true or false.

A sentence that contains two sentence letters requires four lines for a complete truth table, as we saw above in the table for $(H \land I)\onlyif I$.

A sentence that contains three sentence letters requires eight lines, as in this example. Here the reference columns are included so you can see how to arrange the truth values for the individual sentence letters so that all the possibilities are covered.

\begin{center}
\begin{tabu}{c|c|c|@{\TTon}*{5}{c}@{\TToff}}
$M$	&	$N$	&	$P$	&	$M$	&	\land	\tikz[overlay, shift={(-1.25ex,-52pt)}, gray] \draw (0pt,0pt) ellipse (2ex and 66pt);			&	$(N$	&	\lor	&	$P)$\\
\hline
%           M        &     N   v   P
T		& T 		& T 		& T 		& T & T & T & T\\
T 		& T 		& F 		& T 		& T & T & T & F\\
T 		& F 		& T 		& T 		& T & F & T & T\\
T 		& F 		& F 		& T 		& F & F & F & F\\
F 		& T 		& T 		& F 		& F & T & T & T\\
F 		& T 		& F 		& F 		& F & T & T & F\\
F 		& F 		& T 		& F 		& F & F & T & T\\
F 		& F 		& F 		& F 		& F & F & F & F
\end{tabu}
\end{center}
\label{contingentsentence3.1}
From this table, we know that the sentence $M\land(N\lor P)$ might be true or false, depending on the truth values of $M$, $N$, and $P$.

A complete truth table for a sentence that contains four different sentence letters requires 16 lines. For five letters, 32 lines are required. For six letters, 64 lines, and so on. To be perfectly general: If a complete truth table has $n$ different sentence letters, then it must have $2^n$ rows.

By convention, the reference columns are filled in with the right most row alternating Ts and Fs. The next column over alternates sets of two Ts and two Fs. For the third column from the right, you have sets of four Ts and four Fs. This continues until you reach the leftmost column, which will always have the top have all Ts and the bottom half all Fs. This convention is completely arbitrary. There are other ways to be sure that all the possible combinations are covered, but everything is easier if we all stick to the same pattern.

\section*{Key Terms}
\begin{multicols}{2}
\begin{sortedlist}
\sortitem{Semantically contingent in SL}{}
\sortitem{Semantically logically equivalent in SL}{}
\sortitem{Semantically consistent in SL}{}
\sortitem{Semantically valid in SL}{}
\sortitem{Semantic contradiction in SL}{}
\sortitem{Semantic tautology in SL}{}
\sortitem{Complete truth table}{}
\sortitem{Truth assignment}{}
\sortitem{Truth-functional connective}{}
\sortitem{Nonlogical symbol}{}
\sortitem{Logical constant}{}
\sortitem{Interpretation}{}
\end{sortedlist}
\end{multicols}
