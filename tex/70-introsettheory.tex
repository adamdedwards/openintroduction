\chapter{Introduction to Set Theory}
\markright{Chapter \ref{ch:introsettheory}: Set Theory}
\label{ch:introsettheory}
\setlength{\parindent}{1em}

\section{What is a set?}

One of the major advantages that predicate logic has over propositional logic is that it enables us to \emph{quantify} over a collection of objects. Because propositional logic is only about complete sentences in a language, we don't have the ability there to talk about how some objects are related to other objects, or to talk about all the objects there are.

However, predicate logic still has some limitations. It is unwieldy in predicate logic to talk about \emph{specific} quantities. Consider the following first order sentence:
\[\exists x \forall y (Fx \eand (Fy \rightarrow x=y))\]
This sentence says that there is an object, x, that has property $F$ and that for any other object y, y has $F$ only if x and y are identical. This sentence is equivalent to the English sentence ``There is only one F.''

We can extend this approach to create greater and greater quantities, if we want. We can even say things like ``There are between three and seven Fs that are also Gs.'' However, these sentences get unwieldy extremely fast.

To help us talk about collections of objects in a more precise and concise way, we are going to introduce some new logical machinery. We are going to introduce the notion of a set.

\newglossaryentry{set}
{
name=Set,
description={A set is an abstract collection of objects that are unique and unordered.}
}

A \textsc{\gls{set}} is an abstract. mathematical object that represents a collection of objects. There are a few restrictions on sets that we'll discuss in a bit. First, let's just look at the notation for sets.

We notate sets with capital letters like $X$, $Y$, or $Z$. Some authors will also put sets in boldface, but we won't do that here. The contents of a set can be represented in one of two ways. The first way to represent the contents of a set is \emph{extensionally}, which means we simply list all the objects in the set one by one:
\begin{figure}
\[X=\{a,b,d,e,1,2,3\}\]
	\caption{A set defined extensionally.}
	\label{fig:setextension}
\end{figure}
We can put anything we like into a set, but each object is only listed once. We call the objects inside of a set the \textsc{\glspl{element}} of that set. We also always surround the elements of a set with curly braces to indicate that they are together inside of the set.

The other way to define a set is \emph{intensionally} which means we give an unambiguous condition that tells us whether or not an object is in the set. Suppose we want to create a set that contains all and only the even natural numbers. Such a set would be defined intensionally as follows:
\begin{figure}
\[\mathbb{E}=\{n \in \mathbb{N} | n \text{ is even.}\}\]
	\caption{A set defined intensionally.}
	\label{fig:setintension}
\end{figure}
We read this as ``The set of even integers is equal to the set that contains all natural numbers such that the natural number is even.'' That's pretty wordy, but the basic idea is straightforward. Unlike in an extensional definition where we simply list the elements of the set inside the curly braces, an intensional definition has two parts. On the lefthand side is the ``population'' or the collection of all the objects that \emph{could} be in the set. On the righthand side is the ``condition'' which an element from the population must satisfy in order to be in our set. So in the example above, we are drawing from the set of all the natural numbers and only putting the even ones into our new set.
