
%%%%%%%%%%%%%%%%% practice problems


\practiceproblems
\noindent\noindent\problempart Identify the main connective in the each sentence.

\begin{longtabu}{p{.1\linewidth}p{.9\linewidth}}
\textbf{Example}: & $(A \eif C) \eand \enot D$ \\
\textbf{Answer}: & $(A \eif C) \circled[gray, shape=circle]{\eand} \enot D$\\
\end{longtabu}



\begin{exercises}

\item \iflabelexists{showanswers}{$\circled[red, shape=circle]{\enot}(A \eor \enot B)$}{$\enot(A \eor \enot B) $}

\item \iflabelexists{showanswers}{$\enot(A \eor \enot B) \circled[red, shape=circle]{\eor} \enot(A \eand D)$}	{$\enot(A \eor \enot B) \eor \enot(A \eand D)$}

\item \iflabelexists{showanswers}{$[\enot(A \eor \enot B) \eor \enot (A \eand D)] \circled[red, shape=circle]{\eif} E$}{$[ \enot(A \eor \enot B) \eor \enot (A \eand D)] \eif E$}

\item \iflabelexists{showanswers}{$[(A \eif B) \eand C]$ \circled[red, shape=circle]{\eiff} $[A \eor (B \eand C)]$}{$[(A \eif B) \eand C] \eiff [A \eor (B \eand C)]$ }

\item \iflabelexists{showanswers}{\circled[red, shape=circle]{\enot} $\enot \enot [A \eor (B \eand (C \eor D))]$}{$\enot \enot \enot [A \eor (B \eand (C \eor D))] $}
\end{exercises}

\noindent\problempart Identify the main connective in the each sentence.
\begin{exercises}

\item $[(A \eiff B) \eand C] \eif D$  %$[(A \eiff B) \eand C] $\framebox[1.1\width]{\eif}$ D$

\item $[(D \eand (E \eand F)) \eor G] \eiff  \enot [A \eif (C \eor G)] $ %$[(D \eand (E \eand F)) \eor G] $\framebox[1.1\width]{\eiff}$  \enot [A \eif (C \eor G)] $

\item $\enot (\enot Z \eor \enot H) $ %\framebox[1.1\width]{\enot}  $(\enot Z \eor \enot H) $

\item $(\enot (P \eand S) \eiff G) \eand Y $ %$(\enot (P \eand S) \eiff G) $\framebox[1.1\width]{\eand} $ Y $

\item $(A \eand (B \eif C)) \eor \enot D	$  %$(A \eand (B \eif C)) $\framebox[1.1\width]{\eor} $\enot D	$

\end{exercises}

\noindent\problempart Assume A, B, and C are true and X, Y, and Z are false and evaluate the truth of the each sentence by writing a one-line truth table.

\begin{longtabu}{p{.1\textwidth}p{.01\textwidth}p{.01\textwidth}p{.01\textwidth}p{.01\textwidth}p{.01\textwidth}p{.01\textwidth}p{.01\textwidth}p{.01\textwidth}p{.01\textwidth}}
\textbf{Example}: & \multicolumn{9}{p{.9\textwidth}}{$(A \eand \enot X) \eiff (B \eor Y)$ }\\
\textbf{Answer}: & (A &\eand &\enot& X)& \eiff	\tikz[overlay, shift={(-1ex,-6pt)}, gray] \draw (0pt,0pt) ellipse (2ex and 18pt); & (B& \eor& Y)&\\
\cline{2-9}
& T  &    T    &  T    &  F&	T	&	 T&	T   & F&\\
\tabuphantomline
\end{longtabu}

\begin{exercises}
\item $\enot ((A \eand B) \eif X) $

\answer{
\begin{tabu}{c c c c c c}
\enot \tikz[overlay, shift={(-1ex,-6pt)}, red] \draw (0pt,0pt) ellipse (2ex and 18pt);	 &((A &	\eand&	B)&	\eif&	X) \\
\cline{1-6}
 T    &	T&	T&	T&	F&		F \\
\end{tabu}
}

\item $(Y \eor Z) \eiff	 (\enot X \eiff B)$

\answer{
\begin{tabu}{cccccccc}
(Y	&\eor &	Z)	& \eiff \tikz[overlay, shift={(-1ex,-6pt)}, red] \draw (0pt,0pt) ellipse (2ex and 18pt);	&(\enot	&X	&\eiff	&B)\\
\cline{1-8}
F &	F &	F	&	F	&	T &	F &  	T  &	T\\
\end{tabu}
}

\item $[(X \eif A) \eor (A \eif X)] \eand Y$

\answer{
\begin{tabu}{ccccccccc}
[(X &\eif& A)& \eor& (A& \eif& X)] &\eand	\tikz[overlay, shift={(-1ex,-6pt)}, red] \draw (0pt,0pt) ellipse (2ex and 18pt);	& Y	\\
\cline{1-9}
F&  T&  T&   T&  T&  F& F &   F&  F\\
\end{tabu}
}

\item $(X  \eif  A) \eor  (A \eif X)$

\answer{
\begin{tabu}{ccccccc}
(X & \eif & A) &\eor \tikz[overlay, shift={(-1ex,-6pt)}, red] \draw (0pt,0pt) ellipse (2ex and 18pt);	& (A & \eif& X)\\
\cline{1-7}
F&  T&  T&   T&  T&  F& F\\
\end{tabu}
}

\item $[A \eand (Y \eand Z)] \eor A $

\answer{
\begin{tabu}{ccccccc}
[A& \eand &(Y &\eand &Z)]& \eor	\tikz[overlay, shift={(-1ex,-6pt)}, red] \draw (0pt,0pt) ellipse (2ex and 18pt);	& A \\
\cline{1-7}
T&  F&  F&  F& F&   T& T\\
\end{tabu}
}
\end{exercises}


\noindent\problempart
Assume A, B, and C are true and X, Y, and Z are false and evaluate the truth of the each sentence by writing a one-line truth table..


\begin{exercises}
\item $\enot  \enot  (\enot  \enot  \enot A  \eor  X) $

%\begin{tabular}{c|c|ccccccc}
%\cline{2-2}
%1. &	\enot & \enot & (\enot & \enot & \enot &A & \eor & X) \\
%&F	&T	& F& T& F& T & F & F \\
%\cline{2-2}
%\end{tabular}
%\vspace{1em}

\item $(A \eif B) \eif X$

%\begin{tabular}{cccc|c|c}
%\cline{5-5}
%2. &	(A& \eif& B)& \eif& X	\\
%&T &T&T&F&F\\
%\cline{5-5}
%\end{tabular}

\item $((A \eor B) \eand (C \eiff X)) \eor Y$

%\begin{tabular}{cccccccc|c|c}
%\cline{9-9}
%3.&	((A &\eor& B)& \eand& (C& \eiff& X))& \eor& Y	\\
%&T&T&T&F&T&F&F&F&F\\
%\cline{9-9}
%\end{tabular}

\item $(A \eif 	B)	\eor 	(X 	\eand 	(Y 	\eand 	Z))$

%\begin{tabular}{cccc|c|ccccc}
% \cline{5-5}
%4.&	(A&	\eif &	B)&	\eor &	(X &	\eand &	(Y &	\eand &	Z)) \\
%&	T &	T &	T &	T &	F &	F &	F &	F &	F \\
%\cline{5-5}
%\end{tabular}

\item $((A  	\eor 	X) \eif Y) 	\eand B $

%\begin{tabular}{cccccc|c|c}
%\cline{7-7}
%5.&	((A  &	\eor &	X) &	\eif &	Y) &	\eand &	B \\
%&	T &	T &	F &	F &	F &	F &	T \\
%\cline{7-7}
%\end{tabular}

\end{exercises}

\noindent\problempart Write complete truth tables for the following sentences and mark the column that represents the possible truth values for the whole sentence.

\begin{longtabu}{p{.1\linewidth}p{.9\linewidth}}
\textbf{Example}: & $D \eif (D \eand (\enot F \eor F))$ \\
\textbf{Answer}: & \vspace{-8pt} \begin{tabular}[t]{cccccccc}
D 	&\eif 	\tikz[overlay, shift={(-1ex,-20pt)}, gray] \draw (0pt,0pt) ellipse (2ex and 33pt);	&(D 	&\eand 	& (\enot	& F 	& \eor 	&  F))\\
\cline{1-8}
T	&	T	&	T	&	T		&	F	  	&	T	&	T		& T	\\
T	&	T	&	T	&	T		&	T	  	&	F	&	T		& F	\\
F	&	T	&	F	&	F		&	F	  	&	T	&	T		& T	\\
F	&	T	&	F	&	F		&	T	  	&	F	&	T		& F 	\\
\end{tabular}\\
\end{longtabu}



\begin{exercises}

\item $\enot (S \eiff (P \eif S))$

\answer{
\begin{longtabu}{cccccc}
\enot \tikz[overlay, shift={(-1ex,-27pt)}, red] \draw (0pt,0pt) ellipse (2ex and 44pt);	&	(S 	&	\eiff	&	(P 	&	\eif	&	S))	\\
\cline{1-6}
F 		&	T	&	T	&	T	&	T	&	T	\\
F 		&	T	&	T	&	F	&	T	&	T	\\
F 		&	F	&	T	&	T	&	F	&	F	\\
T 		&	F	&	F	&	F	&	T	&	F	\\
\end{longtabu}
}

 \item $\enot [(X \eand Y) \eor (X \eor Y)]$

\answer{
\begin{longtabu}{cccccccc}
\enot	\tikz[overlay, shift={(-1ex,-27pt)}, red] \draw (0pt,0pt) ellipse (2ex and 44pt);	&	 [(X 	&	\eand& 	Y) 	&	\eor 	&	(X 	&	\eor 	&	Y)] \\
\cline{1-8}
F	&	T	&	T	&	T	&	T	&	T	&	T	&	T	\\
F	&	T	&	F	&	F	&	T	&	T	&	T	&	F	\\
F	&	F	&	F	&	T	&	T	&	F	&	T	&	T	\\
T	&	F	&	F	&	F	&	F	&	F	&	F	&	F	\\
\end{longtabu}
}

\item $(A \eif B) \eiff (\enot B\eiff \enot A)$

\answer{
\begin{longtabu}{ccccccccc}
(A 	&	\eif	&	B)	&	 \eiff 	\tikz[overlay, shift={(-1ex,-27pt)}, red] \draw (0pt,0pt) ellipse (2ex and 44pt);	&	(\enot&	B 	&	\eiff 	&	 \enot 	& 	 A) \\
\cline{1-9}
T	&	T	&	T	&	T		&	F	 &	T	&	T	&	F		&	T	\\
T	&	F	&	F	&	T		&	T	 &	F	&	F	&	F		&	T	\\
F	&	T	&	T	&	F		&	F	 &	T	&	F	&	T		&	F	\\
F	&	T	&	F	&	T		&	T	 &	F	&	T	&	T		&	F	\\
\end{longtabu}
}

\item $[C \eiff (D \eor E)] \eand \enot C$

\answer{
\begin{longtabu}{cccccccc}
[C 	&	\eiff 	&	(D 	&	\eor 	&	E)] 	&	\eand 	\tikz[overlay, shift={(-1ex,-52pt)}, red] \draw (0pt,0pt) ellipse (2ex and 77pt);	&	 \enot 	&	 C \\
\cline{1-8}
T	&	T	&	T	&	T	&	T	&	F		&	F		&	T	\\
T	&	T	&	T	&	T	&	F	&	F		&	F		&	T	\\
T	&	T	&	F	&	T	&	T	&	F		&	F		&	T	\\
T	&	F	&	F	&	F	&	F	&	F		&	F		&	T	\\
F	&	F	&	T	&	T	&	T	&	F		&	T		&	F	\\
F	&	F	&	T	&	T	&	F	&	F		&	T		&	F	\\
F	&	F	&	F	&	T	&	T	&	F		&	T		&	F	\\
F	&	T	&	F	&	F	&	F	&	T		&	T		&	F	\\
\end{longtabu}
}

\item $\enot(G \eand (B \eand H)) \eiff (G \eor (B \eor H))$

\answer{
\begin{longtabu}{cccccccccccc}
\enot&	(G 	&\eand &	(B 	&	 \eand 	&	 H))	&	\eiff \tikz[overlay, shift={(-1ex,-52pt)}, red] \draw (0pt,0pt) ellipse (2ex and 77pt); 	&	(G 	& \eor 	& (B 	& \eor	& H))	\\
\cline{1-12}
F	   &	T	&	  T &	T	&	T		&	T	&	F	&	T	&	T	&	T	&	T	&	T	\\
T	   &	T	&	  F &	T	&	F		&	F	&	T	&	T	&	T	&	T	&	T	&	F	\\
T	   &	T	&	 F  &	F	&	F		&	T	&	T	&	T	&	T	&	F	&	T	&	T	\\
T	   &	T	&	 F  &	F	&	F		&	F	&	T	&	T	&	T	&	F	&	F	&	F	\\
T	   &	F	&	F   &	T	&	T		&	T	&	T	&	F	&	T	&	T	&	T	&	T	\\
T	   &	F	&	F   &	T	&	F		&	F	&	T	&	F	&	T	&	T	&	T	&	F	\\
T	   &	F	&	F   &	F	&	F		&	T	&	T	&	F	&	T	&	F	&	T	&	T	\\
T	   &	F	&	F   &	F	&	F		&	F	&	F	&	F	&	F	&	F	&	F	&	F	\\
\end{longtabu}
}

\end{exercises}

\noindent\problempart Write complete truth tables for the following sentences and mark the column that represents the possible truth values for the whole sentence.

\begin{exercises}

\item	$(D \eand \enot D) \eif G $

%\vspace{1em}

%\begin{tabular}{ccccc|c|c}
%\cline{6-6}
%1.	&	(D 	&	 \eand 	& 	 \enot	&	 D) 	&	 \eif 	&	 G \\
%	&	T	&	F		&	F		&	T	&	T	&	T	\\
%	&	T	&	F		&	F		&	T	&	T	&	F	\\
%	&	F	&	F		&	T		&	F	&	T	&	T	\\
%	&	F	&	F		&	T		&	F	&	T	&	F	\\
%\cline{6-6}
%\end{tabular}
%\vspace{1em}


\item	$(\enot P \eor \enot M) \eiff M $

%\begin{tabular}{cccccc|c|c}
%\cline{7-7}
%2.	&	(\enot 	&	P 	&	\eor 	&	\enot 	& 	 M) 	& 	\eiff 	&	 M \\
%	&	F		&	T	&	F	&	F		&	T	&	T	&	T	\\
%	&	F		&	T	&	T	&	T		&	F	&	F	&	F	\\
%	&	T		&	F	&	T	&	F		&	T	&	T	&	T	\\
%	&	T		&	F	&	T	&	T		&	F	&	T	&	F	\\
%\cline{7-7}
%\end{tabular}
%\vspace{1em}



\item	$\enot \enot (\enot A \eand \enot B)  $

%\begin{tabular}{c|c|cccccc}
%\cline{2-2}
%3.	&	\enot		&	 \enot 	&	(\enot 	& 	 A 	& \eand 	& 	\enot 	&	 B)  \\
%	&	F		&	T		&	F		&	T	&	F	&	F		&	T	\\
%	&	F		&	T		&	F		&	T	&	F	&	T		&	F	\\
%	&	F		&	T		&	T		&	F	&	F	&	F		&	T	\\
%	&	T		&	F		&	T		&	F	&	T	&	T		&	F	\\
%\cline{2-2}
%\end{tabular}
%\vspace{1em}



\item 	$[(D \eand R) \eif I] \eif \enot(D \eor R) $

%\begin{tabular}{cccccc|c|cccc}
%\cline{7-7}
%4.	&	[(D 	& 	 \eand 	& 	 R)	& 	\eif 	&	I] 	&	\eif 	&	 \enot 	&	(D 	&	 \eor 	& R) \\
%	&	T	&	T		&	T	&	T	&	T	&	F	&	F		&	T	&	T		&T	\\
%	&	T	&	T		&	T	&	F	&	F	&	T	&	F		&	T	&	T		&T	\\
%	&	T	&	F		&	F	&	T	&	T	&	F	&	F		&	T	&	T		&F	\\
%	&	T	&	F		&	F	&	T	&	F	&	F	&	F		&	T	&	T		&F	\\
%	&	F	&	F		&	T	&	T	&	T	&	F	&	F		&	F	&	T		&T	\\
%	&	F	&	F		&	T	&	T	&	F	&	F	&	F		&	F	&	T		&T	\\
%	&	F	&	F		&	F	&	T	&	T	&	T	&	T		&	F	&	F		&F	\\
%	&	F	&	F		&	F	&	T	&	F	&	T	&	T		&	F	&	F		&F	\\
%\cline{7-7}
%\end{tabular}
%
%\vspace{1em}


\item	$\enot [(D \eiff O) \eiff A] \eif (\enot D \eand O) $

%\begin{tabular}{ccccccc|c|cccc}
%\cline{8-8}
%5.	&	\enot 	&	[(D 	&	\eiff 	&	O) 	&	\eiff 	&	 A]	& 	\eif 	 &	(\enot 	& 	D 	 & 	 \eand &O) \\
%	&	F		&	T	&	T	&	T	&	T	&	T	&	T	&	F		&	T	&	F	&T	\\
%	&	T		&	T	&	T	&	T	&	F	&	F	&	F	&	F		&	T	&	F	&T	\\
%	&	T		&	T	&	F	&	F	&	F	&	T	&	F	&	F		&	T	&	F	&F	\\
%	&	F		&	T	&	F	&	F	&	T	&	F	&	T	&	F		&	T	&	F	&F	\\
%	&	T		&	F	&	F	&	T	&	F	&	T	&	T	&	T		&	F	&	T	&T	\\
%	&	F		&	F	&	F	&	T	&	T	&	F	&	T	&	T		&	F	&	T	&T	\\
%	&	F		&	F	&	T	&	F	&	T	&	T	&	T	&	T		&	F	&	F	&F	\\
%	&	T		&	F	&	T	&	F	&	F	&	F	&	T	&	T		&	F	&	F	&F	\\
%\cline{8-8}
%\end{tabular}
%\vspace{1em}


\end{exercises}


% *********************************************
% *   Using Truth Tables								*
% *********************************************

\section{Using Truth Tables}

Because truth table show all the possible interpretations of a sentence or set of sentences we can use them to explore the logical properties we first introduced in Chapter \ref{chap:whatisformallogic}.

\subsection{Tautologies, contradictions, and contingent sentences}
We defined a tautology as a statement that must be true as a matter of logic, no matter how the world is (p. \pageref{def:tautology}). A statement like ``Either it is raining or it is not raining'' is always true, no matter what the weather is like outside. Something similar goes on in truth tables. With a complete truth table, we consider all of the ways that the world might be. Each line of the truth table corresponds to a way the world might be. This means that if the sentence is true on every line of a complete truth table, then it is true as a matter of logic, regardless of what the world is like.

We can use this fact to create a test for whether a sentence is a tautology: if the column under the main connective of a sentence is a T on every row, the sentence is a tautology. We already have seen an example of this. On page \pageref{tautology3.1} that the sentence $(H \eand I)\eif H$ had only T's under its main connective, so it is a tautology.

Not every tautology in English will correspond to a tautology in SL. The sentence ``All bachelors are unmarried'' is a tautology in English, but we cannot represent it as a tautology in SL, because it just translates as a single sentence letter, like $B$. On the other hand, if something is a tautology in SL, it will also be a tautology in English. No matter how you translate $A \eor \enot A$, if you translate the $A$s consistently, the statement will be a tautology.

\newglossaryentry{semantic tautology in SL}
{
name=semantic tautology in SL,
description={A statement that has only Ts in the column under the main connective of its complete truth table.}
}

\label{semantic_definitions_in_SL}
Rather than thinking of complete truth tables as an imperfect test for the English notion of a tautology, we can define a separate notion of a tautology in SL based on truth tables. A statement is a \textsc{\gls{semantic tautology in SL}} \label{def:semantic_tautology_in_sl} if and only if the column  under the main connective in the complete truth table for the sentence contains only Ts. This is the semantic definition of a tautology in SL, because it uses truth tables. Later we will create a separate, syntactic definition and show that it is equivalent to the semantic definition. We will be doing the same thing for all the concepts defined in this section.

\newglossaryentry{semantic contradiction in SL}
{
name=semantic contradiction in SL,
description={A statement that has only Fs in the column under the main connective of its complete truth table.}
}

We defined a contradiction as a sentence that is false no matter how the world is (p. \pageref{def:contradiction}). This means we can define a \textsc{\gls{semantic contradiction in SL}} \label{def:semantic_contradiction_in_sl} as a sentence that has only Fs in the column under them main connective of its complete truth table. We saw on page \pageref{contradiction3.1} that the sentence $[(C\eiff C) \eif C] \eand \enot(C \eif C)$ was a contradiction in this sense. As with the definition of a semantic tautology, this is a semantic definition because it uses truth tables.

\newglossaryentry{semantically contingent in SL}
{
name=semantically contingent in SL,
description={A property held by a sentence in SL if and only if the complete truth table for that sentence has both Ts and Fs under its main connective.}
}

Finally, a sentence is contingent if it is sometimes true and sometimes false (p. \pageref{def:contingent_statement}). Similarly, a sentence is \textsc{\gls{semantically contingent in SL}} \label{def:semantically_contingent_in_sl} if and only if its complete truth table for has both Ts and Fs under the main connective. We saw on page \pageref{contingentsentence3.1} that the sentence $M \eand (N \eor P)$ was contingent.

\subsection{Logical equivalence}

\newglossaryentry{semantically logically equivalent in SL}
{
name=semantically logically equivalent in SL,
description={A property held by pairs of sentences if and only if the complete truth table for those sentences has identical columns under the two main connectives.}
}

Two sentences are logically equivalent in English if they have the same truth value as a matter of logic (p. \pageref{def:logical_equivalence}). Once again, we can use truth tables to define a similar property in SL: Two sentences are \textsc{\gls{semantically logically equivalent in SL}} \label{def:semantically_logically_equivalent_in_sl} if they have the same truth value on every row of a complete truth table.

Consider the sentences $\enot(A \eor B)$ and $\enot A \eand \enot B$. Are they logically equivalent? To find out, we construct a truth table.
\begin{center}
\begin{tabu}{ccccc|cccccc}
\enot	\tikz[overlay, shift={(-1ex,-30pt)}, gray] \draw (0pt,0pt) ellipse (2ex and 44pt);		&	$(A$	&	\eor	&	$B)$	&	&	&	\enot	&	$A$	&	\eand	\tikz[overlay, shift={(-1ex,-30pt)}, gray] \draw (0pt,0pt) ellipse (2ex and 44pt); &	\enot	&	$B$\\
\hline
F	& 	T 		& T 		& T 		& 	&	&	F & T & F & F & T\\
F 	&	T 		& T 		& F 		& 	&	&	F & T & F & T & F\\
F 	& 	F 		& T		& T 		& 	&	&	T & F & F & F & T\\
T 	& 	F 		& F 		& F 		& 	&	&	T & F & T & T & F
\end{tabu}
\end{center}
Look at the columns for the main connectives; negation for the first sentence, conjunction for the second. On the first three rows, both are F. On the final row, both are T. Since they match on every row, the two sentences are logically equivalent.

\subsection{Consistency}

\newglossaryentry{semantically consistent in SL}
{
name=semantically consistent in SL,
description={A property held by sets of sentences if and only if the complete truth table for that set contains one line on which all the sentences are true}
}

A set of sentences in English is consistent if it is logically possible for them all to be true at once (p. \pageref{def:inconsistency}).
This means that a sentence is \textsc{\gls{semantically consistent in SL}} \label{def:semantically_consistent_in_sl} if and only if there is at least one line of a complete truth table on which all of the sentences are true. It is semantically inconsistent otherwise.

Consider the three sentences $A \eif B$, $B \eif C$ and $C \eif A$. Since we are considering them as a set, we will put curly braces around them, as is done in set theory: \{$A \eif B, B \eif C, C \eif A$\}. The conditionals in this set form a little loop, but it is possible for all the sentences to be true at the same time, as this truth table shows.

\begin{longtabu}{cccc|ccccc|cccc}
A	&	\eif	&	B	&	&	&	B	&	\eif	&	C	&	&	&	C	&	\eif	&	A	\\
\cline{1-13}
T	\tikz[overlay, shift={(100pt,1ex)}, gray] \draw (0pt,0pt) ellipse (132pt and 2ex); &	T		&	T	&	&	&	T	&	T		&	T	&	&	&	T	&		T	&	T	\\
T	&	T		&	T	&	&	&	T	&	F		&	F	&	&	&	F	&		T	&	T	\\
T	&	F		&	F	&	&	&	F	&	T		&	T	&	&	&	T	&		T	&	T	\\
T	&	F		&	F	&	&	&	F	&	T		&	F	&	&	&	F	&		T	&	T	\\
F	&	T		&	T	&	&	&	T	&	T		&	T	&	&	&	T	&		F	&	F	\\
F	&	T		&	T	&	&	&	T	&	F		&	F	&	&	&	F	&		T	&	F	\\
F	&	T		&	F	&	&	&	F	&	T		&	T	&	&	&	T	&		F	&	F	\\
F	\tikz[overlay, shift={(100pt,1ex)}, gray] \draw (0pt,0pt) ellipse (132pt and 2ex); &	T		&	F	&	&	&	F	&	T		&	F	&	&	&	F	&		T	&	F	\\
\end{longtabu}

\subsection{Validity}

\newglossaryentry{semantically valid in SL}
{
name=semantically valid in SL,
description={A property held by arguments if and only if the complete truth table for the argument contains no rows where the premises are all true and the conclusion false.}
}

Logic is the study of argument, so the most important use of truth tables is to test the validity of arguments. An argument in English is valid if it is logically impossible for the premises to be true and for the conclusion to be false at the same time (p. \pageref{def:valid}). So we can define an argument as \textsc{\gls{semantically valid in SL}} \label{def:semantically_valid_in_sl} if there is no row of a complete truth table on which the premises are all marked ``T'' and the conclusion is marked ``F.'' An argument is invalid if there is such a row.

Consider this argument:
\begin{earg}
\item[1.] $\enot L \eif (J \eor L)$
\item[2.] $\enot L$
\item[] \textcolor{white}{.}\sout{\hspace{.2\linewidth}} \textcolor{white}{.}
\item[$\therefore$] $J$
\end{earg}
Is it valid? To find out, we construct a truth table.

\begin{center}
\tabulinesep=.5ex
\begin{longtabu}{c|c|@{\TTon}*{6}{c}@{\TToff}|@{\TTon}*{2}{c}@{\TToff}|@{\TTon}c@{\TToff}}
$J$&$L$&\enot&$L$&\eif \tikz[overlay, shift={(-1.25ex,-24pt)}, gray] \draw (0pt,0pt) ellipse (2ex and 36pt);&$(J$&\eor&$L)$&\enot\tikz[overlay, shift={(-1ex,-24pt)}, gray] \draw (0pt,0pt) ellipse (2ex and 36pt);&L&J\tikz[overlay, shift={(-.75ex,-24pt)}, gray] \draw (0pt,0pt) ellipse (2ex and 36pt);\\
\hline
%J   L   -   L      ->     (J   v   L)
 T & T & F & T & T & T & T & T & F & T & T\\
 T & F & T & F & T & T & T & F & T & F & T\\
 F & T & F & T & T & F & T & T & F & T & F\\
 F & F & T & F & F & F & F & F & T & F & F
\end{longtabu}
\end{center}

Yes, the argument is valid.
The only row on which both the premises are T is the second row, and on that row the conclusion is also T.

In Chapters 1 and 2 we used the three dots $\therefore$ to represent an inference in English. We used this symbol to represent any kind of inference. The truth table method gives us a more specific notion of a valid inference. We will call this semantic entailment and represent it using a new symbol, $\sdtstile{}{},$ called the ``double turnstile.'' \label{defDoubleTurnstile} The $\sdtstile{}{}$ is like the $\therefore$, except for arguments verified by truth tables. When you use the double turnstile, you write the premises as a set, using curly brackets, \{ and \}, which mathematicians use in set theory. The argument above would be written  $ \{ \enot L \eif (J \eor L), \enot L \} \sdtstile{}{} J$.

More formally, we can define the double turnstile this way: $ \{ \script{A_1}\ldots \script{A_n} \} \sdtstile{}{} \script{B} $ if and only if there is no truth value assignment for which \script{A_1}\ldots \script{A_n} are true and \script{B} is false. Put differently, it means that \script{B} is true for any and all truth value assignments for which \script{A_1}\ldots \script{A_n} are true.

We can also use the double turnstile to represent other logical notions. Since a tautology is always true, it is like the conclusion of a valid argument with no premises. The string $\sdtstile{}{}\script{C}$ means that \script{C} is true for all truth value assignments. This is equivalent to saying that the sentence is entailed by anything. We can represent logical equivalence by writing the double turnstile in both directions: $\script{A} \ndststile{}{} \hspace{.5em} \sdtstile{}{} \script{B}$ For instance, if we want to point out that the sentence $A \eand B$ is equivalent to $B \eand A$ we would write this: $A \eand B \ndststile{}{} \hspace{.5em} \sdtstile{}{} B \eand A$.

%%%%%%%%%%%%%%%%%%Practice Problems

\practiceproblems

If you want additional practice, you can construct truth tables for any of the sentences and arguments in the exercises for the previous chapter.


\noindent\problempart Determine whether each sentence is a tautology, a contradiction, or a contingent sentence, using a complete truth table.

\begin{longtabu}{p{.1\linewidth}p{.9\linewidth}}
\textbf{Example}: & $(A \eif B) \eor (B \eif A)$ \\
\textbf{Answer}: & \vspace{-8pt}\begin{tabular}[t]{cccccccc}
	 (A 	 	 & 	 \eif 	& 	 B) 	 	 & 	 \eor \tikz[overlay, shift={(-.75ex,-24pt)}, gray] \draw (0pt,0pt) ellipse (2ex and 36pt);	 & 	(B 	 	 & 	 \eif	 	 	 & 	 A)	 	 & 	 Tautology\\
\cline{1-7}
 T 	 	 & 	 T 		& 	T 	 	 & 	 T 		 & 	 T 	 	 & 	 T 	 	 & 	T 	 	 & 	 \\
 T 	 	 & 	 F 		& 	F 	 	 & 	 T 	 	 & 	 F 	 	 & 	 T 	 	 & 	T 	 	 & 	  \\
 F 	 	 & 	 T 		& 	T 	 	 & 	 T 	 	 & 	 T 	 	 & 	 F 	 	 & 	F 	 	 & 	 \\
 F 	 	 & 	 T		& 	F 	 	 & 	 T 	 	 & 	 F 	 	 & 	 T 	 	 & 	F 	 	 & 	 \\

\end{tabular}\\
\end{longtabu}

\begin{exercises}
\item $A \eif A$

\answer{
\begin{longtabu}{ccccc}
A 	&\eif \tikz[overlay, shift={(-1ex,-12pt)}, red] \draw (0pt,0pt) ellipse (2ex and 24pt);	& A & Tautology\\
\cline{1-3}
T		&	T	& T	 &			\\
F		&	T	& F	 &			\\
\end{longtabu}
}


\item $C \eif\enot C$

\answer{
\begin{longtabu}{ccccc}

C 	& \eif \tikz[overlay, shift={(-1ex,-12pt)}, red] \draw (0pt,0pt) ellipse (2ex and 24pt);	& \enot 	& C 	& Contingent \\
\cline{1-4}
T	&	F	&	F	& 	T	&			\\
F	&	T	&	T	& 	F	&	\\

\end{longtabu}
}

\item $(A \eiff B) \eiff \enot(A\eiff \enot B)$ %tautology

\answer{
\begin{longtabu}{cccccccccc}
(A 	& \eiff 	& B) 	& \eiff \tikz[overlay, shift={(-1ex,-30pt)}, red] \draw (0pt,0pt) ellipse (2ex and 44pt);	& \enot 	& (A 	& \eiff 	& \enot 	& B) 	& Tautology \\
\cline{1-9}
T	&	T	&	T 	&	T	&	T	&	T	&	F	&	F	& 	T	&	\\
T	&	F	&	F	&	T	&	 F	&	T	&	T	&	T	& 	F	&	\\
F	&	F	&	T	&	T	&	 F	&	F	&	T	&	F	& 	T	&	\\
F	&	T	&	F	&	T	&	 T	&	F	&	F	&	T	& 	F	&	\\
\end{longtabu}
}


\item $(A \eand B) \eif (B \eor A)$  %taut

\answer{
    \begin{longtabu}{cccccccc}
(A  	 	 & 	 \eand  	  & 	 B)  & 	 \eif  \tikz[overlay, shift={(-1ex,-30pt)}, red] \draw (0pt,0pt) ellipse (2ex and 44pt);	 & 	 (B 	 	 & 	 \eor  	 & 	 A)   	 	 & 	 Tautology\\
\cline{1-7}
T 	 	 & 	 T 	 	 & 	 T 	& 	 T 	 	 & 	 T 	 	 & 	 T 	 	 & 	T 	 	 & 	   \\
T 	 	 & 	 F 	 	 & 	 F 	& 	 T 	 	 & 	 F 	 	 & 	 T 	 	 & 	T 	 	 & 	   \\
F 	 	 & 	 F 	 	 & 	 T 	& 	 T 	 	 & 	 T 	 	 & 	 T 	 	 & 	F 	 	 & 	   \\
F 	 	 & 	 F 	 	 & 	 F 	& 	 T 	 	 & 	 F 	 	 & 	 F 	 	 & 	F 	 	 & 	   \\
\end{longtabu}
}


\item $[(\enot A \eor A) \eor B] \eif B$ %taut.

\answer{
 \begin{longtabu}{ccccccccc}
[(\enot  	  & 	 A  	 	 & 	 \eor  	 & 	A)	  	 & 	\eor	 	 & 	B]   	 & 	 \eif	\tikz[overlay, shift={(-1ex,-30pt)}, red] \draw (0pt,0pt) ellipse (2ex and 44pt);	 & 	 B 	 	 & 	 Contingent sentence \\
\cline{1-8}
F 	 	 & 	 T 	 	 & 	 T 	 	 & 	 T 	 	 & 	 T 	 	 & 	 T 	 	 & 	 T 	 	 & 	 T 	 	 & 	 \\
F 	 	 & 	 T 	 	 & 	 T 	 	 & 	 T 	 	 & 	 T 	 	 & 	 F 	 	 & 	 F 	 	 & 	 F 	 	 & 	 \\
T 	 	 & 	 F 	 	 & 	 T 	 	 & 	 F 	 	 & 	 T 	 	 & 	 T 	 	 & 	 T 	 	 & 	 T 	 	 & 	 \\
T 	 	 & 	 F 	 	 & 	 T 	 	 & 	 F 	 	 & 	 T 	 	 & 	 F 	 	 & 	 F 	 	 & 	 F 	 	 & 	 \\

\end{longtabu}
}

\item $[(A \eor B) \eand \enot A] \eand (B \eif A)$ %Contradiction.

\answer{
\begin{longtabu}{ccccccccccc}
[(A  	 &	\eor  & 	 B) 	 & 	 \eand 	& 	\enot & 	 A] 	 	 & 	 \eand \tikz[overlay, shift={(-1ex,-30pt)}, red] \draw (0pt,0pt) ellipse (2ex and 44pt); & (B 	 	 & 	 \eif  	 & 	 A) 	 	& Contradiction. \\
\cline{1-10}
T 	 & 	 T 	 & 	T 	 & 	 F 	 	 & 	 F 	  & 	 T 	 	 & 	 F 	  & 	 T 	 	 & 	 T 	 	 & 	 T 	 	 & 	  \\
T 	 & 	 T 	 & 	F 	 & 	 F 	 	 & 	 F 	 & 	 T 	 	 & 	 F 	 & 	 F 	 	 & 	 T 	 	 & 	 T 	 	 & 	  \\
F 	 & 	 T 	 & 	T 	 & 	 T 	 	 & 	 T 	 & 	 F 	 	 & 	 F 	 & 	 T 	 	 & 	 F 	 	 & 	 F 	 	 & 	 \\
F 	 & 	 F 	 & 	F 	 & 	 F 	 	 & 	 T 	 & 	 F 	 	 & 	 F 	 & 	 F 	 	 & 	 T 	 	 & 	 F 	 	 & 	 \\
\end{longtabu}
}
\end{exercises}

\noindent\problempart Determine whether each sentence is a tautology, a contradiction, or a contingent sentence, using a complete truth table.
\begin{exercises}
\item $\enot B \eand B$ \vspace{.5ex}%contra


\item $\enot D \eor D$ \vspace{.5ex}%taut


\item $(A\eand B) \eor (B\eand A)$\vspace{.5ex} %contingent


\item $\enot[A \eif (B \eif A)]$\vspace{.5ex} %contra


\item $A \eiff [A \eif (B \eand \enot B)]$ \vspace{.5ex}%contra


\item $[(A \eand B) \eiff B] \eif (A \eif B)$ \vspace{.5ex}% contingent.

\end{exercises}



\noindent\problempart \label{pr.TT.equiv} Determine whether each the following statements are equivalent using complete truth tables. If the two sentences really are logically equivalent, write "Logically equivalent." Otherwise write, "Not logically equivalent."

\begin{longtabu}{p{.1\linewidth}p{.9\linewidth}}
\textbf{Example}: & $A \eor B  \ndststile{}{} \hspace{.5em} \sdtstile{}{} \enot A \eif B $\\
\textbf{Answer}: & \vspace{-8pt}\begin{tabular}[t]{ccccccccc}
A	&	\eor \tikz[overlay, shift={(-.75ex,-24pt)}, gray] \draw (0pt,0pt) ellipse (2ex and 36pt);	&	B	&	&	\enot	&	A	&	\eif \tikz[overlay, shift={(-.75ex,-24pt)}, gray] \draw (0pt,0pt) ellipse (2ex and 36pt);&	B	&	Logically Equivalent\\
\cline{1-3} \cline{5-7}
T	&	T		&	T	&	&		F	&	T	&	T		&	T	&	\\
T	&	T		&	F	&	&		F	&	T	&	T		&	F	&	\\
F	&	T		&	T	&	&		T	&	F	&	T		&	T	&	\\
F	&	F		&	F	&	&		T	&	F	&	F		&	F	&	\\
\end{tabular}\\
\end{longtabu}


\begin{exercises}
\item $A\ndststile{}{} \hspace{.5em} \sdtstile{}{} \enot A$\vspace{.5ex} %No

\answer{
\begin{longtabu}{ccccc}
A 	 \tikz[overlay, shift={(-1.25ex,-24pt)}, red] \draw (0pt,0pt) ellipse (2ex and 36pt);	 & 	  	 	 & 	 \enot \tikz[overlay, shift={(-.75ex,-24pt)}, red] \draw (0pt,0pt) ellipse (2ex and 36pt);	 	 & 	 A 	 	 & 	 Not logically equivalent \\
\cline{1-1} \cline{3-4}
T 	 	 & 	   	 	 & 	 F 	 	 & 	 T 	 	 & 	  \\
T 	 	 & 	   	 	 & 	 F 	 	 & 	 T 	 	 & 	  \\
F 	 	 & 	   	 	 & 	 T 	 	 & 	 F 	 	 & 	  \\
F 	 	 & 	   	 	 & 	 T 	 	 & 	 F 	 	 & 	  \\
\end{longtabu}
}

\item $A \eand \enot A\ndststile{}{} \hspace{.5em} \sdtstile{}{} \enot B \eiff B$\vspace{.5ex} %Yes

\answer{
\begin{longtabu}{cccccccccc}

A	 & 	 	\eand \tikz[overlay, shift={(-1.25ex,-24pt)}, red] \draw (0pt,0pt) ellipse (2ex and 36pt);	 & 	 \enot	  & 	 A	 	 & 	 	 	 & 	 \enot	 & 	 B 	 	& 	\eiff \tikz[overlay, shift={(-1.25ex,-24pt)}, red] \draw (0pt,0pt) ellipse (2ex and 36pt);	 & 	 B 	 & 	 Logically equivalent \\
\cline{1-4} \cline{6-9}
T 	 	 & 	 F 	 	 & 	 F 	 	 & 	 T 	 	 & 	  	 	 & 	 F 	 	 & 	 T 	 	 & 	 F 	 	 & 	 T 	 	 & 	  \\
T 	 	 & 	 F 	 	 & 	 F 	 	 & 	 T 	 	 & 	  	 	 & 	 T 	 	 & 	 F 	 	 & 	 F 	 	 & 	 F 	 	 & 	  \\
F 	 	 & 	 F 	 	 & 	 T 	 	 & 	 F 	 	 & 	  	 	 & 	 F 	 	 & 	 T 	 	 & 	 F 	 	 & 	 T 	 	 & 	  \\
F 	 	 & 	 F 	 	 & 	 T 	 	 & 	 F 	 	 & 	  	 	 & 	 T 	 	 & 	 F 	 	 & 	 F 	 	 & 	 F 	 	 & 	  \\
\end{longtabu}
}

\item $[(A \eor B) \eor C]\ndststile{}{} \hspace{.5em} \sdtstile{}{} [A \eor (B \eor C)]$\vspace{.5ex} %Yes

\answer{
\begin{longtabu}{cccccccccccc}
(A		 & 	 \eor	 & 	 	B) 	 & 	\eor	\tikz[overlay, shift={(-1ex,-52pt)}, red] \draw (0pt,0pt) ellipse (2ex and 77pt); 	 	 & 	 C	 	 & 	 	 	 & A 	 	 & 	\eor	\tikz[overlay, shift={(-1ex,-52pt)}, red] \draw (0pt,0pt) ellipse (2ex and 77pt); 	 	 & 	(B 	 	 & 	 \eor 	 & 	C) 	 	 & 	Logically equivalent  \\
\cline{1-5} \cline{7-11}
T	 	 & 	 T	 	 & 	 	T 	 & 	 T	 	 & 	T 	 	 & 	 	 	 & 	 T	 	 & 	T 	 	 & 	 T	 	 & 	 T	 	 & 	 T	 	 & 	  \\
T	 	 & 	 T	 	 & 	 	 T	 & 	 T	 	 & 	 F	 	 & 	 	 	 & 	 T	 	 & 	 T	 	 & 	 T	 	 & 	 T	 	 & 	 F	 	 & 	  \\
T	 	 & 	 T	 	 & 	 	 F	 & 	 T	 	 & 	 T	 	 & 	 	 	 & 	 T	 	 & 	 T	 	 & 	 F	 	 & 	 T	 	 & 	 T	 	 & 	  \\
T	 	 & 	 T	 	 & 	 	 F	 & 	 T	 	 & 	 F	 	 & 	 	 	 & 	 T	 	 & 	 T	 	 & 	 F	 	 & 	 T	 	 & 	 F	 	 & 	  \\
F	 	 & 	 T	 	 & 	 	 T	 & 	 T	 	 & 	 T	 	 & 	 	 	 & 	 F	 	 & 	 T	 	 & 	 T	 	 & 	 T	 	 & 	 T	 	 & 	  \\
F	 	 & 	 T	 	 & 	 	 T	 & 	 T	 	 & 	 F	 	 & 	 	 	 & 	 F	 	 & 	 T	 	 & 	 T	 	 & 	 T	 	 & 	 F	 	 & 	  \\
F	 	 & 	 F	 	 & 	 	 F	 & 	 T	 	 & 	 T	 	 & 	 	 	 & 	 F	 	 & 	 T	 	 & 	 F	 	 & 	 F	 	 & 	 T	 	 & 	  \\
F	 	 & 	 F	 	 & 	 	 F	 & 	 F	 	 & 	 F	 	 & 	 	 	 & 	 F	 	 & 	 F	 	 & 	 F	 	 & 	 F	 	 & 	 F	 	 & 	  \\

\end{longtabu}
}

\item $A \eor (B \eand C)\ndststile{}{} \hspace{.5em} \sdtstile{}{} (A \eor B) \eand (A \eor C)$\vspace{.5ex} %Equivalent

\answer{
\begin{longtabu}{cccccccccccccc}
A	 & 	 \eor \tikz[overlay, shift={(-1ex,-52pt)}, red] \draw (0pt,0pt) ellipse (2ex and 77pt); 		 & 	(B 	 	 & 	 \eand 	 & 	 C)	 	 & 	 	 	 & 	 (A	 	 & 	 	\eor	 & 	 	B) 	 & 	 \eand \tikz[overlay, shift={(-1ex,-52pt)}, red] \draw (0pt,0pt) ellipse (2ex and 77pt); 		 & 	 (A	 	 & 	 \eor 	 & 	 C) 	& Logically equivalent\\
\cline{1-13}
T	 & 	 T	 	 & 	 T	 	 & 	 	T 	 & 	T 	 	 & 	 	 	 & 	 T	 	 & 	 T	 	 & 	 	T 	 & 	 T	 	 & 	 T	 	 & 	T 	 	 & 	 T 	 & \\
T	 & 	 T	 	 & 	 T	 	 & 	 	 F	 & 	 F	 	 & 	 	 	 & 	 T	 	 & 	 T	 	 & 	 	 T	 & 	 T	 	 & 	 T	 	 & 	 T	 	 & 	 F 	 &  \\
T	 & 	 T	 	 & 	 F	 	 & 	 F	 	 & 	 T	 	 & 	 	 	 & 	 T	 	 & 	 T	 	 & 	 F	 	 & 	 T	 	 & 	 T	 	 & 	 T	 	 & 	 T 	 &  \\
T	 & 	 T	 	 & 	 F	 	 & 	 F	 	 & 	 F	 	 & 	 	 	 & 	 T	 	 & 	 T	 	 & 	 F	 	 & 	 T	 	 & 	 T	 	 & 	 T	 	 & 	 F 	 &   \\
F	 & 	 T	 	 & 	 T	 	 & 	 T	 	 & 	 T	 	 & 	 	 	 & 	 F	 	 & 	 T	 	 & 	 T	 	 & 	 T	 	 & 	 F	 	 & 	 T	 	 & 	 T 	 &  \\
F	 & 	 F	 	 & 	 T	 	 & 	 F	 	 & 	 F	 	 & 	 	 	 & 	 F	 	 & 	 T	 	 & 	 T	 	 & 	 F	 	 & 	 F	 	 & 	 F	 	 & 	 F 	 &   \\
F	 & 	 F	 	 & 	 F	 	 & 	 F	 	 & 	 T	 	 & 	 	 	 & 	 F	 	 & 	 F	 	 & 	 F	 	 & 	 F	 	 & 	 F	 	 & 	 T	 	 & 	 T 	 &  \\
F	 & 	 F	 	 & 	 F	 	 & 	 F	 	 & 	 F	 	 & 	 	 	 & 	 F	 	 & 	 F	 	 & 	 F	 	 & 	 	F 	 & 	 F	 	 & 	 F	 	 & 	F 	 &    \\
\end{longtabu}
}

\item $[A \eand (A \eor B)] \eif B\ndststile{}{} \hspace{.5em} \sdtstile{}{} A \eif B$\vspace{.5ex} %Equivalent.

\answer{
\begin{longtabu}{cccccccccccc}
[A	 & 	\eand 	 & 	 (A	 	 & 	 \eor	 & 	B)] 	 	 & \eif
\tikz[overlay, shift={(-.75ex,-24pt)}, red] \draw (0pt,0pt) ellipse (2ex and 36pt); 	 & 	 	B 	 & 	 	 	 & 	 A	 	 & \eif
\tikz[overlay, shift={(-.75ex,-24pt)}, red ] \draw (0pt,0pt) ellipse (2ex and 36pt); 	 & 	 	B 	 & 	Logically equivalent  \\
\cline{1-7} \cline{9-11}
T	  & 	 	T 	 & 	 T	 	 & 	 T	 	 & 	T 	 	 & 	 T	 	 & 	 T	 	 & 	 	 	 & 	 T	 	 & 	T 	 	 & 	T 	 	 & 	  \\
T	  & 	 	T 	 & 	 T	 	 & 	 T	 	 & 	 F	 	 & 	 F	 	 & 	 F	 	 & 	 	 	 & 	 T	 	 & 	 F	 	 & 	 F	 	 & 	  \\
F	  & 	 	 F	 & 	 F	 	 & 	 T	 	 & 	 T	 	 & 	 T	 	 & 	 T	 	 & 	 	 	 & 	 F	 	 & 	 T	 	 & 	 T	 	 & 	  \\
F	  & 	 	 F	 & 	 F	 	 & 	 F	 	 & 	 F	 	 & 	 T	 	 & 	 F	 	 & 	 	 	 & 	 F	 	 & 	 T	 	 & 	 F	 	 & 	  \\
\end{longtabu}
}

\end{exercises}


\noindent\problempart
\label{pr.TT.equiv}
Determine whether each the following statements of equivalence are true or false using complete truth tables. If the two sentences really are logically equivalent, write "Logically equivalent." Otherwise write, "Not logically equivalent."
\begin{exercises}
\item $A\eif A\ndststile{}{} \hspace{.5em} \sdtstile{}{} A \eiff A$ \vspace{.5ex}%No
\item $\enot(A \eif B)\ndststile{}{} \hspace{.5em} \sdtstile{}{} \enot A \eif \enot B$\vspace{.5ex} %No
\item $A \eor B\ndststile{}{} \hspace{.5em} \sdtstile{}{} \enot A \eif B$ \vspace{.5ex}%equivalent.
\item$(A \eif B) \eif C\ndststile{}{} \hspace{.5em} \sdtstile{}{} A \eif (B \eif C)$\vspace{.5ex} %not equivalent.
\item $A \eiff (B \eiff C)\ndststile{}{} \hspace{.5em} \sdtstile{}{} A \eand (B \eand C)$ \vspace{.5ex}%not equivalent.
\end{exercises}


\noindent\problempart \label{pr.TT.consistent} Determine whether each set of sentences is consistent or inconsistent using a complete truth table.

\begin{longtabu}{p{.1\linewidth}p{.9\linewidth}}
\textbf{Example}: & \{$\enot(A \eor B)$, $\enot A \eor B$, $A \eor \enot B$\}\\
\textbf{Answer}: & \begin{tabular}[t]{ccccccccccccccc}
\enot	&	(A	&	\eor	&	B),	&	&	\enot	&	A	&	\eor	&	B,	&	&	A	&	\eor	&	\enot	&	B	&	Consistent \\
\cline{1-4}	\cline{6-9}	\cline{11-14}
F		&	T	&	T		&	T	&	&		F	&	T	&	T		&	T	&	&	T	&	T		&	F		&	T	&\\
F		&	T	&	T		&	F	&	&		F	&	T	&	F		&	F	&	&	T	&	T		&	T		&	F	&\\
F		&	F	&	T		&	T	&	&		T	&	F	&	T		&	T	&	&	F	&	F		&	F		&	T	&\\
\textbf{T}	\tikz[overlay, shift={(120pt,1ex)}, gray] \draw (0pt,0pt) ellipse (154pt and 2ex);	&	F	&	F		&	F	&	&		T	&	F	&	\textbf{T}		&	F	&	&	F	&	\textbf{T}		&	T		&	F	&\\
\end{tabular}\\
\end{longtabu}


\begin{exercises}
\item \{$A \eand \enot B$, $\enot(A \eif B)$, $B \eif A$\}\vspace{.5ex} %Consistent

\answer{
\begin{longtabu}{cccccccccccccc}
A 					 & \eand 		&  \enot & B & & \enot  		& 	 (A	  & 	 \eif	 	 & 	 B)		 & 	 & 	 B	 	 & 	\eif 	 	 & 	A 	 	 & 	 Consistent \\
\cline{1-4} \cline{6-9}\cline{11-13}
T 					 & 	 F	 		&  F	 & T & & F	 		& 	 T	  & 	 T	 	 & 	T 	 	 & 	 & 	 T	 	 & 	 T	 	 & T	 	 	&	  \\
T \tikz[overlay, shift={(110pt,1ex)}, red] \draw (0pt,0pt) ellipse (143pt and 2ex); & 	{\color{black}\textbf{T}}	 & T	 & F & & {\color{black}\textbf{T}}	 & 	 T	 & 	 F	 	 & 	 F	 	 & 	 & 	 F	 	 & 	 {\color{black}\textbf{T}}	 	 & 	 T	 	 & 	  \\
F	 				 & 	 F	 & 	 F	 & T & 	& 	 F	 & 	 F	 & 	 T	 	 & 	 T	 	 & 	  & 	 T	 	 & 	 F	 	 & 	 F	 	 & 	  \\
F	  				& 	 F	 & 	 T	 & 	F&  & 	 F	 & 	 F	 & 	 T	 	 & 	 F	 	 & 	  & 	 F	 	 & 	 T	 	 & 	 F	 	 & 	  \\
\end{longtabu}
}
\item \{$A \eor B$, $A \eif \enot A$, $B \eif \enot B\}$ \vspace{.5ex}%inconsistent.

\answer{
\begin{longtabu}{cccccccccccccc}
A	 & \eor 	 & B 	 & 	 	 & A 	 & \eif 	 & 	\enot & A 	 & 	 	 & B 	 & \eif 	 & \enot	 & 	B 	 & 	Inconsistent \\
\cline{1-3}\cline{5-8} \cline{10-13}
T	 & 	 T	 &T  	 & 	 	 & T	 & 	 F	 & 	F 	 & T 	 & 	 	 & 	T 	 & 	F 	 & 	 F	 & 	T 	 & 	 \\
T	& 	 T	 & F 	 & 	 	 & 	T 	 & 	 F	 & 	 F	 & 	 T	 & 	 	 & 	F 	 & 	 T	 & 	 T	 & 	 F	 & 	 \\
F	& 	 T	 & 	 T	 & 	 	 & 	F 	 & 	 T	 & 	 T	 & 	F 	 & 	 	 & 	 T	 & 	 F	 & 	 F	 & 	 T	 & 	 \\
F	& 	 F	 & 	 F	 & 	 	 & 	 F	 & 	 T	 & 	 T	 & 	 F	 & 	 	 & 	 F	 & 	 T	 & 	 T	 & 	 F	 & 	 \\
\end{longtabu}
}

\item \{$\enot(\enot A \eor B) $, $A \eif \enot C$, $A \eif (B \eif C)\}$\vspace{.5ex} %Inconsistent

\answer{
\begin{longtabu}{ccccccccccccccccc}
\enot & (\enot & A & \eor &B) &  &A  & \eif 	 &\enot 	 &C & 	 & A &\eif 	& (B 	 &\eif 	& C)	 &Consistent \\
\cline{1-5}\cline{7-10} \cline{12-16}
F 	& 	F	 & 	T & T	 & T & 	  & T & F	 & 	 F&T 	 & 	 &T & T	 & T	 &T 	 &T 	 & \\
F	& 	F	 & 	T & T	 & T & 	  & T & T	 & 	 T& F	 & 	 &T & F	 & T	 & F	 &F 	 & \\
T & 	F 	& 	T & F	 & F & 	  & T & F	 & 	 F& T	 & 	 &T & T	 & F	 & T	 &T 	 & \\
\color{black}\textbf{T}	\tikz[overlay, shift={(140pt,1ex)}, red] \draw (0pt,0pt) ellipse (179pt and 2ex);	&  F	 & 	T & F	 & 	F &  & 	T & {\color{black}\textbf{T}}	 & 	 T&F 	& 	 &T & {\color{black}\textbf{T}}	 & F	 & T	&  F 	 & \\
 F	& 	T	 & 	F & T	 & 	T &  & 	F & T	 & 	 F& T	 & 	 &F	 & F	 & T	 & T	 &T 	 & \\
 F	& 	 T	& 	F & T	 & 	T &  & 	F & T	 & 	T & F & 	 &F	 & T	 & T	 &F 	 &F 	 & \\
 F	& 	 T	& 	F & T	 & 	F &  & 	F & T	 & 	F & T	 & 	 &F	 & T	 & F	 & T	 &T 	 & \\
 F	& 	 T	& 	F & T	 & 	F &  & 	F & T	 & 	T & F	 & 	 &F	 & T	 & F	 & T	 &F 	 & \\
\end{longtabu}
}


\item \{$A \eif B$, $A \eand \enot B$\}\vspace{.5ex} %Inconsistent

\answer{
\begin{longtabu}{ccccccccc}
A	&	\eif	 &	B,	&	&	A	&	\eand	&	\enot	&	B	&	Inconsistent\\
\cline{1-3} \cline{5-8}
T	&	T		&	T	&	&	T	&		F	&		F	&	T	&	\\
T	&	F		&	F	&	&	T	&		T	&		T	&	F	&	\\
F	&	T		&	T	&	&	F	&		F	&		F	&	T	&	\\
F	&	T		&	F	&	&	F	&		F	&		T	&	F	&	\\
\end{longtabu}
}


\item \{$A \eif (B \eif C)$, $(A \eif B) \eif C$, $A \eif C$\}\vspace{.5ex} % consistent.

\answer{
\begin{longtabu}{cccccccccccccccc}
A	&	\eif	&	(B	&	\eif	&	 C)	&	&	(A	&	\eif	&	B)	&	\eif	&C	&	&	A	&	\eif	&	C	&	Consistent\\
\cline{1-5} \cline{7-11} \cline{13-15}
T	&	T		&	T	&	T		&	T	&	&	T	&	T		&	T	&	T		&	T	&	&	T	&	T		&	T	&	\\
T	&	F		&	T	&	F		&	F	&	&	T	&	T		&	T	&	F		&	F	&	&	T	&	F		&	F	&	\\
T	&	T		&	F	&	T		&	T	&	&	T	&	F		&	F	&	T		&	T	&	&	T	&	T		&	T	&	\\
T	&	T		&	F	&	T		&	F	&	&	T	&	F		&	F	&	T		&	F	&	&	T	&	F		&	F	&	\\
F	&	T		&	T	&	T		&	T	&	&	F	&	T		&	T	&	T		&	T	&	&	F	&	T		&	T	&	\\
F	&	T		&	T	&	F		&	F	&	&	F	&	T		&	T	&	T		&	F	&	&	F	&	T		&	F	&	\\
F	&	T		&	F	&	T		&	T	&	&	F	&	T		&	F	&	T		&	T	&	&	F	&	T		&	T	&	\\
F	&	{\color{black}\textbf{T}}	\tikz[overlay, shift={(125pt,1ex)}, red] \draw (0pt,0pt) ellipse (165pt and 2ex);	&	F	&	T		&	F	&	&	F	&	T		&	F	&	{\color{black}\textbf{T}}		&	F	&	&	F	&	{\color{black}\textbf{T}}		&	T	&	\\
\end{longtabu}
}

\end{exercises}

\noindent\problempart
\label{pr.TT.consistent}
Determine whether each set of sentences is consistent or inconsistent, using a complete truth table.
\begin{exercises}
\item \{$\enot B$, $A \eif B$, $A$\} \vspace{.5ex}%inconsistent.
\item \{$\enot(A \eor B)$, $A \eiff B$, $B \eif A$\}\vspace{.5ex} %Consistent
\item \{ $A \eor B$, $\enot B$, $\enot B \eif \enot A$\}\vspace{.5ex} %Inconsistent
\item \{$A \eiff B$, $\enot B \eor \enot A$, $A \eif B$\}\vspace{.5ex} %consistent.
\item \{$(A \eor B) \eor C$, $\enot A \eor \enot B$, $\enot C \eor \enot B$\}\vspace{.5ex} %consistent
\end{exercises}




\noindent\problempart \label{pr.TT.valid} Determine whether each argument is valid or invalid, using a complete truth table.

\begin{longtabu}{p{.1\linewidth}p{.9\linewidth}}
\textbf{Example}: & $A \eor B$, $C \eif A$, $C \eif B \sdtstile{}{} C$   \\
\textbf{Answer}: & \begin{tabular}[t]{cccccccccccccc}
A	&	\eor	&	B	&	&	C	&	\eif	&	A	&	&	C	&	\eif	&	B	&	&	C	&	Invalid	\\
\cline{1-3} \cline{5-7}	\cline{9-11}	\cline{13-13}
T	&	T		&	T	&	&	T	&	T		&	T	&	&	T	&	T		&	T	&	&	T	&	\\

T \tikz[overlay, shift={(100pt,1ex)}, gray] \draw (0pt,0pt) ellipse (132pt and 2ex);	&	\textbf{T}		&	T	&	&	F	&	\textbf{T}		&	T	&	&	F	&	\textbf{T}		&	T	&	&	\textbf{F}	&	\\

T	&	T		&	F	&	&	T	&	T		&	T	&	&	T	&	F		&	F	&	&	T	&	\\
T	&	T		&	F	&	&	F	&	T		&	T	&	&	F	&	T		&	F	&	&	F	&	\\
F	&	T		&	T	&	&	T	&	F		&	F	&	&	T	&	T		&	T	&	&	T	&	\\
F	&	T		&	T	&	&	F	&	T		&	F	&	&	F	&	T		&	T	&	&	F	&	\\
F	&	F		&	F	&	&	T	&	F		&	F	&	&	T	&	F		&	F	&	&	T	&	\\
F	&	F		&	F	&	&	F	&	T		&	F	&	&	F	&	T		&	F	&	&	F	&	\\
\end{tabular}\\
\end{longtabu}


\begin{exercises}
\item $A\eif A \sdtstile{}{} A$ \vspace{.5ex}%invalid

\answer{
\begin{longtabu}{cccccc}
A	&	\eif	&	A	&	&	A	&	Invalid \\
\cline{1-3}	\cline{5-5}
T	&	T		&	T	&	&	T	&		\\

F \tikz[overlay, shift={(33pt,1ex)}, red] \draw (0pt,0pt) ellipse (66pt and 2ex);	&	{\color{black}\textbf{T}}		&	F	&	&	{\color{black}\textbf{F}}	&		\\
\end{longtabu}
}

\item $A\eif B$, $B \sdtstile{}{} A$ %invalid
\answer{
\begin{longtabu}{cccccccc}
A	&	\eif	&	B	&	&	B	&	&	A	&	Invalid \\
\cline{1-3} \cline{5-5}	\cline{7-7}
T	&	T		&	T	&	&	T	&	&	T	&	\\
T	&	F		&	F	&	&	F	&	&	T	&	\\
F \tikz[overlay, shift={(44pt,1ex)}, red] \draw (0pt,0pt) ellipse (88pt and 2ex);	&	{\color{black}\textbf{T}}		&	T	&	&	{\color{black}\textbf{T}}	&	&	{\color{black}\textbf{F}}	&	\\
F	&	T		&	F	&	&	F	&	&	F	&	\\
\end{longtabu}
}

\item $A\eiff B$, $B\eiff C \sdtstile{}{}A\eiff C$ %valid

\answer{
\begin{longtabu}{cccccccccccc}
A	&	\eiff	&	B	&	&	B	&	\eiff	&	C	&	&	A	&	\eiff	&	C 	&	Valid \\
\cline{1-3} \cline{5-7} \cline{9-11}
T	&	T		&	T	&	&	T	&	T		&	T	&	&	T	&	T		&	T	&	\\
T	&	T		&	T	&	&	T	&	F		&	F	&	&	T	&	F		&	F	&	\\
T	&	F		&	F	&	&	F	&	F		&	T	&	&	T	&	T		&	T	&	\\
T	&	F		&	F	&	&	F	&	T		&	F	&	&	T	&	F		&	F	&	\\
F	&	F		&	T	&	&	T	&	T		&	T	&	&	F	&	F		&	T	&	\\
F	&	F		&	T	&	&	T	&	F		&	F	&	&	F	&	T		&	F	&	\\
F	&	T		&	F	&	&	F	&	F		&	T	&	&	F	&	F		&	T	&	\\
F	&	T		&	F	&	&	F	&	T		&	F	&	&	F	&	T		&	F	&	\\
\end{longtabu}
}

\item $A \eif B$, $A \eif C\sdtstile{}{}B \eif C$ %invalid.

\answer{
\begin{longtabu}{cccccccccccc}
A	&	\eif	&	B	&	&	A	&	\eif	&	C	&	&	B	&	\eif	&	C	&	Invalid \\
\cline{1-3} \cline{5-7} \cline{9-11}
T	&	T		&	T	&	&	T	&	T		&	T	&	&	T	&	T		&	T	&	\\
T	&	T		&	T	&	&	T	&	F		&	F	&	&	T	&	F		&	F	&	\\
T	&	F		&	F	&	&	T	&	T		&	T	&	&	F	&	T		&	T	&	\\
T	&	F		&	F	&	&	T	&	F		&	F	&	&	F	&	T		&	F	&	\\
F	&	T		&	T	&	&	F	&	T		&	T	&	&	T	&	T		&	T	&	\\

F \tikz[overlay, shift={(100pt,1ex)}, red] \draw (0pt,0pt) ellipse (132pt and 2ex);	&	{\color{black}\textbf{T}}		&	T	&	&	F	&	{\color{black}\textbf{T}}		&	F	&	&	T	&	{\color{black}\textbf{F}}		&	F	&	\\

F	&	T		&	F	&	&	F	&	T		&	T	&	&	F	&	T		&	T	&	\\
F	&	T		&	F	&	&	F	&	T		&	F	&	&	F	&	T		&	F	&	\\
\end{longtabu}
}

\item $A \eif B$, $B \eif A\sdtstile{}{}A \eiff B$ %valid.

\answer{
\begin{longtabu}{cccccccccccc}
A	&	\eif	&	B	&	&	B	&	\eif	&	A	&	&	A	&	\eiff	&	B	&	Valid	\\
\cline{1-3} \cline{5-7} \cline{9-11}
T	&	T		&	T	&	&	T	&	T		&	T	&	&	T	&	T		&	T	&	\\
T	&	F		&	F	&	&	F	&	T		&	T	&	&	T	&	F		&	F	&	\\
F	&	T		&	T	&	&	T	&	F		&	F	&	&	F	&	F		&	T	&	\\
F	&	T		&	F	&	&	F	&	T		&	F	&	&	F	&	T		&	F	&	\\
\end{longtabu}
}

\end{exercises}

\noindent\problempart
\label{pr.TT.valid}
Determine whether each argument is valid or invalid, using a complete truth table.
\begin{exercises}
\item $A\eor\bigl[A\eif(A\eiff A)\bigr] \sdtstile{}{} A $\vspace{.5ex}%invalid
\item $A\eor B$, $B\eor C$, $\enot B \sdtstile{}{}A \eand C$\vspace{.5ex} %valid
\item $A \eif B$, $\enot A\sdtstile{}{}\enot B$ \vspace{.5ex}%invalid
\item $A$, $B\sdtstile{}{}\enot(A\eif \enot B)$ \vspace{.5ex}%valid
\item $\enot(A \eand B)$, $A \eor B$, $A \eiff B\sdtstile{}{}C$ \vspace{.5ex}%valid
\end{exercises}


% *********************************************
% *   Partial Truth Tables							*
% *********************************************
\section{Partial Truth Tables}

In order to show that a sentence is a tautology, we need to show that it is T on every row. So we need a complete truth table. To show that a sentence is \emph{not} a tautology, however, we only need one line: a line on which the sentence is F. Therefore, in order to show that something is not a tautology, it is enough to provide a one-line \emph{partial truth table}---regardless of how many sentence letters the sentence might have in it.

Consider, for example, the sentence $(U \eand T) \eif (S \eand W)$. We want to show that it is \emph{not} a tautology by providing a partial truth table. To begin, we fill in F for the entire sentence, the reverse of how we started when we were doing complete truth tables.

\begin{center}
\begin{tabu}{c|c|c|c|@{\TTon}*{7}{c}@{\TToff}}
$S$&$T$&$U$&$W$&$(U$&\eand&$T)$&\eif  \tikz[overlay, shift={(-1ex,-6pt)}, gray] \draw (0pt,0pt) ellipse (2ex and 18pt);  &$(S$&\eand&$W)$\\
\hline
   &   &   &   &    &    &    &F&    &    &
\end{tabu}
\end{center}

 The main connective of the sentence is a conditional. In order for the conditional to be false, the antecedent must be true (T) and the consequent must be false (F). So we fill these in on the table:

\begin{center}
\begin{tabu}{c|c|c|c|@{\TTon}*{7}{c}@{\TToff}}
$S$&$T$&$U$&$W$&$(U$&\eand&$T)$&\eif  \tikz[overlay, shift={(-1ex,-6pt)}, gray] \draw (0pt,0pt) ellipse (2ex and 18pt);  &$(S$&\eand&$W)$\\
\hline
   &   &   &   &    &  T  &    &F&    &   F &
\end{tabu}
\end{center}

In order for the $(U\eand T)$ to be true, both $U$ and $T$ must be true.

\begin{center}
\begin{tabu}{c|c|c|c|@{\TTon}*{7}{c}@{\TToff}}
$S$&$T$&$U$&$W$&$(U$&\eand&$T)$&\eif  \tikz[overlay, shift={(-1ex,-6pt)}, gray] \draw (0pt,0pt) ellipse (2ex and 18pt);  &$(S$&\eand&$W)$\\
\hline
   & T & T &   &  T &  T  & T  &F&    &   F &
\end{tabu}
\end{center}

Now we just need to make $(S\eand W)$ false. To do this, we need to make at least one of $S$ and $W$ false. We can make both $S$ and $W$ false if we want. All that matters is that the whole sentence turns out false on this line. Making an arbitrary decision, we finish the table in this way:

\begin{center}
\begin{tabu}{c|c|c|c|@{\TTon}*{7}{c}@{\TToff}}
$S$&$T$&$U$&$W$&$(U$&\eand&$T)$&\eif  \tikz[overlay, shift={(-1ex,-6pt)}, gray] \draw (0pt,0pt) ellipse (2ex and 18pt);  &$(S$&\eand&$W)$\\
\hline
 F & T & T & F &  T &  T  & T  &F&  F &   F & F
\end{tabu}
\end{center}

Showing that something is a contradiction requires a complete truth table. Showing that something is \emph{not} a contradiction requires only a one-line partial truth table, where the sentence is true on that one line.

A sentence is contingent if it is neither a tautology nor a contradiction. So showing that a sentence is contingent requires a \emph{two-line} partial truth table: The sentence must be true on one line and false on the other. For example, we can show that the sentence above is contingent with this truth table:
\begin{center}
\begin{tabu}{c|c|c|c|@{\TTon}*{7}{c}@{\TToff}}
$S$&$T$&$U$&$W$&$(U$&\eand&$T)$&\eif  \tikz[overlay, shift={(-1ex,-12pt)}, gray] \draw (0pt,0pt) ellipse (2ex and 24pt);  &$(S$&\eand&$W)$\\
\hline
 F & T & T & F &  T &  T  & T  &F&  F &   F & F\\
 F & T & F & F &  F &  F  & T  &T&  F &   F & F
\end{tabu}
\end{center}
Note that there are many combinations of truth values that would have made the sentence true, so there are many ways we could have written the second line.

Showing that a sentence is \emph{not} contingent requires providing a complete truth table, because it requires showing that the sentence is a tautology or that it is a contradiction.  If you do not know whether a particular sentence is contingent, then you do not know whether you will need a complete or partial truth table. You can always start working on a complete truth table. If you complete rows that show the sentence is contingent, then you can stop. If not, then complete the truth table. Even though two carefully selected rows will show that a contingent sentence is contingent, there is nothing wrong with filling in more rows.

Showing that two sentences are logically equivalent requires providing a complete truth table. Showing that two sentences are \emph{not} logically equivalent requires only a one-line partial truth table: Make the table so that one sentence is true and the other false.

Showing that a set of sentences is consistent requires providing one row of a truth table on which all of the sentences are true. The rest of the table is irrelevant, so a one-line partial truth table will do. Showing that a set of sentences is inconsistent, on the other hand, requires a complete truth table: You must show that on every row of the table at least one of the sentences is false.

Showing that an argument is valid requires a complete truth table. Showing that an argument is \emph{invalid} only requires providing a one-line truth table: If you can produce a line on which the premises are all true and the conclusion is false, then the argument is invalid.

\begin{table}
\begin{mdframed}[style=mytablebox]
\begin{center}
\begin{tabu}{X[1,l,b] X[1,l,b] X[1,l,b]}
\tabulinesep=1ex
\underline{Property}	& Truth table required \newline \underline{to show presence}		&	Truth table required \newline \underline{to show absence} \\
being a tautology		& complete 													 		& one-line partial \\
being a contradiction 	& complete 													 		& one-line partial \\
contingency				& two-line partial 													& complete truth \\
equivalence				& complete 													 		& one-line partial\\
consistency				& one-line partial 													& complete \\
validity					& complete 															& one-line partial \\
\end{tabu}
\end{center}
\end{mdframed}
\caption{Complete or partial truth tables to test for different properties}
\label{table.CompleteVsPartial}
\end{table}

Table \ref{table.CompleteVsPartial} summarizes when a complete truth table is required and when a partial truth table will do.

%\section{The material conditional}
%\label{MaterialConditional}

%The material conditional has some odd properties. For one thing, it does not require that the antecedent and consequent are related in any way.

%contradiction in the antecedent

%tautology in the consequent


%%%%%%%%%%%%%%%%% practice problems

\practiceproblems
\noindent\problempart \label{pr.TT.TTorC} Determine whether each sentence is a tautology, a contradiction, or a contingent sentence. Justify your answer with a complete or partial truth table where appropriate.


\begin{exercises}
\item  $A \eif \enot A$ \vspace{.5ex}

\answer{

\begin{longtabu}{cccc}
A&\eif \tikz[overlay, shift={(-1ex,-12pt)}, red] \draw (0pt,0pt) ellipse (2ex and 24pt);&\enot&A\\\hline
T&F&F&T\\
F&T&T&F
\end{longtabu}
Contingent	 \vspace{6pt}
}
%	T letter, 2 connectives

\item $A \eif (A \eand (A \eor B))$ \vspace{.5ex}

\answer{


\begin{longtabu}{ccc@{}ccc@{}ccc@{}c@{}c}
A&\eif \tikz[overlay, shift={(-1ex,-30pt)}, red] \draw (0pt,0pt) ellipse (2ex and 44pt);
 &(&A&\eand&(&A&\eor&B&)&)\\\hline
T&T&&T&T&&T&T&T&&\\
T&T&&T&T&&T&T&F&&\\
F&T&&F&F&&F&T&T&&\\
F&T&&F&F&&F&F&F&&
\end{longtabu}


Tautology \vspace{6pt}
}
%			2 letters, 3 connectives

\item $(A \eif B) \eiff (B \eif A)$ 	\vspace{.5ex}				%

\answer{

\begin{longtabu}{ccccccc}
(A	&	\eif 	&	B) 	&	\eiff \tikz[overlay, shift={(-1ex,-12pt)}, red] \draw (0pt,0pt) ellipse (2ex and 24pt);	&	(B 	&	\eif 	&	A) \\ \hline
T	&	T		&	T	&	T		&	T	&	T		&	T	\\
T	&	F		&	F	&	F		&	F	&	T		&	T	\\
\end{longtabu}
Contingent \vspace{6pt}
}
%		2 letters, 3 connectives

\item $A \eif \enot(A \eand (A \eor B)) $	\vspace{.5ex}

\answer{

\begin{longtabu}{cccccccc}
A	&	\eif	&	 \enot	&	(A	&	\eand	&	(A	&	\eor	&	B)) \\ \hline
	&			&			&		&			&		&			&	\\
	&			&			&		&			&		&			&	\\
\end{longtabu}
}


%
% 2 letters, 4 connectives

\item $\enot B \eif [(\enot A \eand A) \eor B]$\vspace{.5ex}

\answer{

\begin{longtabu}{ccccccc}

\end{longtabu}
}

%{\color{red}
%$
%\begin{array}{cc|cccc@{}c@{}cccc@{}ccc@{}c}
%a&b&\enot&b&\rightarrow&(&(&\enot&a&\eand&a&)&\eor&b&)\\\hline
%T&T&F&T&\mathbf{T}&&&F&T&F&T&&T&T&\\
%T&F&T&F&\mathbf{F}&&&F&T&F&T&&F&F&\\
%F&T&F&T&\mathbf{T}&&&T&F&F&F&&T&T&\\
%F&F&T&F&\mathbf{F}&&&T&F&F&F&&F&F&
%\end{array}
%$
%Contingent	 \vspace{6pt}
%
%}
%	2 letters, 5 connectives

\item $\enot(A \eor B) \eiff (\enot A \eand \enot B)$ \vspace{.5ex}

\answer{

\begin{longtabu}{ccccccc}

\end{longtabu}
}

%{\color{red}
%$
%\begin{array}{cc|cc@{}ccc@{}ccc@{}ccccc@{}c}
%a&b&\enot&(&a&\eor&b&)&\leftrightarrow&(&\enot&a&\eand&\enot&b&)\\\hline
%T&T&F&&T&T&T&&\mathbf{T}&&F&T&F&F&T&\\
%T&F&F&&T&T&F&&\mathbf{T}&&F&T&F&T&F&\\
%F&T&F&&F&T&T&&\mathbf{T}&&T&F&F&F&T&\\
%F&F&T&&F&F&F&&\mathbf{T}&&T&F&T&T&F&
%\end{array}
%$
%
%Tautology \vspace{6pt}
%}
%2 letters, 6 connectives

\item $[(A \eand B) \eand C] \eif B$\vspace{.5ex}

\answer{

\begin{longtabu}{ccccccc}

\end{longtabu}
}

%
%{\color{red}
%$
%\begin{array}{ccc|c@{}c@{}ccc@{}ccc@{}ccc}
%a&b&c&(&(&a&\eand&b&)&\eand&c&)&\rightarrow&b\\\hline
%T&T&T&&&T&T&T&&T&T&&\mathbf{T}&T\\
%T&T&F&&&T&T&T&&F&F&&\mathbf{T}&T\\
%T&F&T&&&T&F&F&&F&T&&\mathbf{T}&F\\
%T&F&F&&&T&F&F&&F&F&&\mathbf{T}&F\\
%F&T&T&&&F&F&T&&F&T&&\mathbf{T}&T\\
%F&T&F&&&F&F&T&&F&F&&\mathbf{T}&T\\
%F&F&T&&&F&F&F&&F&T&&\mathbf{T}&F\\
%F&F&F&&&F&F&F&&F&F&&\mathbf{T}&F
%\end{array}
%$
%
%Tautology \vspace{6pt}
%}
%
%3 letters, 3 connectives

\item $\enot\bigl[(C\eor A) \eor B\bigr]$\vspace{.5ex}

\answer{

\begin{longtabu}{ccccccc}

\end{longtabu}
}


%
%{\color{red}
%$
%\begin{array}{ccc|cc@{}c@{}ccc@{}ccc@{}c}
%a&b&c&\enot&(&(&c&\eor&a&)&\eor&b&)\\\hline
%T&T&T&\mathbf{F}&&&T&T&T&&T&T&\\
%T&T&F&\mathbf{F}&&&F&T&T&&T&T&\\
%T&F&T&\mathbf{F}&&&T&T&T&&T&F&\\
%T&F&F&\mathbf{F}&&&F&T&T&&T&F&\\
%F&T&T&\mathbf{F}&&&T&T&F&&T&T&\\
%F&T&F&\mathbf{F}&&&F&F&F&&T&T&\\
%F&F&T&\mathbf{F}&&&T&T&F&&T&F&\\
%F&F&F&\mathbf{T}&&&F&F&F&&F&F&
%\end{array}
%$
%
%Contingent \vspace{6pt}
%
%}
%	 	3 letters, 3 connectives

\item $\bigl[(A\eand B) \eand\enot(A\eand B)\bigr] \eand C$ \vspace{.5ex}


\answer{

\begin{longtabu}{ccccccc}

\end{longtabu}
}

%
%{\color{red}
%$
%\begin{array}{ccc|c@{}c@{}ccc@{}cccc@{}ccc@{}c@{}ccc}
%a&b&c&(&(&a&\eand&b&)&\eand&\enot&(&a&\eand&b&)&)&\eand&c\\\hline
%T&T&T&&&T&T&T&&F&F&&T&T&T&&&\mathbf{F}&T\\
%T&T&F&&&T&T&T&&F&F&&T&T&T&&&\mathbf{F}&F\\
%T&F&T&&&T&F&F&&F&T&&T&F&F&&&\mathbf{F}&T\\
%T&F&F&&&T&F&F&&F&T&&T&F&F&&&\mathbf{F}&F\\
%F&T&T&&&F&F&T&&F&T&&F&F&T&&&\mathbf{F}&T\\
%F&T&F&&&F&F&T&&F&T&&F&F&T&&&\mathbf{F}&F\\
%F&F&T&&&F&F&F&&F&T&&F&F&F&&&\mathbf{F}&T\\
%F&F&F&&&F&F&F&&F&T&&F&F&F&&&\mathbf{F}&F
%\end{array}
%$
%
%Contradiction \vspace{6pt}
%
%}
%
%% 	3 letters, 5 connectives
%
\item $(A \eand B) ]\eif[(A \eand C) \eor (B \eand D)]$ \vspace{.5ex}


\answer{

\begin{longtabu}{ccccccc}

\end{longtabu}
}


%
%{\color{red}
%$
%\begin{array}{cccc|c@{}c@{}ccc@{}c@{}ccc@{}c@{}ccc@{}ccc@{}ccc@{}c@{}c}
%a&b&c&d&(&(&a&\eand&b&)&)&\eif&(&(&a&\eand&c&)&\eor&(&b&\eand&d&)&)\\\hline
%T&T&T&T&&&T&T&T&&&\mathbf{T}&&&T&T&T&&T&&T&T&T&&\\
%T&T&F&F&&&T&T&T&&&\mathbf{F}&&&T&F&F&&F&&T&F&F&&\\
%\end{array}
%$
%
%Contingent \vspace{6pt}
%}
%
%	4 letters, 5 connectives
\end{exercises}

\noindent\problempart
\label{pr.TT.TTorC}
Determine whether each sentence is a tautology, a contradiction, or a contingent sentence. Justify your answer with a complete or partial truth table where appropriate.
\begin{exercises}
\item  $\enot (A \eor A)$\vspace{.5ex}							%	Contradiction		1 letter, 2 connectives
\item $(A \eif B) \eor (B \eif A)$\vspace{.5ex}					%	Tautology			2 letters, 2 connectives
\item $[(A \eif B) \eif A] \eif A$\vspace{.5ex}					%	Tautology			2 letters, 3 connectives
\item $\enot[( A \eif B) \eor (B \eif A)]$\vspace{.5ex}			%	Contradiction		2 letters, 4 connectives
\item $(A \eand B) \eor (A \eor B)$\vspace{.5ex} 				%	Contingent		2 letters, 5 connectives
\item $\enot(A\eand B) \eiff A$\vspace{.5ex} 					%contingent			2 letters, 3 connectives
\item $A\eif(B\eor C)$\vspace{.5ex} 							%contingent			3 letters, 2 connectives
\item $(A \eand\enot A) \eif (B \eor C)$\vspace{.5ex} 			%tautology			3 letters, 4 connectives
\item $(B\eand D) \eiff [A \eiff(A \eor C)]$\vspace{.5ex}			%contingent			4 letters, 4 connectives
\item $\enot[(A \eif B) \eor (C \eif D)]$\vspace{.5ex} 			% Contingent. 		4 letters, 4 connectives
\end{exercises}


\noindent\problempart
\label{pr.TT.equiv}
Determine whether each the following statements of equivalence are true or false using complete truth tables. If the two sentences really are logically equivalent, write "Logically equivalent." Otherwise write, "Not logically equivalent."
\begin{exercises}
\item $A\ndststile{}{} \hspace{.5em} \sdtstile{}{}\enot A$\vspace{.5ex} 											%No		1 letter, 1 connective, matching
\item $A\eif A\ndststile{}{} \hspace{.5em} \sdtstile{}{}A \eiff A$\vspace{.5ex} 									%No		1 letter, 1 connectives, matching
\item $A	\eand (B	\eand C)\ndststile{}{} \hspace{.5em} \sdtstile{}{} A \eand \enot A$\vspace{.5ex}  					%No		2 letters, 4 connectives, matching
\item $A \eand \enot A\ndststile{}{} \hspace{.5em} \sdtstile{}{}\enot B \eiff B$ \vspace{.5ex}						%Yes		2 letters, 4 connectives, matching
\item $\enot(A \eif B)\ndststile{}{} \hspace{.5em} \sdtstile{}{}\enot A \eif \enot B$\vspace{.5ex}					%No		2 letters, 5 connectives, matching
\item $A \eiff B\ndststile{}{} \hspace{.5em} \sdtstile{}{}\enot[(A \eif B) \eif \enot (B\eif A)]$\vspace{.5ex}			%yes		2 letters, 6 connectives, matching
\item $(A \eand B) \eif (\enot A \eor \enot B)\ndststile{}{} \hspace{.5em} \sdtstile{}{} \enot(A\eand B)$\vspace{.5ex}	%Yes		2 letters, 7 connectives, matching
\item $[(A \eor B) \eor C]\ndststile{}{} \hspace{.5em} \sdtstile{}{}[A \eor (B \eor C)]$\vspace{.5ex} 				%Yes		3 letters, 4 connectives, matching
\item $ (Z \eand (\enot R \eif O))\ndststile{}{} \hspace{.5em} \sdtstile{}{} \enot (R \eif \enot O) $\vspace{.5ex}		%			3 letters, 6 connectives, not matching

\end{exercises}

\noindent\problempart
Determine whether each the following statements of equivalence are true or false using complete truth tables. If the two sentences really are logically equivalent, write "Logically equivalent." Otherwise write, "Not logically equivalent."
\begin{exercises}
\item $A\ndststile{}{} \hspace{.5em} \sdtstile{}{}A \eor A$\vspace{.5ex} 												%Yes		1 letter, 1 connective, matching
\item $A\ndststile{}{} \hspace{.5em} \sdtstile{}{}A \eand A$\vspace{.5ex} 												%Yes		1 letter, 1 connective, matching
\item $A \eor \enot B\ndststile{}{} \hspace{.5em} \sdtstile{}{}A\eif B$\vspace{.5ex} 									%No		2 letters, 3 connectives, matching
\item $(A \eif B)\ndststile{}{} \hspace{.5em} \sdtstile{}{}(\enot B \eif \enot A)$\vspace{.5ex} 							%Yes		2 letters, 4 connectives, matching
\item $\enot(A \eand B)\ndststile{}{} \hspace{.5em} \sdtstile{}{}\enot A \eor \enot B$ \vspace{.5ex}						%Yes		2 letters, 5 connectives, matching
\item $ ((U \eif (X \eor X)) \eor U) \ndststile{}{} \hspace{.5em} \sdtstile{}{} \enot (X \eand (X \eand U)) $\vspace{.5ex}	% 			2 letters, 6 connectives, matching
\item $ ((C \eand (N \eiff C)) \eiff C) \ndststile{}{} \hspace{.5em} \sdtstile{}{} (\enot \enot \enot N \eif C) $\vspace{.5ex}	% 			2 letters, 7 connectives, matching
\item $[(A \eor B) \eand C]\ndststile{}{} \hspace{.5em} \sdtstile{}{}[A \eor (B \eand C)]$\vspace{.5ex} 					%No		3 letters, 4 connectives, matching
\item $((L \eand C) \eand I)\ndststile{}{} \hspace{.5em} \sdtstile{}{}L \eor C$\vspace{.5ex}								%No		3 letters, not matching
\end{exercises}


\noindent\problempart
\label{pr.TT.consistent}
Determine whether each set of sentences is consistent or inconsistent. Justify your answer with a complete or partial truth table where appropriate.
\begin{exercises}
\item \{$A\eif A$, $\enot A \eif \enot A$, $A\eand A$, $A\eor A$\} \vspace{.5ex}%consistent
\item \{$A \eif \enot A$, $\enot A \eif A$\}\vspace{.5ex}%inconsistent.
\item \{$A\eor B$, $A\eif C$, $B\eif C$\}\vspace{.5ex} %consistent
\item \{$A \eor B$, $A \eif C$, $B \eif C$, $\enot C$\}\vspace{.5ex} %	Inconsistent
\item \{$B\eand(C\eor A)$, $A\eif B$, $\enot(B\eor C)$\}\vspace{.5ex}  %inconsistent
\item \{$(A \eiff B) \eif B$,  $B \eif \enot (A \eiff B)$, $A \eor B$\} \vspace{.5ex} %	Consistent
\item \{$A\eiff(B\eor C)$, $C\eif \enot A$, $A\eif \enot B$\}\vspace{.5ex} %consistent
\item  \{$A \eiff B$,  $\enot B \eor \enot A$,  $A \eif  B$\} \vspace{.5ex}% Consistent
\item \{$A \eiff B$, $A \eif C$, $B \eif D$, $\enot(C \eor D)$\}\vspace{.5ex} %consitent
\item \{$\enot (A \eand \enot B)$,  $B \eif \enot A$, $\enot B$ \} \vspace{.5ex} %Consistent
\end{exercises}

\noindent\problempart
\label{pr.TT.consistent}
Determine whether each set of sentences is consistent or inconsistent. Justify your answer with a complete or partial truth table where appropriate.
\begin{exercises}
\item \{$A \eand B$, $C\eif \enot B$, $C$\} \vspace{.5ex}%inconsistent
\item \{$A\eif B$, $B\eif C$, $A$, $\enot C$\}\vspace{.5ex} %inconsistent
\item \{$A \eor B$, $B\eor C$, $C\eif \enot A$\}\vspace{.5ex} %consistent
\item \{$A$, $B$, $C$, $\enot D$, $\enot E$, $F$\}\vspace{.5ex} %consistent
\item \{$A \eand (B \eor C)$, $\enot(A \eand C)$, $\enot(B \eand C)$\} \vspace{.5ex}%consistent
\item \{$A \eif B$, $B \eif C$, $\enot(A \eif C)$\} \vspace{.5ex} %inconsistent

%\begin{tabular}{ccc|ccc|ccccc}
%A 	&\eif	&B 	&B 	&\eif 	&C 	&\enot 	&(A 	&\eif 	&C) 	&Inconsistent\\
%\cline{1-10}
%T 	&T 	&T 	&T 	&T 	&T 	&F 		&T 	&T 	&T 	& \\
%T&T&T&T&F&F&T&T&F&F&\\
%T&F&F&F&T&T&F&T&T&T&\\
%T&F&F&F&T&F&T&T&F&F&\\
%F&T&T&T&T&T&F&F&T&T&\\
%F&T&T&T&F&F&F&F&T&F&\\
%F&T&F&F&T&T&F&F&T&T&\\
%F&T&F&F&T&F&F&F&T&F&\\
%\end{tabular}

\end{exercises}


\noindent\problempart Determine whether each argument is valid or invalid. Justify your answer with a complete or partial truth table where appropriate.
\label{pr.TT.valid}
\begin{exercises}

\item $A\eif(A\eand\enot A)\sdtstile{}{}\enot A$% valid

\answer{
 \begin{longtabu}{ccccccc|cc}
A	&	\eif	&	(A	&	\eand	&	\enot	&	A)	&	&	\enot	&	A	\\ \hline
T	&	F		&	T	&	F		&	F		&	T	&	&		F	&	T\\
F	&	T		&	F	&	F		&	T		&	F	&	&		T	&	F\\
\end{longtabu}
Valid
}



\item $A \eor B$, $A \eif B$, $B \eif A \sdtstile{}{} A \eiff B$  % Valid

\answer{

\begin{longtabu}{cccc|cccc|cccc|ccc}
A	&	\eor 	&	B	&		&	 A	&	\eif	&	B	&		&	B	&	\eif	&	A	&		&	A	&	\eiff	&	B\\ \hline
T	&			&	T	&		&	T	&			&	T	&		&	T	&			&	T	&		&	T	&			&	T	\\
T	&			&	F	&		&	T	&			&	F	&		&	F	&			&	T	&		&	T	&			&	F	\\
F	&			&	T	&		&	F	&			&	T	&		&	T	&			&	F	&		&	F	&			&	T	\\
F	&			&	F	&		&	F	&			&	F	&		&	F	&			&	F	&		&	F	&			&	F	\\
\end{longtabu}
}


\item $A\eor(B\eif A)\sdtstile{}{}\enot A \eif \enot B$ %valid

\answer{

\begin{longtabu}{ccccccc}

\end{longtabu}
}


\item $A \eor B$, $A \eif B$, $ B \eif A \sdtstile{}{} A \eand B$ %valid

\answer{

\begin{longtabu}{ccccccc}

\end{longtabu}
}


\item $(B\eand A)\eif C$, $(C\eand A)\eif B\sdtstile{}{}(C\eand B)\eif A$ % invalid

\answer{

\begin{longtabu}{ccccccc}

\end{longtabu}
}


\item $\enot (\enot A \eor \enot B)$, $A \eif \enot C \sdtstile{}{} A \eif (B \eif C)$ % invalid.

\answer{

\begin{longtabu}{ccccccc}

\end{longtabu}
}


\item $A \eand (B \eif C)$, $\enot C \eand (\enot B \eif \enot A)\sdtstile{}{}C \eand \enot C$ % valid

\answer{

\begin{longtabu}{ccccccc}

\end{longtabu}
}


\item $A \eand B$, $\enot A \eif \enot C$, $B \eif \enot D \sdtstile{}{} A \eor B$ % Invalid

\answer{

\begin{longtabu}{ccccccc}

\end{longtabu}
}


\item $A \eif B\sdtstile{}{}(A \eand B) \eor (\enot A \eand \enot B)$ % invalid

\answer{

\begin{longtabu}{ccccccc}

\end{longtabu}
}


\item $\enot A \eif B$,$ \enot B \eif C $,$ \enot C \eif A \sdtstile{}{} \enot A \eif (\enot B \eor \enot C) $% Invalid

\answer{

\begin{longtabu}{ccccccc}

\end{longtabu}
}


\end{exercises}

\noindent\problempart Determine whether each argument is valid or invalid. Justify your answer with a complete or partial truth table where appropriate.
\label{pr.TT.valid}
\begin{exercises}
\item $A\eiff\enot(B\eiff A)\sdtstile{}{}A$ % invalid

\answer{

\begin{longtabu}{ccccccc}

\end{longtabu}
}


\item $A\eor B$, $B\eor C$, $\enot A\sdtstile{}{}B \eand C$ % invalid

\answer{

\begin{longtabu}{ccccccc}

\end{longtabu}
}


\item $A \eif C$, $E \eif (D \eor B)$, $B \eif \enot D\sdtstile{}{}(A \eor C) \eor (B \eif (E \eand D))$ % invalid

\answer{

\begin{longtabu}{ccccccccccccccccccccc}

\end{longtabu}
}


\item $A \eor B$, $C \eif A$, $C \eif B\sdtstile{}{}A \eif (B \eif C)$ % invalid

\answer{

\begin{longtabu}{ccccccc}

\end{longtabu}
}


\item $A \eif B$, $\enot B \eor A\sdtstile{}{}A \eiff B$ % valid

\answer{

\begin{longtabu}{ccccccc}

\end{longtabu}
}


\end{exercises}

\noindent\problempart
\label{pr.TT.concepts}
Answer each of the questions below and justify your answer.
\begin{exercises}
\item Suppose that \script{A} and \script{B} are logically equivalent. What can you say about $\script{A}\eiff\script{B}$?
%\script{A} and \script{B} have the same truth value on every line of a complete truth table, so $\script{A}\eiff\script{B}$ is true on every line. It is a tautology.
\item Suppose that $(\script{A}\eand\script{B})\eif\script{C}$ is contingent. What can you say about the argument ``\script{A}, \script{B}, $\therefore$\ \script{C}''?
%The sentence is false on some line of a complete truth table. On that line, \script{A} and \script{B} are true and \script{C} is false. So the argument is invalid.
\item Suppose that $\{\script{A},\script{B}, \script{C}\}$ is inconsistent. What can you say about $(\script{A}\eand\script{B}\eand\script{C})$?
%Since there is no line of a complete truth table on which all three sentences are true, the conjunction is false on every line. So it is a contradiction.
\item Suppose that \script{A} is a contradiction. What can you say about the argument \{\script{A}, \script{B}\} $\sdtstile{}{}$  \script{C}?
%Since \script{A} is false on every line of a complete truth table, there is no line on which \script{A} and \script{B} are true and \script{C} is false. So the argument is valid.
\item Suppose that \script{C} is a tautology. What can you say about the argument \{\script{A}, \script{B}\} $\sdtstile{}{}$ \script{C}''?
%Since \script{C} is true on every line of a complete truth table, there is no line on which \script{A} and \script{B} are true and \script{C} is false. So the argument is valid.
%\item Suppose that \script{A} and \script{B} are logically equivalent. What can you say about $(\script{A}\eor\script{B})$?
%Not much. $(\script{A}\eor\script{B})$ is a tautology if \script{A} and \script{B} are tautologies; it is a contradiction if they are contradictions; it is contingent if they are contingent.
\item Suppose that \script{A} and \script{B} are \emph{not} logically equivalent. What can you say about $(\script{A}\eor\script{B})$?
%\script{A} and \script{B} have different truth values on at least one line of a complete truth table, and $(\script{A}\eor\script{B})$ will be true on that line. On other lines, it might be true or false. So $(\script{A}\eor\script{B})$ is either a tautology or it is contingent; it is \emph{not} a contradiction.
\end{exercises}

% *********************************************
% *   Expressive Completeness	      						*
% *********************************************

\section{Expressive Completeness}
\label{sec:expressive_completeness}

We could leave the biconditional (\eiff) out of the language. If we did that, we could still write ``$A\eiff B$'' so as to make sentences easier to read, but that would be shorthand for $(A\eif B) \eand (B\eif A)$. The resulting language would be formally equivalent to SL, since $A\eiff B$ and $(A\eif B) \eand (B\eif A)$ are logically equivalent in SL. If we valued formal simplicity over expressive richness, we could replace more of the connectives with notational conventions and still have a language equivalent to SL.

There are a number of equivalent languages with only two connectives. You could do logic with only the negation and the material conditional. Alternately you could just have the negation and the disjunction. You will be asked to prove that these things are true in the last problem set. You could even have a language with only one connective, if you designed the connective right. The \emph{Sheffer stroke} is a logical connective with the following characteristic truth table:
\begin{center}
\begin{tabular}{c|c|c}
\script{A} & \script{B} & \script{A}$|$\script{B}\\
\hline
T & T & F\\
T & F & T\\
F & T & T\\
F & F & T
\end{tabular}
\end{center}
The Sheffer stroke has the unique property that it is the only connective you need to have a complete system of logic. You will be asked to prove that this is true in the last problem set also.


%\fix{Summary of test conditions}

\practiceproblems
\noindent\problempart
\begin{exercises}
\item In section \ref{sec:expressive_completeness}, we said that you could have a language that only used the negation and the material conditional. Prove that this is true by writing sentences that are logically equivalent to each of the following using only parentheses, sentence letters, negation (\enot), and the material conditional (\eif).
\begin{enumerate}
\item $A\eor B$
%$\enot A \eif B$
\item $A\eand B$
%$\enot(A \eif \enot B)$
\item $A\eiff B$
%$\enot [(A\eif B) \eif \enot(B\eif A)]$
\end{enumerate}

\item We also said in section 3.5 that you could have a language which used only the negation and the disjunction. Show this: Using only parentheses, sentence letters, negation (\enot), and disjunction (\eor), write sentences that are logically equivalent to each of the following.
\begin{enumerate}

\item $A \eand B$
%$\enot(\enot A \eor \enot B)$
\item $A \eif B$
%$\enot A \eor B$
\item $A \eiff B$
%$\enot(\enot A \eor \enot B) \eor \enot(A \eor B)$
\end{enumerate}

\item Write a sentence using the connectives of SL that is logically equivalent to $(A|B)$.
\item Every sentence written using a connective of SL can be rewritten as a logically equivalent sentence using one or more Sheffer strokes. Using only the Sheffer stroke, write sentences that are equivalent to each of the following.
%...
\begin{enumerate}
\setcounter{eargnum}{\arabic{OLDeargnum}}
\item $\enot A$
\item $(A\eand B)$
\item $(A\eor B)$
\item $(A\eif B)$
\item $(A\eiff B)$
\end{enumerate}
\end{exercises}


%%%% A recursive definition of truth in SL

%[go back and explicitly mark the section on a recursive definition of a sentence in SL as optional and then rework it so it is parallel to this passage.]

%In the optional later sections of the chapter, we gave a recursive definition of what it meant to be a sentence in SL. We will also end this chapter with a recursive definition that summarizes the material in the chapter. In this case, we are going to give a recursive definition of truth in SL. [sources: magnus's original treatment. Hodges in the logic handbook. Tarski. Be sure to motivate this and explain its relationship to a recursive definition of a truth table.]



%%%Below here is Magnus's original version of recursive definition of truth in SL


%
%Formally, what we want is a function that assigns a 1 or 0 to each of the sentences of SL. We can interpret this function as a definition of truth for SL if it assigns 1 to all of the true sentences of SL and 0 to all of the false sentences of SL. Call this function ``$v$'' (for ``valuation''). We want $v$ to a be a function such that for any sentence \script{A}, $v(\script{A})=1$ if \script{A} is true and $v(\script{A})=0$ if \script{A} is false.
%
%Recall that the recursive definition of a wff for SL had two stages: The first step said that atomic sentences (solitary sentence letters) are wffs. The second stage allowed for wffs to be constructed out of more basic wffs. There were clauses of the definition for all of the sentential connectives. For example, if \script{A} is a wff, then \enot\script{A} is a wff.
%
%Our strategy for defining the truth function, $v$, will also be in two steps. The first step will handle truth for atomic sentences; the second step will handle truth for compound sentences.
%
%
%\subsection{Truth in SL}
%How can we define truth for an atomic sentence of SL? Consider, for example, the sentence $M$. Without an interpretation, we cannot say whether $M$ is true or false. It might mean anything. If we use $M$ to symbolize ``The moon orbits the Earth'', then $M$ is true. If we use $M$ to symbolize ``The moon is a giant turnip'', then $M$ is false.
%
%Moreover, the way you would discover whether or not $M$ is true depends on what $M$ means. If $M$ means ``It is Monday,'' then you would need to check a calendar. If $M$ means ``Jupiter's moon Io has significant volcanic activity,'' then you would need to check an astronomy text---and astronomers know because they sent satellites to observe Io.
%
%When we give a symbolization key for SL, we provide an {interpretation} of the sentence letters that we use. The key gives an English language sentence for each sentence letter that we use. In this way, the interpretation specifies what each of the sentence letters \emph{means}. However, this is not enough to determine whether or not that sentence is true. The sentences about the moon, for instance, require that you know some rudimentary astronomy. Imagine a small child who became convinced that the moon is a giant turnip. She could understand what the sentence ``The moon is a giant turnip'' means, but mistakenly think that it was true.
%
%Consider another example: If $M$ means ``It is morning now'', then whether it is true or not depends on when you are reading this. I know what the sentence means, but---since I do not know when you will be reading this---I do not know whether it is true or false.
%
%So an interpretation alone does not determine whether a sentence is true or false. Truth or falsity depends also on what the world is like. If $M$ meant ``The moon is a giant turnip'' and the real moon were a giant turnip, then $M$ would be true. To put the point in a general way, truth or falsity is determined by an interpretation \emph{plus} a way that the world is.
%
%\begin{center}
%INTERPRETATION + STATE OF THE WORLD $\Longrightarrow$ TRUTH/FALSITY
%\end{center}
%
%In providing a logical definition of truth, we will not be able to give an account of how an atomic sentence is made true or false by the world. Instead, we will introduce a \emph{truth value assignment}. Formally, this will be a function that tells us the truth value of all the atomic sentences. Call this function ``$a$'' (for ``assignment''). We define $a$ for all sentence letters \script{P}, such that
%\begin{displaymath}
%a(\script{P}) =
%\left\{
%	\begin{array}{ll}
%	1 & \mbox{if \script{P} is true},\\
%	0 & \mbox{otherwise.}
%	\end{array}
%\right.
%\end{displaymath}
%This means that $a$ takes any sentence of SL and assigns it either a one or a zero; one if the sentence is true, zero if the sentence is false. The details of the function $a$ are determined by the meaning of the sentence letters together with the state of the world. If $D$ means ``It is dark outside'', then $a(D)=1$ at night or during a heavy storm, while $a(D)=0$ on a clear day.
%
%You can think of $a$ as being like a row of a truth table. Whereas a truth table row assigns a truth value to a few atomic sentences, the truth value assignment assigns a value to every atomic sentence of SL. There are infinitely many sentence letters, and the truth value assignment gives a value to each of them. When constructing a truth table, we only care about sentence letters that affect the truth value of sentences that interest us. As such, we ignore the rest. Strictly speaking, every row of a truth table gives a \emph{partial} truth value assignment.
%
%It is important to note that the truth value assignment, $a$, is not part of the language SL. Rather, it is part of the mathematical machinery that we are using to describe SL. It encodes which atomic sentences are true and which are false.
%
%
%We now define the truth function, $v$, using the same recursive structure that we used to define a wff of SL.
%
%\begin{enumerate}
%\item If \script{A} is a sentence letter, then $v(\script{A})=a(\script{A})$.
%%\setcounter{Example}{\arabic{enumi}}\end{enumerate}
%%...
%% Break out of the {enumerate} environment to say something about what is
%% going on. Using \setcounter in this way preserves the numbering, so
%% that the list can resume after the comments.
%
%%This is a mathematical equals sign, not the identity predicate we defined for QL.
%
%% Resume the {enumerate} environment and restore the counter.
%%...
%%\begin{enumerate}\setcounter{enumi}{\arabic{Example}}
%\item If \script{A} is ${\enot}\script{B}$ for some sentence \script{B}, then
%\begin{displaymath}v(\script{A}) =
%	\left\{\begin{array}{ll}
%	1 & \mbox{if $v(\script{B}) = 0$},\\
%	0 & \mbox{otherwise.}
%	\end{array}\right.
%\end{displaymath}
%
%\item If \script{A} is $(\script{B}\eand\script{C})$ for some sentences \script{B,C}, then
%\begin{displaymath}v(\script{A}) =
%	\left\{\begin{array}{ll}
%	1 & \mbox{if $v(\script{B}) = 1$ and $v(\script{C}) = 1$,}\\
%	0 & \mbox{otherwise.}
%	\end{array}\right.
%\end{displaymath}
%\setcounter{Example}{\arabic{enumi}}\end{enumerate}
%%...
%
%It might seem as if this definition is circular, because it uses the word ``and'' in trying to define ``and.'' Notice, however, that this is not a definition of the English word ``and''; it is a definition of truth for sentences of SL containing the logical symbol ``\eand.'' We define truth for object language sentences containing the symbol ``\eand'' using the metalanguage word ``and.'' There is nothing circular about that.
%
%%...
%\begin{enumerate}\setcounter{enumi}{\arabic{Example}}
%\item If \script{A} is $(\script{B}\eor\script{C})$ for some sentences \script{B,C}, then
%\begin{displaymath}v(\script{A}) =
%	\left\{\begin{array}{ll}
%	0 & \mbox{if $v(\script{B}) = 0$ and $v(\script{C}) = 0$,}\\
%	1 & \mbox{otherwise.}
%	\end{array}\right.
%\end{displaymath}
%%\setcounter{Example}{\arabic{enumi}}\end{enumerate}
%%...
%%Notice that this defines truth for sentences containing the symbol ``\eor''' using the word ``and.''
%%...
%%\begin{enumerate}\setcounter{enumi}{\arabic{Example}}
%\item If \script{A} is $(\script{B}\eif\script{C})$ for some sentences \script{B,C}, then
%\begin{displaymath}v(\script{A}) =
%	\left\{\begin{array}{ll}
%	0 & \mbox{if $v(\script{B}) = 1$ and $v(\script{C}) = 0$,}\\
%	1 & \mbox{otherwise.}
%	\end{array}\right.
%\end{displaymath}
%
%\item If \script{A} is $(\script{B}\eiff\script{C})$ for some sentences \script{B,C}, then
%\begin{displaymath}v(\script{A}) =
%	\left\{\begin{array}{ll}
%	1 & \mbox{if $v(\script{B}) = v(\script{C})$},\\
%	0 & \mbox{otherwise.}
%	\end{array}\right.
%\end{displaymath}
%\end{enumerate}
%
%Since the definition of $v$ has the same structure as the definition of a wff, we know that $v$ assigns a value to \emph{every} wff of SL. Since the sentences of SL and the wffs of SL are the same, this means that $v$ returns the truth value of every sentence of SL.
%
%Truth in SL is always truth \emph{relative to} some truth value assignment, because the definition of truth for SL does not say whether a given sentence is true or false. Rather, it says how the truth of that sentence relates to a truth value assignment.
%

%%%%%%%%%%%%%%%%%%%%%%%%		 Key Terms
