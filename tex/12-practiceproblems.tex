
%%%%%%%%%%%  Practice Problems %%%%%%%%%%%


\practiceproblems
\noindent \problempart \label{pr.EnglishTautology} Label the following tautology, contradiction, or contingent statement.

\begin{longtabu}{p{.1\linewidth}p{.9\linewidth}}
\textbf{Example}: & Caesar crossed the Rubicon. \\
\textbf{Answer}: & Contingent statement. \\
&The Rubicon is a river in Italy. When General Julius Caesar took his army across it, he was committing to a revolution against the Roman Republic. Since that time, ``crossing the Rubicon'' has been a expression referring to making an irreversible decision. This kind of decision certainly seems to be contingent. Caesar could have decided otherwise.\\
\end{longtabu}

\begin{exercises}
\item Someone once crossed the Rubicon. \answer{\underline{Contingent statement}}
\item No one has ever crossed the Rubicon. \answer{\underline{Contingent  statement}}
\item If Caesar crossed the Rubicon, then someone has. \answer{\underline{Tautology}}
\item Even though Caesar crossed the Rubicon, no one has ever crossed the Rubicon. \answer{\underline{Contradiction}}
\item If anyone has ever crossed the Rubicon, it was Caesar. \answer{\underline{Contingent statement}}
\end{exercises}

\noindent \problempart Label the following tautology, contradiction, or contingent statement.
\begin{exercises}
\item Elephants dissolve in water. \answer{\underline{Contingent}}
\item Wood is a light, durable substance useful for building things. \answer{\underline{Contingent}}
\item If wood were a good building material, it would be useful for building things. \answer{\underline{Tautology}}
\item I live in a three story building that is two stories tall. \answer{\underline{Contradiction}}
\item If gerbils were mammals they would nurse their young. \answer{\underline{Tautology}}
\end{exercises}

\noindent \problempart Label the following logically equivalent, contradictory, or neither.

\begin{longtabu}{p{.1\linewidth}p{.9\linewidth}}
\textbf{Example}: &  All students who study will pass the test. \\
& If Jeremy studies, he will pass the test. \\
\textbf{Answer}: & Neither. \\
&If the first statement is true, then the second statement has to be true, but the reverse is not the case. It might be that Jeremy will pass the test if he studies, but some other students are going to fail no matter what.\\
\end{longtabu}


\begin{exercises}
\item Elephants dissolve in water.	\\
	If you put an elephant in water, it will dissolve.
\answer{\\\underline{Logically equivalent}}

\item All mammals dissolve in water.\\
	If you put an elephant in water, it will dissolve.
\answer{\\ \underline{Neither}}

\item Elephants are bigger than lions. \\
Elephants are smaller or the same size as lions.
\answer{\\ \underline{Contradictory}}

\item The Eurasian elephant is an herbivore \\
All the Eurasian elephant sometimes eats meat
\answer{\\ \underline{Contradictory}}

\item Elephants dissolve in water. 	\\
	All mammals dissolve in water.
\answer{\\ \underline{Neither}}

\end{exercises}


\noindent \problempart Label the following logically equivalent, contradictory, or neither.

\begin{exercises}
\item  Thelonious Monk played piano.	\\
John Coltrane played tenor sax.
\answer{\\ \underline{Neither}}

\item  Thelonious Monk played gigs with John Coltrane.	\\
	John Coltrane played gigs with Thelonious Monk.
\answer{\\ \underline{Logically equivalent}}

\item  All professional piano players have big hands.	\\
	Piano player Bud Powell had big hands.
	\answer{\\ \underline{Neither}}

\item  Bud Powell suffered from severe mental illness.	 \\
	All piano players suffer from severe mental illness.
	\answer{\\ \underline{Neither}}

\item John Coltrane was deeply religious.	 \\
John Coltrane was moderately or not at all religious
\answer{\\ \underline{Contradictory}}
\end{exercises}


\noindent \problempart Consider again the statements on p.\pageref{MartianGiraffes}:
\begin{enumerate}[label=(\alph*)]
\item \label{itm:at_least_four}There are at least four giraffes at the wild animal park.
\item \label{itm:exactly_seven} There are exactly seven gorillas at the wild animal park.
\item \label{itm:not_more_than_two} There are not more than two Martians at the wild animal park.
\item \label{itm:martians} Every giraffe at the wild animal park is a Martian.
\end{enumerate}
Now mark each of the following sets of statements consistent or inconsistent.
\begin{longtabu}{p{.1\linewidth}p{.9\linewidth}}
\textbf{Example}: & Statements \ref{itm:at_least_four}, \ref{itm:not_more_than_two}, and \ref{itm:martians}\\
\textbf{Answer}: & Inconsistent. If there are at least four giraffes, and every one of them is Martian, there can't be no more than two Martians in the park.\\
\end{longtabu}



\begin{exercises}
\item Statements \ref{itm:exactly_seven}, \ref{itm:not_more_than_two}, and \ref{itm:martians} \answer{\underline{consistent}}
\item Statements \ref{itm:at_least_four}, \ref{itm:exactly_seven}, \ref{itm:not_more_than_two}, and \ref{itm:martians} \answer{\underline{inconsistent}}
\item Statements \ref{itm:at_least_four}, \ref{itm:exactly_seven}, and \ref{itm:martians}\answer{\underline{consistent}}
\item Statements \ref{itm:at_least_four}, \ref{itm:exactly_seven}, and \ref{itm:not_more_than_two} \answer{\underline{consistent}}
\end{exercises}

\noindent \problempart Consider the following set of statements.
\begin{enumerate}[label=(\alph*)]
\item \label{itm:allmortal} All people are mortal.
\item \label{itm:socperson} Socrates is a person.
\item \label{itm:socnotdie} Socrates will never die.
\item \label{itm:socmortal} Socrates is mortal.
\end{enumerate}
Which combinations of statements form consistent sets? Mark each “consistent” or “inconsistent.”
\begin{exercises}
\item Statements \ref{itm:allmortal}, \ref{itm:socperson}, and \ref{itm:socnotdie}  \answer{\underline{Inconsistent}}
\item Statements \ref{itm:socperson}, \ref{itm:socnotdie}, and \ref{itm:socmortal} \answer{\underline{Inconsistent}}
\item Statements \ref{itm:socperson} and \ref{itm:socnotdie} \answer{\underline{Consistent}}
\item Statements \ref{itm:allmortal} and \ref{itm:socmortal} \answer{\underline{Consistent}}
\item Statements \ref{itm:allmortal}, \ref{itm:socperson}, \ref{itm:socnotdie}, and \ref{itm:socmortal} \answer{\underline{Inconsistent}}
\end{exercises}

\noindent \problempart \label{pr.EnglishCombinations} Which of the following is possible? If it is possible, give an example. If it is not possible, explain why.


\begin{longtabu}{p{.1\linewidth}p{.9\linewidth}}
\textbf{Example}: & A valid argument that has one false premise and one true premise.\\
\textbf{Answer}: & Possible: Example: If Taylor Swift were a kangaroo, she would be a marsupial (true). Taylor Swift is a kangaroo. (False.) Therefore Taylor Swift is a marsupial (false.)\\ &Remember, if an argument is valid, the only thing that can't happen is for it to have all true premises and a false conclusion. So if you don't specify a false conclusion anything is possible.\\
\end{longtabu}



\begin{exercises}
\item A false tautology.

\answer{Impossible. Tautologies, by definition, are always true.}

\item A valid argument that has a false conclusion

\answer{\underline{Possible}. Example: If grass is green, then I am the pope. (False) Grass is green. (True) \therefore  I am the pope. (False)}

\item A valid argument, the conclusion of which is a contradiction

\answer{\underline{Possible}. The conclusion is always false, but if the premises are also always false, you are fine. Example: If A, then not A. \therefore If B, then not B. \\}

\item An invalid argument, the conclusion of which is a tautology

\answer{\underline{Impossible}. If the conclusion is always true, then the there is no way for all the premises to be true and conclusion false.\\}

\item A tautology that is contingent

\answer{\underline{Impossible}. Contradictions, contingencies, and tautologies are exclusive categories. If you are one, you can't be either of the others. \\}


\item Two logically equivalent sentences, both of which are tautologies

\answer{\underline{Possible} In fact, all tautologies are logically equivalent. Logically equivalent sentences always have the same truth value, and all tautologies are always true. \\}


\item Two logically equivalent sentences, one of which is a tautology and one of which is contingent

\answer{\underline{Impossible}. A tautology is always true, but contingent sentences can be false. Therefore they can have different truth values. \\}


\item Two logically equivalent sentences that together are an inconsistent set

\answer{\underline{Possible} Two contradictions are logically equivalent, however it is impossible for them to both be true, because it is impossible for either one to be true. \\}


\item A consistent set of sentences that contains a contradiction

\answer{\underline{Impossible}. The contradiction can never be true, so the whole set cannot never all be true. \\}


\item An inconsistent set of sentences that contains a tautology
\answer{\underline{Possible}. Example: A, Not A, If A then A.}
\end{exercises}

\noindent \problempart Which of the following is possible? If it is possible, give an example. If it is not possible, explain why.
\answer{All answers, except for the last question, are by Ben Sheredos}
\begin{exercises}
\item A valid argument, whose premises are all tautologies, and whose conclusion is contingent
\answer{Not Possible. If the argument is valid, then the conclusion must be true if the premises are true. If the premises are \textit{tautologies}, then the premises are \textit{always} true, and so the conclusion also must always be true.}

\item A valid argument with true premises and a false conclusion
\answer{ \textit{Absolutely not!} This contradicts the very definition of a valid argument.
}
\item A consistent set of sentences that contains two sentences that are not logically equivalent
\answer{ Most definitely. Here are two sentences that are consistent but not logically equivalent: ``Today is a Wednesday'' and ``I like pie.''
}
\item A consistent set of sentences, all of which are contingent
\answer{For sure. See the examples given in the previous answer. Both are contingent (sometimes it's not Wednesday today, and I might've hated pie.)
}
\item A false tautology
\answer{Not possible. By definition, a tautology is always true.
}
\item A valid argument with false premises
\answer{ Yup. Because validity only requires that \textit{if} the premises are true, \textit{then} the conclusion must be true. But all of them could be false, and the argument would remain valid.
}
\item A logically equivalent pair of sentences that are not consistent
\answer{ Careful here. Our definition of consistency is that a set of statements are consistent if they could all be true at the same time. Well, consider the case of 2 statements which are logically equivalent, and which are both \textit{contradictions}. Neither can be true. So they cannot \textit{both} be true. So they are not consistent.
}
\item A tautological contradiction
\answer{ Impossible. This is gibberish-nonsense.
}
\item A consistent set of sentences that are all contradictions
\answer{ Nope: see again \#7 above. If a set of statements contains nothing but contradictions, then none of them can be true. But if none of them can be true, then they cannot be true together, and so they cannot be consistent.
}

\item A valid argument, whose premises are all tautologies, and whose conclusion is contingent.
\answer{Impossible. If the conclusion is contingent, then it could be false, in which case you would have true premises and a false conclusion, which would make the argument invalid.}

\end{exercises}
