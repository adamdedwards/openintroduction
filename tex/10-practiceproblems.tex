%%%% Practice Problems


%\practiceproblems
Throughout the book, you will find a series of practice problems that review and explore the material covered in the chapter. There is no substitute for actually working through some problems, because logic is more about a way of thinking than it is about memorizing facts. %The answers to some of the problems are provided at the end of the book in Appendix \ref{app.solutions}; the problems that are solved in the appendix are marked with a star (\solutions.)

%\noindent\problempart Decide whether the following passages are statements in the logical sense and give reasons for your answers.

\begin{longtabu} to \textwidth {X[1.2,p]X[6,p]}
\textbf{Example}: & Did you follow the instructions? \\
\textbf{Answer}: & Not a statement, a question. \\
\end{longtabu}


\begin{enumerate}
\item England is smaller than China. %%\answerblank{\underline{Statement}}{\vspace{.25in}}
\item Greenland is south of Jerusalem. %%\answerblank{\underline{Statement}}{\vspace{.25in}}
\item Is New Jersey east of Wisconsin? %%\answerblank{\underline{A question, not a Statement.}}{\vspace{.25in}}
\item The atomic number of helium is 2. %%\answerblank{\underline{Statement}}{\vspace{.25in}}
\item The atomic number of helium is $\pi$. %%\answerblank{\underline{Statement}}{\vspace{.25in}}
\item I hate overcooked noodles. %%\answerblank{\underline{Statement}}{\vspace{.25in}}
\item Blech! Overcooked noodles! %%\answerblank{\underline{An exclamation, not a statement.}}{\vspace{.25in}}
\item Overcooked noodles are disgusting.%%\answerblank{\underline{Statement}}{\vspace{.25in}}
\item Take your time. %%\answerblank{\underline{A command, not a Statement}}{\vspace{.25in}}
\item This is the last question. %%\answerblank{\underline{Statement}}{\vspace{.25in}}
\end{enumerate}


%\noindent\problempart Decide whether the following passages are statements in the logical sense and give reasons for your answers.
%\answer{Answers from Ben Sheredos.}
\begin{enumerate}
\item Is this a question? %\answer{\underline{Question, not a statement.}}
\item Nineteen out of the 20 known species of Eurasian elephants are extinct. %\answer{\underline{Statement; has to be true or false (might be false bc 20 is the wrong number, or because they are not extinct, etc.)}}
\item The government of the United Kingdom has formally apologized for the way it treated the logician Alan Turing. %\answer{\underline{ Statement: has to be true or false; they either have or have not apologized}}

\item Texting while driving %\answer{\underline{Not a statement, but a sentence fragment}}
\item Texting while driving is dangerous. %\answer{\underline{Statement; has to be true or false.}}
\item Insanity ran in the family of logician Bertrand Russell, and he had a life-long fear of going mad. %\answer{\underline{Complex, but a statement: both halves are true or false, so is the whole.}}
\item For the love of Pete, put that thing down before someone gets hurt!  %\answer{\underline{Not a statement: First bit is an exclamation, second is a command.}}
\item Don't try to make too much sense of this. %\answer{\underline{Not a statement, a command.}}
\item Never look a gift horse in the mouth.  %\answer{\underline{Not a statement, a command.}}
\item The physical impossibility of death in the mind of someone living  %\answer{\underline{ Not a statement, sentence fragment.}}
\end{enumerate}

%\noindent\problempart Rewrite each of the following arguments in canonical form. Be sure to remove all indicator words and keep the premises and conclusion as complete sentences. Write the indicator words and phrases separately and state whether they are premise or conclusion indicators.

%NTS: when writing these problems, be sure to include a mix of conclusion-first, conclusion-last and conclusion middle, as well as a mix of arguments with true and false premises and a variety of indicator words (or lack thereof).

\begin{longtabu} to \textwidth {X[1,c] X[10,l]}
\textbf{Example}: & \textit{An ancient philosopher writes} We should not be distressed or concerned by the thought of our our own death in any way. Why? Look back on the time before you were born: It is a time you did not exist, but it does not trouble you in any way. The time after you die is also a time when you will not exist, so it shouldn't trouble you either. (Based on Lucretius \citet{Lucretius2001} 3.972---75)\\
%\textbf{Answer}: &
%\begin{kormanize}
%\premise{The time before you were born is a time you did not exist.}
%\premise{You are not troubled by the time before you were born.}
%\premise{The time after you die is also a time you will not exist.}
%\conclusion{ We should not be distressed or concerned by the thought of our our own death.}
%\end{kormanize} \\
%Premise indicator: So &\\
\end{longtabu}

\begin{itemize}

\item \textit{A detective is speaking:~}Henry's finger-prints were found on the stolen computer. So, I infer that Henry stole the computer.

%\answer{
%\begin{kormanize}
%\premise{Henry's finger-prints were found on the stolen computer.}
%\conclusion{ Henry stole the computer.}
%\end{kormanize}
%Conclusion indicator word: So, I infer that


\item \textit{Monica is wondering about her co-workers political opinions} You cannot both oppose abortion and support the death penalty, unless you think there is a difference between fetuses and felons. Steve opposes abortion and supports the death penalty. Therefore Steve thinks there is a difference between fetuses and felons.
		%Conclusion-last

%\answer{
%\begin{kormanize}
%\premise{You cannot both oppose abortion and support the death penalty, unless you think there is a difference between fetuses and felons.}
%\premise{ Steve opposes abortion and supports the death penalty.}
%\conclusion{ Steve thinks there is a difference between fetuses and felons.}
%\end{kormanize}
%Conclusion Indicator: Therefore


\item \textit{The Grand Moff of Earth defense considers strategy} We know that whenever people from one planet invade another, they always wind up being killed by the local diseases, because in 1938, when Martians invaded the Earth, they were defeated because they lacked immunity to Earth's diseases. Also, in 1942, when Hitler's forces landed on the Moon, they were killed by Moon diseases.
		%Conclusion-first

%\answer{
%\begin{kormanize}
%\premise{In 1938, when Martians invaded the Earth, they were defeated because they lacked immunity to Earth's diseases.}
%\premise{In 1942, when Hitler's forces landed on the Moon, they were killed by Moon diseases.}
%\conclusion Whenever people from one planet invade another, they always wind up being killed by the local diseases.}
%\end{kormanize}
%Premise indicator: Because


\item If you have slain the Jabberwock, my son, it will be a frabjous day. The Jabberwock lies there dead, its head cleft with your vorpal sword. This is truly a fabjous day.
%Conclusion-last
%\answer{
%\begin{kormanize}
%\premise{If you have slain the Jabberwock, my son, it will be a frabjous day.}
%\premise{The Jabberwock lies there dead.}
%\conclusion{ This is truly a frabjous day.}
%\end{kormanize}
%Indicators: none

\item \textit{A detective trying to crack a case thinks to herself} Miss Scarlett was jealous that Professor Plum would not leave his wife to be with her. Therefore she must be the killer, because she is the only one with a motive.
%Conclusion-middle
%\answer{
%\begin{kormanize}
%\premise{ Miss Scarlett was jealous that Professor Plum would not leave his wife to be with her.}
%\premise{ Miss Scarlett is the only one with a motive.}
%\conclusion{ Miss Scarlett must be the killer.}
%\end{kormanize}

%Premise Indicator: Because \\
%Conclusion Indicator: Therefore}

\end{itemize}

\textbf{Rewrite each of the following arguments in canonical form. Be sure to remove all indicator words and keep the premises and conclusion as complete sentences. Write the indicator words and phrases separately and state whether they are premise or conclusion indicators.}

\begin{itemize}

%\answer{Answers from Ben Sheredos.}

\item \textit{A pundit is speaking on a Sunday political talk show} Hillary Clinton should drop out of the race for Democratic Presidential nominee. For every day she stays in the race, McCain gets a day free from public scrutiny and the members of the Democratic party get to fight one another. %Conclusion-first

%\answer{
%\begin{kormanize}
%\premise{For every day Hillary Clinton stays in the race, John McCain gets a day free from public scrutiny and the members of the Democratic party get to fight one another.}
%\conclusion{ Hillary Clinton should drop out of the race for Democratic Presidential Nominee.}
%\end{kormanize}

%``For'' could be a premise-indicator, functioning like ``since.''}

\item You have to be smart to understand the rules of Dungeons and Dragons. Most smart people are nerds. So, I bet most people who play D\&D are nerds. %Conclusion-last

% \answer{
%\begin{kormanize}
%\premise{You have to be smart to understand the rules of D\&D.}
%\premise{Most smart people are nerds.}
%\conclusion{ $[I bet]$ most people who play D\&D are nerds.}
%\end{kormanize}

%``So'' is definitely a conclusion-indicator; ``I bet'' is probably part of a conclusion-indicator as well, with the speaker indicating that they think this argument is a bit weak.}

\item Any time the public receives a tax rebate, consumer spending increases. Since the public just received a tax rebate, consumer spending will increase. %Conclusion-last

%\answer{
%\begin{kormanize}
%\premise{Any time the public receives a tax rebate, consumer spending increases.}
%\premise{The public just received a tax rebate.}
%\conclusion{Consumer spending will increase.}
%\end{kormanize}

%``Since'' is a premise-indicator, but the last sentence needs to be split up into premise and conclusion. This would be more clear if the speaker said ``\underline{Since} the public just received a tax rebate, \underline{it follows that} consumer spending will increase.''}


\item Isabelle is taller than Jacob. Kate must also be taller than Jacob, because she is taller than Isabelle. %conclusion-middle

%\answer{
%\begin{kormanize}
%\premise{Isabelle is taller than Jacob.}
%\premise{Kate is taller than Isabelle.}
%\conclusion{ Kate is taller than Jacob.}
%\end{kormanize}
%``Must'' is a conclusion indicator, ``because'' is a premise-indicator, and so the last sentence has to be split up to put this argument into canonical form.}

\end{itemize}




\section{Practice Problems}

\textbf{Identify each passage below as an argument or a nonargument, and give reasons for your answers. If it is a nonargument, say what kind of nonargument you think it is. If it is an argument, write it out in canonical form.}

\begin{longtabu} to \textwidth {X[1.2,p]X[10,p]}
\textbf{Example}: & \textit{One student speaks to another student who has missed class:} The instructor passed out the syllabus at 9:30. Then he went over some basic points about reasoning, arguments and explanations. Then he said we should call it a day. \\
\textbf{Answer}: & Not an argument, because none of the statements provide any support for any of the others. This is probably better classified as a narration because the events are in temporal sequence. \\
\end{longtabu}

\begin{itemize}

\item \textit{An anthropology teacher is speaking to her class:~}Different gangs use different colors to distinguish themselves. Here are some illustrations: biologists tend to wear some blue, while the philosophy gang wears black.

%\answer{Not an argument. Expository passage. The students probably will believe the teacher as soon as she makes an assertion. The word ``illustration'' is also a clue.}

\item The economy has been in trouble recently. And it's certainly true that cell phone use has been rising during that same period. So, I suspect increasing cell phone use is bad for the economy.


%\answer{Argument. The indicator ``so'' is a clue.

%\begin{kormanize}
%\premise{The economy has been in trouble recently.}
%\premise{Cell phone use has been rising during that same period.}
%\conclusion{ Cell phone use is bad for the economy.}
%\end{kormanize}}

\item \textit{At Widget-World Corporate Headquarters:} We believe that our company must deliver a quality product to our customers. Our customers also expect first-class customer service. At the same time, we must make a profit.

%\answer{Not an argument. The speaker is not using any of the propositions as reasons to believe or explain any of the others; rather she is simply asserting various things.}

\item \textit{Jack is at the breakfast table and shows no sign of hurrying. Gill says:} You should leave now. It's almost nine a.m. and it takes three hours to get there.

%\answer{Arguing. Jack's inaction suggests that he does believe that he needs to leave now and so Gill provides reasons that might convince him. Notice that there are no argument flag words or phrases.

%This example also includes the word ``should'' in its conclusion. Words such as ``ought'' and ``should'' indicate that the speaker is trying to get the audience to do or believe something that they are not currently doing or believing.

%\begin{kormanize}
%\premise{It's almost nine a.m.}
%\premise{It takes three hours to get there.}
%\conclusion{You should leave now.}
%\end{kormanize}}

\item \textit{In a text book on the brain:} Axons are distinguished from dendrites by several features, including shape (dendrites often taper while axons usually maintain a constant radius), length (dendrites are restricted to a small region around the cell body while axons can be much longer), and function (dendrites usually receive signals while axons usually transmit them).

%\answer{Not an argument. Expository passage. The features named just fill in the first statement.}

\end{itemize}

\textbf{Identify each passage below as an argument or a nonargument, and give reasons for your answers. If it is a nonargument, say what kind of nonargument you think it is. If it is an argument, write it out in canonical form.}

\begin{enumerate}

\item \textit{Suzi doesn't believe she can quit smoking. Her friend Brenda says} Some people have been able to give up cigarettes by using their will-power. Everyone can draw on their will-power. So, anyone who wants to give up cigarettes can do so.

\item \textit{The words of the Preacher, son of David, King of Jerusalem} I have seen something else under the sun: The race is not to the swift or the battle to the strong, nor does food come to the wise or wealth to the brilliant or favor to the learned; but time and chance happen to them all. (Ecclesiastes 9:11, New International Version)

\item \textit{An economic development expert is speaking.} The introduction of cooperative marketing into Europe greatly increased the prosperity of the farmers, so we may be confident that a similar system in Africa will greatly increase the prosperity of our farmers.

\item \textit{From the CBS News website, US section.} Headline: ``FBI nabs 5 in alleged plot to blow up Ohio bridge.'' Five alleged anarchists have been arrested after a months-long sting operation, charged with plotting to blow up a bridge in the Cleveland area, the FBI announced Tuesday. CBS News senior correspondent John Miller reports the group had been involved in a series of escalating plots that ended with their arrest last night by FBI agents. The sting operation supplied the anarchists with what they thought were explosives and bomb-making materials. At no time during the case was the public in danger, the FBI said. \citep{CBSNews2012}

\item \textit{At a school board meeting.} Since creationism can be discussed effectively as a scientific model, and since evolutionism is fundamentally a religious philosophy rather than a science, it is unsound educational practice for evolution to be taught and promoted in the public schools to the exclusion or detriment of special creation. (Kitcher \cite*{Kitcher1982}, p. 177, citing Morris \cite*{Morris1975}.)

\end{enumerate}
