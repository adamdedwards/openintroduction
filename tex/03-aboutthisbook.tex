
\setlength{\parindent}{1em}
\chapter{About this Book}

This book was created by combining two previous books on logic and critical thinking, both made available under a Creative Commons license, and then adding some material so that the coverage matched that of commonly used logic textbooks.

Cathal Woods' \textit{Introduction to Reasoning} \parencite*{Woods2014} formed the basis of most of Part III: Critical Thinking and Part IV: Inductive and Scientific Reasoning of the complete version of this text. \nix{[Details]} Some of \textit{Introduction to Reasoning}: Propositional and Categorical Reasoning was incorporated into Part III: Formal Logic. Parts of the section ``The Venn Diagram Method'' were folded into sections 12.1 and 12.3.% [more details]

P.D. Magnus' \textit{For All} X \parencite*{Magnus2008} formed the basis of Part II: Formal Logic, but the development was slightly more indirect. I began using \textit{For All} X in my own logic classes in 2009, but I quickly realized I needed to make changes to make it appropriate for the community college students I was teaching. In 2010 I began developing \textit{For All} X \textit{: The Lorain County Remix} and using it in my classes. The main change I made was to separate the discussions of sentential and quantificational logic and to add exercises. It is this remixed version that became the basis for Part III: Formal Logic complete version of this text. 

 Complete version information is available at \textbookhomepage.


 \begin{adjustwidth}{2em}{0em} 
 J. Robert Loftis \\
\noindent \emph{Elyria, Ohio, USA} 
\end{adjustwidth}

