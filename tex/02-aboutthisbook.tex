
\setlength{\parindent}{1em}
\chapter{About this Book}

This book was created by combining two previous books on logic and critical thinking, both made available under a Creative Commons license, and then adding some material so that the coverage matched that of commonly used logic textbooks.

P.D. Magnus' \textit{For All} X \parencite*{Magnus2008} formed the basis of Part \ref{part:formal_logic}: Formal Logic. I began using \textit{For All} X in my own logic classes in 2009, but I quickly realized I needed to make changes to make it appropriate for the community college students I was teaching. In 2010 I began developing \textit{For All} X \textit{: The Lorain County Remix} and using it in my classes. The main change I made was to separate the discussions of sentential and quantificational logic and to add exercises. It is this remixed version that became the basis for Part \ref{part:formal_logic}: Formal Logic complete version of this text. 

Similarly, Part \ref{part:CT}: Critical Thinking\iflabelexists{part:inductive_scientific}{ and Part \ref{part:inductive_scientific}: Inductive and Scientific Reasoning.}{ } grew out of Cathal Woods' \textit{Introduction to Reasoning}. In the Spring of 2011, I began to use an early version of this text (\cite{Woods2011}) in my critical thinking courses. I kept up with the updates and changes to the text until the release of \cite{Woods2014}, all the while gradually merging the material with the work in \textit{For All X}. After that point, my version forks from Woods's.

On May 20, 2016, I posted the combined textbook to Github and all subsequent changes have been tracked there: \url{https://github.com/rob-helpy-chalk/openintroduction}



 \begin{adjustwidth}{2em}{0em} 
 J. Robert Loftis \\
\noindent \emph{Elyria, Ohio, USA} 
\end{adjustwidth}

