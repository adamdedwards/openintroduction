\newglossaryentry{logic}
{
name=logic,
description={The study of the structure of inference or of the normative rules for reasoning.}
}
\newglossaryentry{premises}
{
name=premises,
description={The part of an argument that provides evidence.}
}

\newglossaryentry{conclusion}
{
name=conclusion,
description={The part of an argument that is being provided evidence.}
}

\newglossaryentry{metareasoning}
{
name=metareasoning,
description={Using reasoning to study reasoning. See also \emph{metacognition}.}
}

\newglossaryentry{metacognition}
{
name=metacognition,
description={Thought processes that are applied to other thought processes See also \emph{metareasoning}.}
}

\newglossaryentry{content neutrality}
{
name=content neutrality,
description={the feature of the study of logic that makes it indifferent to the topic being argued about. If a method of argument is considered rational in one domain, it should be considered rational in any other domain, all other things being equal.}
}

\newglossaryentry{formal logic}
{
name=formal logic,
description={A way of studying logic that achieves content neutrality by replacing parts of the arguments being studied with abstract symbols. Often this will involve the construction of full formal languages.}
}

\newglossaryentry{critical thinking}
{
name=critical thinking,
description={The use of metareasoning to improve our reasoning in practical situations. Sometimes the term is also used to refer to the results of this effort at self improvement, that is, reasoning in practical situations that has been sharpened by reflection and metareasoning.}
}

\newglossaryentry{critical thinker}
{
name=critical thinker,
description={A person who has both sharpened their reasoning abilities using metareasoning and deploys those sharpened abilities in real world situations.}
}

\newglossaryentry{informal logic}
{
name=informal logic,
description={The study of arguments given in ordinary language.}
}

\newglossaryentry{rhetoric}
{
name=rhetoric,
description={The study of effective persuasion.}
}

%%%%%%%%%%%%%%%%%%%%%%%%%%%%%%%%%%%%%%%%%%%%%%%%%%%%%%%%%%%%%%%%%%%%%%%%%%%%%%%%%%%%%%%%%%%%%%%%%%%%%%%%%%%%%%%%%%%%

\newglossaryentry{statement}
{
name=statement,
description={A unit of language that can be true or false.}
}

\newglossaryentry{truth evaluable}
{
name=truth evaluable,
description={A property of some objects (such as bits of language, maps, or diagrams) that means they can be appropriately assessed as either true or false.}
}

\newglossaryentry{practical argument}
{
name=practical argument,
description={An argument whose conclusion is a statement that someone should do something.}
}


\newglossaryentry{argument}
{
name=argument,
description={A collection of statements, called the premises, that provides evidential support to another statement, called the conclusion.}
}

\newglossaryentry{premise}
{
name=premise,
description={A statement in an argument that provides evidence for the conclusion.}
}

\newglossaryentry{conclusion}
{
name=conclusion,
description={The statement that is being supported in an argument.}
}

\newglossaryentry{premise indicator}
{
name=Premise Indicator,
description={A word or phrase such as ``because'' used to indicate that what follows is the premise of an argument.}
}

\newglossaryentry{conclusion indicator}
{
name=Conclusion Indicator,
description={A word or phrase such as ``therefore'' used to indicate that what follows is the conclusion of an argument.}
}

\newglossaryentry{canonical form}
{
name=Canonical form,
description={A method for representing arguments where each premise is numbered and written on a separate line. The premises are followed by a horizontal bar and then the conclusion. Statements in the argument may be paraphrased for brevity and indicator words are removed.}
}

\newglossaryentry{inference}
{
name=inference,
description={the act of coming to believe a conclusion on the basis of some set of premises.}
}

\newglossaryentry{simple statement of belief}
{
name=simple statement of belief,
description={A kind of nonargumentative passage where the speaker simply asserts what they believe without giving reasons. }
}

\newglossaryentry{expository passage}
{
name=expository passage,
description={A nonargumentative passage that organizes statements around a central theme or topic statement.}
}

\newglossaryentry{narrative}
{
name=narrative,
description={A nonargumentative passage that describes a sequence of events or actions.}
}

\newglossaryentry{socratic elenchus}
{
name=Socratic elenchus,
description={A style of argumentation in which two statements (usually definitions) are shown to be contradictory by means of critical questioning, so an alternative is sought.}
}
