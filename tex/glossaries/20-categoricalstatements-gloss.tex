\newglossaryentry{quantified categorical statement}
{
  name=quantified categorical statement,
  description={A statement that makes a claim about a certain quantity of the members of a class or group.}
}

\newglossaryentry{quantifier}
{
  name=quantifier,
  description={The part of a categorical sentence that specifies a portion of a class.}
}

\newglossaryentry{subject class}
{
  name=subject class,
  description={The first class named in a quantified categorical statement.}
}

\newglossaryentry{predicate class}
{
  name=predicate class,
  description={The second class named in a quantified categorical statement.}
  }

\newglossaryentry{copula}
{
  name=copula,
  description={The form of the verb ``to be'' that links subject and predicate.}
}

\newglossaryentry{quantity}
{
name=quantity,
description={The portion of the subject class described by a categorical statement. Generally ``some'' or ``none.''}
}

\newglossaryentry{universal}
{
name=universal,
description={The quantity of a statement that uses the quantifier ``all.''}
}

\newglossaryentry{particular}
{
name=particular,
description={The quantity of a statement that uses the quantifier ``some.''}
}

\newglossaryentry{quality}
{
name=quality,
description={The status of a categorical statement as affirmative or negative.}
}

\newglossaryentry{negative}
{
name=negative,
description={The quality of a statement containing a ``not'' or ``no.''}
}

\newglossaryentry{affirmative}
{
name=affirmative,
description={The quality of a statement without a ``not'' or a ``no.''}
}

\newglossaryentry{statement mood}
{
name=statement mood,
description={The classification of a categorical statement based on its quantity and quality.}
}

\newglossaryentry{mood-A statement}
{
name=mood-A statement,
description={A quantified categorical statement of the form ``All $S$ are $P$.''}
}

\newglossaryentry{mood-E statement}
{
name=mood-E statement,
description={A quantified categorical statement of the form ``No $S$ are $P$.''}
}

\newglossaryentry{mood-I statement}
{
name=mood-I statement,
description={A quantified categorical statement of the form ``Some $S$ are $P$.''}
}

\newglossaryentry{mood-O statement}
{
name=mood-O statement,
description={A quantified categorical statement of the form ``Some $S$ are not $P$.''}
}

\newglossaryentry{distribution}
{
name=distribution,
description={A property of the terms of a categorical statement that is present when the statement makes a claim about the whole term.}
}

\newglossaryentry{Venn diagram}
{
name=Venn diagram,
description={A diagram that represents categorical statements using circles that stand for classes.}
}

\newglossaryentry{translation key}
{
name=translation key,
description={A list that assigns English phrases or sentences to variable names. Also called a ``symbolization key''  or simply a ``dictionary.''}
}

\newglossaryentry{logically structured English}
{
name=logically structured English,
description={English that has been regimented into a standard form to make its logical structure clear and to remove ambiguity. A stepping stone to full-fledged formal languages.}
}

\newglossaryentry{standard form for a categorical statement}
{
name=standard form for a categorical statement,
description={A categorical statement that has been put into logically structured English, with the following elements in the following order: (1) The quantifiers ``all,'' ``some,'' or ``no''; (2) the subject term; (3) the copula ``are'' or ``are not''; and (4) the predicate term.}
}

\newglossaryentry{truth value}
{
  name=truth value,
  description={The status of a statement with relationship to truth. For  this textbook, this means the status of a statement as true or false}
}

\newglossaryentry{conversion}
{
name=conversion,
description={The process of changing a sentence by reversing the subject and predicate.}
}

\newglossaryentry{complement}
{
name=complement,
description={The class of everything that is not in a given class.}
}

\newglossaryentry{obversion}
{
name=obversion,
description={The process of transforming a categorical statement by changing its quality and replacing the predicate with its complement.}
}

\newglossaryentry{contraposition}
{
name=contraposition,
description={The process of transforming a categorical statement by reversing subject and predicate and replacing them with their complements.}
}
