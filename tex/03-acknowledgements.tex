
\chapter{Acknowledgments}

Thank you to all of the mentors and students that have supported this project and have helped it become a reality. Special thanks to Jonathan Livengood for the inspiration to write a free logic textbook and to Ashli Anda for her inexhaustable support.

\begin{adjustwidth}{2em}{0em}
Adam Edwards \\
\noindent \emph{Champaign, Illinois}
\end{adjustwidth}

\pagebreak


Thanks first of all go to the authors of the textbooks here stitched together: P.D. Magnus for \emph{For All} X and Cathal Woods for \emph{Introduction to Reasoning}. My thanks go to them for writing the excellent textbooks that have been incorporated into this one, for making those publicly available under Creative Commons licenses, and for giving their blessing to this derivative work.

In general, this book would not be possible without a culture of sharing knowledge.   The book was typeset using \LaTeX$2\varepsilon$ developed by Leslie Lamport. Lamport was building on \TeX by Donald Knuth. Peter Selinger built on what Lamport made by developing the Fitch typesetting format that the proofs were laid out in. Diagrams were made in TikZ by Till Tantu. All of these are coding systems are not only freely available online, they have extensive user support communities. Add-on packages are designed, manuals are written, questions are answered in discussion forums, all by people who are donating their time and expertise.

The culture of sharing isn't just responsible for the typesetting of this book; it was essential to the content. Essential background information comes from the free online \textit{Stanford Encyclopedia of Philosophy}. Primary sources from the history of logic came from \textit{Project Gutenberg}. Logicians, too, can and should create free knowledge.

Many early adopters of this text provided invaluable feedback, including Jeremy Dolan, Terry Winant, Benjamin Lennertz, Ben Sheredos, and Michael Hartsock. Lennertz, in particular, provided useful edits. Helpful comments were also made by Ben Cordry, John Emerson, Andrew Mills, Nathan Smith, Vera Tobin, Cathal Woods, and many more that I have forgot to mention, but whose emails are probably sitting on my computer somewhere.

I would also like to thank Lorain County Community College for providing the sabbatical leave that allowed me to write the sections of this book on Aristotelian logic. Special thanks goes to all the students at LCCC who had to suffer through earlier versions of this work and provided much helpful feedback. Most importantly, I would like to thank Molly, Caroline and Joey for their incredible love and support.

 \begin{adjustwidth}{2em}{0em}
 J. Robert Loftis \\
\noindent \emph{Elyria, Ohio, USA}
\end{adjustwidth}

\pagebreak


Intellectual debts too great to articulate are owed to scholars too many to enumerate. At different points in the work, readers might detect the influence of various works of Aristotle, Toulmin (especially \cite{Toulmin1958}), Fisher and Scriven \cite{Fisher1997}, Walton (especially \cite{Walton1996}), Epstein \cite{Epstein2002}, Johnson-Laird (especially \cite{johnson2006we}), Scriven \cite{Scriven1962}, Giere \cite{giere1997understanding} and the works of the Amsterdam school of pragma-dialectics \cite{van2002argumentation}.

Thanks are due to Virginia Wesleyan College for providing me with Summer Faculty Development funding in 2008 and 2010 and a Batten professorship in 2011. These funds, along with some undergraduate research funds (also provided by VWC), allowed me to hire students Gaby Alexander (2008), Ksera Dyette (2009), Mark Jones (2008), Andrea Medrano (2011), Lauren Perry (2009), and Alan Robertson (2010). My thanks to all of them for their hard work and enthusiasm.

For feedback on the text, thanks are due to James Robert (Rob) Loftis (Lorain County Community College) and Bill Roche (Texas Christian University). Answers (to exercises) marked with ``(JRL)'' are by James Robert Loftis.

Particular thanks are due to my (once) Ohio State colleague Bill Roche. The book began as a collection of lecture notes, combining work by myself and Bill.

\begin{adjustwidth}{2em}{0em}
Cathal Woods\\
\noindent\emph{Norfolk, Virginia, USA}\\
\noindent(Taken from \emph{Introduction to Reasoning} (\citeyear{Woods2014}))
\end{adjustwidth}


\vspace{3cm}


\noindent The author would like to thank the people who made this project possible. Notable among these are Cristyn Magnus, who read many early drafts; Aaron Schiller, who was an early adopter and provided considerable, helpful feedback; {and} Bin Kang, Craig Erb, Nathan Carter, Wes McMichael, and the students of Introduction to Logic, who detected various errors in previous versions of the book.


\begin{adjustwidth}{2em}{0em}
P.D. Magnus \\
\noindent\emph{Albany, New York, USA}\\
\noindent(Taken from \emph{For All X} (\citeyear{Magnus2008}))
\end{adjustwidth}
