%\addcontentsline{toc}{chapter}{C\ Quick Reference}
%\pagestyle{plain}
%{\LARGE \bf Quick Reference}
%\setlength{\parindent}{0em}

\chapter{Quick Reference}

%%%%%%%%%%%%%%%%%%%%%%%%%%%%%%%%%%%%% Characteristic Truth Tables %%%%%%%%%%%%%%%%%%%%%%%%%%

\section*{Characteristic Truth Tables}
\label{app.CharacteristicTTs}

\begin{tabular}{c|c}
\script{A} & \enot\script{A}\\
\hline
T & F\\
F & T 
\end{tabular}
\hspace{1in}
\begin{tabular}{c|c|c|c|c|c}
\script{A} & \script{B} & \script{A}\eand\script{B} & \script{A}\eor\script{B} & \script{A}\eif\script{B} & \script{A}\eiff\script{B}\\
\hline
T & T & T & T & T & T\\
T & F & F & T & F & F\\
F & T & F & T & T & F\\
F & F & F & F & T & T
\end{tabular}


%\vfill

%**************************************
% 		Symbolization 						*
%**************************************

\section*{Symbolization}
\label{app.symbolization}

\iflabelexists{part:quant_logic}{%This heading is only necessary if there is a quantificational part to distinguish it from
\subsection*{Sentential Connectives (chapter \ref{chap:SL})} 
}{}

\begin{longtabu}{X[1,r,m]X[1,1,m]}
It is not the case that $P$. & $\enot P$\\
Either $P$, or $Q$. & $(P \eor Q)$\\
Neither $P$, nor $Q$. & $\enot(P \eor Q)$\ or \ $(\enot P \eand \enot Q)$\\
Both $P$, and $Q$. & $(P \eand Q)$\\
If $P$, then $Q$. & $(P \eif Q)$\\
$P$ only if $Q$. & $(P \eif Q)$\\
$P$ if and only if $Q$. & $(P \eiff Q)$\\
Unless $P$, $Q$. $P$ unless $Q$. & $(P \eor Q)$\\
\\
\end{longtabu}

\iflabelexists{part:quant_logic}{  %Optional section if quantification logic is included
\subsection*{Predicates (chapter \ref{chap:QL})}
\label{SymbolizingPredicates}

\begin{longtabu}{X[1,r,m]X[1,1,m]}
All $F$s are $G$s. & $\forall x(Fx \eif Gx)$\\
Some $F$s are $G$s. & $\exists x(Fx \eand Gx)$\\
Not all $F$s are $G$s. & $\enot\forall x(Fx \eif Gx)$\ or\ $\exists x(Fx \eand \enot Gx)$\\
No $F$s are $G$s. & $\forall x(Fx \eif\enot Gx)$\ or\ $\enot\exists x(Fx \eand Gx)$\\
\end{longtabu}
\subsection*{Identity (section \ref{sec.identity})}
\begin{longtabu}{X[1.6,r,m]X[1.5,1,m]}
Only $j$ is $G$. & $\forall x(Gx \eiff x=j)$\\
Everything besides $j$ is $G$. & $\forall x(x \neq j \eif Gx)$\\
$j$ is more $R$ than anyone else. & $\forall x(x\neq j \eif Rjx)$\\
The $F$ is $G$. & $\exists x(Fx \eand \forall y(Fy \eif x=y) \eand Gx)$ 
\end{longtabu}
\vspace{-12pt}
\begin{longtabu}{X[r,p,m]X[1,p,m]}
\multicolumn{2}{c}{\emph{`The F is not G' can be translated two ways:}} \\
It is not the case that the F is G. (wide)& $\enot\exists x(Fx \eand \forall y(Fy \eif x=y) \eand Gx)$\\
The $F$ is non-$G$. (narrow) & $\exists x(Fx \eand \forall y(Fy \eif x=y) \eand \enot Gx)$
\end{longtabu}

}{}

\iflabelexists{part:quant_logic}{  %Optional section if quantification logic is included

%****************************************
%* Using Identity to Symbolize Quantities
%****************************************

% BEGIN: symbolizing cardinality

%\newpage
\section*{Using identity to symbolize quantities}
\subsection*{There are at least \blank\ $F$s.}

\label{summary.atleast}
\begin{longtabu}{X[1,r]X[7,1]}
\textbf{one} & $\exists xFx$ \\

\textbf{two} & $\exists x_1\exists x_2(Fx_1 \eand Fx_2 \eand x_1 \neq x_2)$\\

\textbf{three} & $\exists x_1\exists x_2\exists x_3(Fx_1 \eand Fx_2 \eand Fx_3 \eand x_1 \neq x_2 \eand x_1 \neq x_3 \eand x_2 \neq x_3)$\\

\textbf{four} & $\exists x_1\exists x_2\exists x_3\exists x_4 (Fx_1 \eand Fx_2 \eand Fx_3 \eand Fx_4 \eand x_1 \neq x_2 \eand x_1 \neq x_3 \eand x_1 \neq x_4 \eand x_2 \neq x_3 \eand x_2 \neq x_4 \eand x_3 \neq x_4)$\\

\textbf{n} & $\exists x_1\cdots\exists x_n(Fx_1 \eand\cdots\eand Fx_n \eand x_1 \neq x_2 \eand\cdots\eand x_{n-1}\neq x_n)$ \\

\end{longtabu}


\subsection*{There are at most \blank\ $F$s.}
\label{summary.atmost}

One way to say `at most $n$ things are $F$' is to put a negation sign in front of one of the symbolizations above and say $\enot$ `at least $n+1$ things are $F$.' Equivalently:

\begin{longtabu}{X[1,r]X[7,1]}
\textbf{one} &  $\forall x_1\forall x_2\bigl[(Fx_1 \eand Fx_2) \eif x_1=x_2\bigr]$  \\
\vspace{6pt}  
\textbf{two}  & \vspace{3pt} $\forall x_1\forall x_2\forall x_3\bigl[(Fx_1 \eand Fx_2 \eand Fx_3) \eif (x_1=x_2 \eor x_1=x_3 \eor x_2=x_3)\bigr]$ \\

\textbf{three} & $\forall x_1\forall x_2\forall x_3\forall x_4\bigl[(Fx_1 \eand Fx_2 \eand Fx_3 \eand Fx_4) \eif (x_1=x_2 \eor x_1=x_3 \eor x_1=x_4 \eor x_2=x_3 \eor x_2=x_4 \eor x_3=x_4)\bigr]$ \\

\textbf{n} & $\forall x_1\cdots\forall x_{n+1}
\bigl[(Fx_1\eand \cdots \eand Fx_{n+1}) \eif (x_1=x_2 \eor \cdots \eor x_n=x_{n+1})\bigr]$ 
\end{longtabu}

\subsection*{There are exactly \blank\ $F$s.}
\label{summary.exactly}

One way to say `exactly $n$ things are $F$' is to conjoin two of the symbolizations above and say `at least $n$ things are $F$' \eand\ `at most $n$ things are $F$.' The following equivalent formulae are shorter:
\begin{longtabu}{X[1,r]X[7,1]}
\textbf{zero} & $\forall x\enot Fx$ \\

\textbf{one} & $\exists x\bigl[Fx \eand \enot\exists y(Fy \eand x\neq y)\bigr]$ \\

\textbf{two} &  $\exists x_1\exists x_2\bigl[Fx_1 \eand Fx_2 \eand x_1 \neq x_2 \eand \enot\exists y\bigl(Fy \eand y\neq x_1 \eand y \neq x_2\bigr) \bigr]$ \\

\textbf{three} & $\exists x_1\exists x_2\exists x_3\bigl[Fx_1 \eand Fx_2 \eand Fx_3 \eand x_1 \neq x_2 \eand x_1 \neq x_3 \eand x_2 \neq x_3 \eand \enot\exists y(Fy \eand y \neq x_1 \eand y \neq x_2 \eand y\neq x_3) \bigr]$ \\

\textbf{n} & $\exists x_1\cdots\exists x_n\bigl[Fx_1 \eand\cdots\eand Fx_n  \eand x_1 \neq x_2 \eand\cdots\eand x_{n-1}\neq x_n \eand  \enot\exists y(Fy \eand y\neq x_1 \eand \cdots \eand y\neq x_n)\bigr]$ \\
%\item[one] $\exists x\forall y\bigl[Fx \eand (Fy \eif y = x)\bigr]$
%\item[two] $\exists x\exists y\forall z\Bigl(Fx \eand Fy \eand \bigl[Fz \eif (z=x \eor z=y)\bigr] \eand x \neq y\Bigr)$
%\item[three] $\exists x_1\exists x_2\exists x_3\forall y\Bigl(Fx_1 \eand Fx_2 \eand Fx_3 \eand [Fy \eif (y=x_1 \eor y=x_2 \eor y=x_3)] \eand x_1 \neq x_2 \eand x_1 \neq x_3 \eand x_2 \neq x_3\Bigr)$
%\item[n] $\exists x_1\cdots\exists x_n\forall y\Bigl(Fx_1 \eand\cdots\eand Fx_n \eand \bigl[Fy \eif (y=x_1 \eor \cdots \eor y=x_n)\bigr] \eand x_1 \neq x_2 \eand\cdots\eand x_{n-1}\neq x_n\Bigr)$ 
\end{longtabu}

\subsection*{Specifying the size of the UD}

Removing $F$ from the symbolizations above produces sentences that talk about the size of the UD. For instance, `there are at least 2 things (in the UD)' may be symbolized as $\exists x\exists y(x \neq y)$.

%\begin{table}
%	Sometimes it is easier to show something by providing proofs than it is by providing models. Sometimes it is the other way round.
%	\begin{center}
%	\begin{tabular*}{\textwidth}{p{10em}|p{10em}|p{10em}|}
%	\cline{2-3}
%	 & {\centerline{YES}} & {\centerline{NO}}\\
%	\cline{2-3}
%	Is \script{A} a tautology? & prove $\vdash\script{A}$ & give a model in which \script{A} is false\\
%	\cline{2-3}
%	Is \script{A} a contradiction? &  prove $\vdash\enot\script{A}$ & give a model in which \script{A} is true\\
%	\cline{2-3}
%	Is \script{A} contingent? & give a model in which \script{A} is true and another in which \script{A} is false & prove $\vdash\script{A}$ or $\vdash\enot\script{A}$\\
%	\cline{2-3}
%	Are \script{A} and \script{B} equivalent? & prove \mbox{$\script{A}\vdash\script{B}$} and \mbox{$\script{B}\vdash\script{A}$}  & give a model in which \script{A} and \script{B} have different truth values\\
%	\cline{2-3}
%	Is the set \model{A} consistent? & give a model in which all the sentences in \model{A} are true & taking the sentences in \model{A}, prove \script{B} and \enot\script{B}\\
%	\cline{2-3}
%	Is the argument \mbox{`\script{P}, \therefore\ \script{C}'} valid? & prove $\script{P}\vdash\script{C}$ & give a model in which \script{P} is true and \script{C} is false\\
%	\cline{2-3}
%	\end{tabular*}
%	\end{center}
%\end{table}

}{} %End quantifier section

%%%%%%%%%%%%%%%%%%%%%%%%%%%%%%%%%%%%% % Basic Proof Rules 
%%%%%%%%%%%%%%%%%%%%%%%%%%
%
%% eliminate page numbers
%%\pagestyle{empty}
%%\twocolumn
%
%
%%  BEGIN: Rules of proof
%% change margins so that all the rules will fit
%\setlength{\topmargin}{0 in}
%\setlength{\headheight}{0 in}
%\setlength{\headsep}{0 in}
%\setlength{\textheight}{9 in}
%%\setlength{\evensidemargin}{0.25 in}
%%\setlength{\oddsidemargin}{0.25 in}
%\setlength{\textwidth}{6 in}
%\newpage
%% This starts a new page and skips a page if necessary so as
%% to start on an even numbered page.
%% That way, the rules of proof will be on facing pages.
%% It fills it in with a somewhat gratuitous reference table.
%\ifthenelse{\isodd{\thepage}}{
%%	\ \vspace{2 in}\par\centerline{[ This page intentionally left blank. ]}	
%%	\newpage
%}{}


\label{ProofRules}
\section*{Basic Rules of Proof}

\begin{longtabu}{X[1,l]X[1,1]}
\textsc{Reiteration} \\ 

\begin{proof}	\have[m]{a}{\script{A}}
	\have[\ ]{c}{\script{A}} \by{R}{a}\end{proof}

&
\\
\multicolumn{2}{l}{\vspace{-6pt}\textsc{Conjunction Introduction}}\\

\begin{proof}	
\have[m]{a}{\script{A}}
\have[n]{b}{\script{B}}	
\have[\ ]{c}{\script{A}\eand\script{B}} \ai{a, b}
\end{proof}

&

\begin{proof}
\have[m]{a}{\script{A}}
\have[n]{b}{\script{B}}	
\have[\ ]{c}{\script{B}\eand\script{A}} \ai{a, b}
\end{proof}
\vspace{6pt}
\\

\multicolumn{2}{l}{\vspace{-6pt}\textsc{Conjunction Elimination}}\\

\begin{proof}	
\have[m]{ab}{\script{A}\eand\script{B}}	
\have[\ ]{a}{\script{A}} \ae{ab}
\end{proof}

&

\begin{proof}	
\have[m]{ab}{\script{A}\eand\script{B}}	
\have[\ ]{a}{\script{B}} \ae{ab}
\end{proof}
\vspace{6pt}
\\

\multicolumn{2}{l}{\vspace{-6pt}\textsc{Disjunction Introduction}}\\

\begin{proof}
	\have[m]{a}{\script{A}}
	\have[\ ]{ab}{\script{A}\eor\script{B}}\oi{a}
\end{proof}

&

\begin{proof}
	\have[m]{a}{\script{A}}
	\have[\ ]{ab}{\script{B}\eor\script{A}}\oi{a}
\end{proof}
\vspace{6pt}
\\

\multicolumn{2}{l}{\vspace{-6pt}\textsc{Disjunction Elimination}}\\

\begin{proof}
	\have[m]{ab}{\script{A}\eor\script{B}}
	\have[n]{nb}{\enot\script{B}}
	\have[\ ]{a}{\script{A}} \oe{ab,nb}
\end{proof}
&
\begin{proof}
	\have[m]{ab}{\script{A}\eor\script{B}}
	\have[n]{na}{\enot\script{A}}
	\have[\ ]{b}{\script{B}} \oe{ab,nb}
\end{proof}
\vspace{6pt}
\\
\multicolumn{2}{l}{\vspace{-6pt}\textsc{Conditional Introduction}}\\

\begin{proof}
	\open
		\hypo[m]{a}{\script{A}} \by{want \script{B}}{}
		\have[n]{b}{\script{B}}
	\close
	\have[\ ]{ab}{\script{A}\eif\script{B}}\ci{a-b}
\end{proof}
&\vspace{6pt}
\\
\multicolumn{2}{l}{\vspace{-6pt}\textsc{Conditional Elimination}}\\

\begin{proof}
	\have[m]{ab}{\script{A}\eif\script{B}}
	\have[n]{a}{\script{A}}
	\have[\ ]{b}{\script{B}} \ce{ab,a}
\end{proof}
\vspace{6pt}
\\
\multicolumn{2}{l}{\vspace{-6pt}\textsc{Biconditional Introduction}}\\

\begin{proof}
	\open
		\hypo[m]{a1}{\script{A}} \by{want \script{B}}{}
		\have[n]{b1}{\script{B}}
	\close
	\open
		\hypo[p]{b2}{\script{B}} \by{want \script{A}}{}
		\have[q]{a2}{\script{A}}
	\close
	\have[\ ]{ab}{\script{A}\eiff\script{B}}\bi{a1-b1,b2-a2}
\end{proof}
&
\vspace{6pt}\\
\multicolumn{2}{l}{\vspace{-6pt}\textsc{Biconditional Elimination}}\\

\begin{proof}
	\have[m]{ab}{\script{A}\eiff\script{B}}
	\have[n]{a}{\script{B}}
	\have[\ ]{b}{\script{A}} \be{ab,a}
\end{proof}
&
\begin{proof}
	\have[m]{ab}{\script{A}\eiff\script{B}}
	\have[n]{a}{\script{A}}
	\have[\ ]{b}{\script{B}} \be{ab,a}
\end{proof}
\vspace{6pt}
\\

\multicolumn{2}{l}{\vspace{-6pt}\textsc{Negation Introduction}}\\

\begin{proof}
	\open
		\hypo[m]{a}{\script{A}} \by{for reductio}{}
		\have[n][-1]{b}{\script{B}}
		\have{nb}{\enot\script{B}}
	\close
	\have[\ ]{na}{\enot\script{A}}\ni{a-nb}
\end{proof}
&
\vspace{6pt}
\\
\multicolumn{2}{l}{\vspace{-6pt}\textsc{Negation Elimination}}\\

\begin{proof}
	\open
		\hypo[m]{na}{\enot\script{A}} \by{for reductio}{}
		\have[n][-1]{b}{\script{B}}
		\have{nb}{\enot\script{B}}
	\close
	\have[\ ]{a}{\script{A}}\ne{na-nb}
\end{proof}
\\
\end{longtabu}


%********************************   
%*  More rules 
%********************************

\iflabelexists{part:quant_logic}{  %Optional section if quantification logic is included

\newpage[4]
\section*{Quantifier Rules}
\vspace{-9pt}
\textsc{Existential Introduction} \vspace{.5ex}
\begin{proof}	\have[m]{a}{\script{A}}
	\have[\ ]{c}{\exists \script{x}\script{A}[\script{x}||\script{c}]} \Ei{a}\end{proof} \vspace{.5ex}
\script{x} may replace some or all occurrences of $c$ in \script{A}.\\ 

\vspace{18pt}
\noindent\textsc{Existential Elimination}\vspace{.5ex}
\vspace{-9pt}
\begin{proof}	\have[m]{a}{\exists \script{x}\script{A}}
	\open	
		\hypo[n]{b}{\script{A}[\script{c}|\script{x}]}
		\have[p]{c}{\script{B}}
	\close
	\have[\ ]{d}{\script{B}} \Ee{a,b-c}\end{proof}\vspace{.5ex}
The constant \script{c} must not appear in $\exists\script{x}\script{A}$, in \script{B}, or in any undischarged assumption.\\

\vspace{18pt}
\noindent\textsc{Universal Introduction}\vspace{.5ex}
\vspace{-9pt}
\begin{proof}	\have[m]{a}{\script{A}}
	\have[\ ]{c}{\forall \script{x}\script{A}[\script{x}|\script{c}]} \Ai{a}\end{proof} \vspace{.5ex}
\script{c} must not occur in any undischarged assumptions.\\

\vspace{18pt}
\noindent\textsc{Universal Elimination} \vspace{.5ex}
\vspace{-9pt}
\begin{proof}	\have[m]{a}{\forall \script{x}\script{A}}
	\have[\ ]{c}{\script{A}[\script{c}|\script{x}]} \Ae{a}\end{proof} \vspace{.5ex}



\section*{Identity Rules}

\begin{longtabu}{X[1,l]X[1,l]}
\textsc{Identity Introduction}\vspace{.5ex}
\begin{proof}
	\have[\ \,\,\,]{x}{\script{c}=\script{c}} \by{=I}{}
\end{proof}\vspace{.5ex}

&

\textsc{Identity Elimination}\vspace{.5ex}
\begin{proof}	\have[m]{e}{\script{c}=\script{d}}
	\have[n]{a}{\script{A}}
	\have[\ ]{ea1}{\script{A}[\script{c}||\script{d}]} \by{=E}{e,a}	\have[\ ]{ea2}{\script{A}[\script{d}||\script{c}]} \by{=E}{e,a}
\end{proof} \vspace{9pt}

One constant may replace some or all occurrences of the other.\vspace{.5ex}

\end{longtabu}
}{}

\iflabelexists{whole_slproof_chap}{%rules in full SL proof chapter

\begin{multicols}{2}
\section*{Derived Rules}

\textsc{Constructive Dilemma (CD)}

\begin{proof}
	\have[m]{ab}{\script{A}\eor\script{B}}
	\have[n]{ac}{\script{A}\eif\script{C}}
	\have[p]{bc}{\script{B}\eif\script{C}}
	\have[\ ]{a}{\script{C}} \by{${\eor}\ast$}{ab,ac,bc}
\end{proof}

\noindent\textsc{Modus Tollens (MT)}

\begin{proof}
	\have[m]{ab}{\script{A}\eif\script{B}}
	\have[n]{a}{\enot\script{B}}
	\have[\ ]{b}{\enot\script{A}} \by{MT}{ab,a}
\end{proof}

\noindent\textsc{Hypothetical Syllogism (HS)}

\begin{proof}
	\have[m]{ab}{\script{A}\eif\script{B}}
	\have[n]{bc}{\script{B}\eif\script{C}}
	\have[\ ]{ac}{\script{A}\eif\script{C}}\by{HS}{ab,bc}
\end{proof}

\vfill

\section*{Replacement Rules}


\textsc{Commutivity} (Comm)
\begin{earg}
\item[] $(\script{A}\eand\script{B}) \Longleftrightarrow (\script{B}\eand\script{A})$\\
\item[] $(\script{A}\eor\script{B}) \Longleftrightarrow (\script{B}\eor\script{A})$\\
\item[] $(\script{A}\eiff\script{B}) \Longleftrightarrow (\script{B}\eiff\script{A})$
\end{earg}


\noindent\textsc{DeMorgan} (DeM)
\begin{earg}
\item[] $\enot(\script{A}\eor\script{B}) \Longleftrightarrow (\enot\script{A}\eand\enot\script{B})$\\
\item[] $\enot(\script{A}\eand\script{B}) \Longleftrightarrow (\enot\script{A}\eor\enot\script{B})$
\end{earg}

\noindent\textsc{Double Negation} (DN)
\begin{earg}
\item[] $\enot\enot\script{A} \Longleftrightarrow \script{A}$
\end{earg}

\noindent\textsc{Material Conditional} (MC)
\begin{earg}
\item[] $(\script{A}\eif\script{B}) \Longleftrightarrow (\enot\script{A}\eor\script{B})$\\
\item[] $(\script{A}\eor\script{B}) \Longleftrightarrow (\enot\script{A}\eif\script{B})$
\end{earg}

\noindent\textsc{Biconditional Exchange} ({\eiff}{ex})\\
\begin{earg}
\item[] $[(\script{A}\eif\script{B})\eand(\script{B}\eif\script{A})] \Longleftrightarrow (\script{A}\eiff\script{B})$
\end{earg}

\noindent\textsc{Quantifier Negation} (QN)\\
\begin{earg}
\item[] $\enot\forall\script{x}\script{A} \Longleftrightarrow \exists\script{x}\enot\script{A}$\\
\item[] $\enot\exists\script{x}\script{A} \Longleftrightarrow \forall\script{x}\enot\script{A}$
\end{earg}

\end{multicols}
}{}
