\chapter{Informal Fallacies}
\markright{Chap \ref{chap:informalfallacies}: Informal Fallacies}
\label{chap:informalfallacies}
\setlength{\parindent}{1em}

\section{What are informal fallacies?}

A fallacy is simply a mistake in reasoning. Some fallacies are formal and some are informal. In chapter \ref{chap:basicevaluation}, we saw that we could define validity formally and thus could determine whether an argument was valid or invalid without even having to know or understand what the argument was about. We saw that we could define certain valid rules of inference, such as \textit{modus ponens} and \textit{modus tollens}. These inference patterns are valid in virtue of their form, not their content. That is, any argument that has the same form as \textit{modus ponens} or modus tollens will automatically be valid. A formal fallacy is simply an argument whose form is invalid. Thus, any argument that has that form will automatically be invalid, regardless of the meaning of the sentences. Two formal fallacies that are similar to, but should never be confused with, modus ponens and modus tollens are denying the antecedent and affirming the consequent. Here are the forms of those invalid inferences:

\begin{figure}
\begin{multicols}{2}
Denying the antecedent
$P \onlyif Q$
$\lnot P$
$\therefore \lnot Q$
Affirming the consequent
$P \onlyif Q$
$Q$
$\therefore P$
\end{multicols}
\end{figure}

Any argument that has either of these forms is an invalid argument. For example:
\begin{enumerate}
\item If Kant was a deontologist, then he was a non-consequentialist.
\item Kant was not a deontologist.
\item Therefore, Kant was a not a non-consequentialist.
\end{enumerate}

The form of this argument is:
\begin{enumerate}
\item $D \onlyif C$
\item $\lnot D$
\item $\therefore\lnot C$
\end{enumerate}

As you can see, this argument has the form of the fallacy, denying the antecedent. Thus, we know that this argument is invalid even if we don't know what ``Kant'' or ``deontologist'' or ``non-consequentialist'' means. (``Kant'' was a famous German philosopher from the early 1800s, whereas ``deontology'' and
``non-consequentialist'' are terms that come from ethical theory.) It is mark of a formal fallacy that we can identify it even if we don't really understand the meanings of the sentences in the argument. Consider the following argument. It's an argument which uses silly, made-up words from Lewis Carrol's ``Jabberwocky.'' See if you can determine whether the argument's form is valid or invalid:

\begin{enumerate}
\item If toves are brillig then toves are slithy.
\item Toves are slithy
\item Therefore, toves are brillig.
\end{enumerate}

You should be able to see that this argument has the form of affirming the
consequent:
\begin{enumerate}
\item $B \onlyif S$
\item $S$
\item $\therefore B$
\end{enumerate}

As such, we know that the argument is invalid, even though we haven't got a clue what ``toves'' are or what ``slithy'' or ``brillig'' means. The point is that we can identify formal fallacies without having to know what they mean. In contrast, informal fallacies are those which cannot be identified without understanding the concepts involved in the argument. A paradigm example of an informal fallacy is the fallacy of composition. We will consider this fallacy in the next section. In the remaining sections, we will consider a number of other informal logical fallacies.

\section{Fallacy of Composition}

Consider the following argument:
Each member on the gymnastics team weighs less than 110 lbs.
Therefore, the whole gymnastics team weighs less than 110 lbs.

This arguments commits the composition fallacy. In the composition fallacy one
argues that since each part of the whole has a certain feature, it follows that the
whole has that same feature. However, you cannot generally identify any
argument that moves from statements about parts to statements about wholes
as committing the composition fallacy because whether or not there is a fallacy
depends on what feature we are attributing to the parts and wholes. Here is an
example of an argument that moves from claims about the parts possessing a
feature to a claim about the whole possessing that same feature, but doesn't
commit the composition fallacy:
Every part of the car is made of plastic. Therefore, the whole car is made
of plastic.
This conclusion does follow from the premises; there is no fallacy here. The
difference between this argument and the preceding argument (about the
gymnastics team) isn't their form. In fact both arguments have the same form:
Every part of X has the feature f. Therefore, the whole X has the feature f.
And yet one of the arguments is clearly fallacious, while the other isn't. The
difference between the two arguments is not their form, but their content. That
is, the difference is what feature is being attributed to the parts and wholes.
Some features (like weighing a certain amount) are such that if they belong to
each part, then it does not follow that they belong to the whole. Other features
(such as being made of plastic) are such that if they belong to each part, it
follows that they belong to the whole.
Here is another example:
Every member of the team has been to Paris. Therefore the team has
been to Paris.
The conclusion of this argument does not follow. Just because each member of
the team has been to Paris, it doesn't follow that the whole team has been to
Paris, since it may not have been the case that each individual was there at the
same time and was there in their capacity as a member of the team. Thus, even
though it is plausible to say that the team is composed of every member of the
team, it doesn't follow that since every member of the team has been to Paris,
the whole team has been to Paris. Contrast that example with this one:

Every member of the team was on the plane. Therefore, the whole team
was on the plane.
This argument, in contrast to the last one, contains no fallacy. It is true that if
every member is on the plane then the whole team is on the plane. And yet
these two arguments have almost exactly the same form. The only difference is
that the first argument is talking about the property, having been to Paris,
whereas the second argument is talking about the property, being on the plane.
The only reason we are able to identify the first argument as committing the
composition fallacy and the second argument as not committing a fallacy is that
we understand the relationship between the concepts involved. In the first case,
we understand that it is possible that every member could have been to Paris
without the team ever having been; in the second case we understand that as
long as every member of the team is on the plane, it has to be true that the
whole team is on the plane. The take home point here is that in order to identify
whether an argument has committed the composition fallacy, one must
understand the concepts involved in the argument. This is the mark of an
informal fallacy: we have to rely on our understanding of the meanings of the
words or concepts involved, rather than simply being able to identify the fallacy
from its form.

\section{Fallacy of Division}
The division fallacy is like the composition fallacy and they are easy to confuse.
The difference is that the division fallacy argues that since the whole has some
feature, each part must also have that feature. The composition fallacy, as we
have just seen, goes in the opposite direction: since each part has some feature,
the whole must have that same feature. Here is an example of a division fallacy:
The house costs 1 million dollars. Therefore, each part of the house costs
1 million dollars.
This is clearly a fallacy. Just because the whole house costs 1 million dollars, it
doesn't follow that each part of the house costs 1 million dollars. However, here
is an argument that has the same form, but that doesn't commit the division
fallacy:


The whole team died in the plane crash. Therefore each individual on the
team died in the plane crash.
In this example, since we seem to be referring to one plane crash in which all the
members of the team died (``the'' plane crash), it follows that if the whole team
died in the crash, then every individual on the team died in the crash. So this
argument does not commit the division fallacy. In contrast, the following
argument has exactly the same form, but does commit the division fallacy:
The team played its worst game ever tonight. Therefore, each individual
on the team played their worst game ever tonight.
It can be true that the whole team played its worst game ever even if it is true
that no individual on the team played their worst game ever. Thus, this
argument does commit the fallacy of division even though it has the same form
as the previous argument, which doesn't commit the fallacy of division. This
shows (again) that in order to identify informal fallacies (like composition and
division), we must rely on our understanding of the concepts involved in the
argument. Some concepts (like ``team'' and ``dying in a plane crash'') are such
that if they apply to the whole, they also apply to all the parts. Other concepts
(like ``team'' and ``worst game played'') are such that they can apply to the
whole even if they do not apply to all the parts.

\section{Begging the question}

Consider the following argument:
Capital punishment is justified for crimes such as rape and murder
because it is quite legitimate and appropriate for the state to put to
death someone who has committed such heinous and inhuman acts.
The premise indicator, ``because'' denotes the premise and (derivatively) the
conclusion of this argument. In standard form, the argument is this:
1. It is legitimate and appropriate for the state to put to death someone
who commits rape or murder.
2. Therefore, capital punishment is justified for crimes such as rape and
murder.


You should notice something peculiar about this argument: the premise is
essentially the same claim as the conclusion. The only difference is that the
premise spells out what capital punishment means (the state putting criminals to
death) whereas the conclusion just refers to capital punishment by name, and
the premise uses terms like ``legitimate'' and ``appropriate'' whereas the
conclusion uses the related term, ``justified.'' But these differences don't add up
to any real differences in meaning. Thus, the premise is essentially saying the
same thing as the conclusion. This is a problem: we want our premise to
provide a reason for accepting the conclusion. But if the premise is the same
claim as the conclusion, then it can't possibly provide a reason for accepting the
conclusion! Begging the question occurs when one (either explicitly or
implicitly) assumes the truth of the conclusion in one or more of the premises.
Begging the question is thus a kind of circular reasoning.
One interesting feature of this fallacy is that formally there is nothing wrong with
arguments of this form. Here is what I mean. Consider an argument that
explicitly commits the fallacy of begging the question. For example,
1. Capital punishment is morally permissible
2. Therefore, capital punishment is morally permissible
Now, apply any method of assessing validity to this argument and you will see
that it is valid by any method. If we use the informal test (by trying to imagine
that the premises are true while the conclusion is false), then the argument
passes the test, since any time the premise is true, the conclusion will have to be
true as well (since it is the exact same statement). Likewise, the argument is
valid by our formal test of validity, truth tables. But while this argument is
technically valid, it is still a really bad argument. Why? Because the point of
giving an argument in the first place is to provide some reason for thinking the
conclusion is true for those who don't already accept the conclusion. But if one
doesn't already accept the conclusion, then simply restating the conclusion in a
different way isn't going to convince them. Rather, a good argument will
provide some reason for accepting the conclusion that is sufficiently
independent of that conclusion itself. Begging the question utterly fails to do
this and this is why it counts as an informal fallacy. What is interesting about
begging the question is that there is absolutely nothing wrong with the
argument formally.


Whether or not an argument begs the question is not always an easy matter to
sort out. As with all informal fallacies, detecting it requires a careful
understanding of the meaning of the statements involved in the argument. Here
is an example of an argument where it is not as clear whether there is a fallacy of
begging the question:
Christian belief is warranted because according to Christianity there exists
a being called ``the Holy Spirit'' which reliably guides Christians towards
the truth regarding the central claims of Christianity.

1 This is a much simplified version of the view defended by Christian philosophers such as Alvin
Plantinga. Plantinga defends (something like) this claim in: Plantinga, A. 2000. Warranted
Christian Belief. Oxford, UK: Oxford University Press.

One might think that there is a kind of circularity (or begging the question)
involved in this argument since the argument appears to assume the truth of
Christianity in justifying the claim that Christianity is true. But whether or not this
argument really does beg the question is something on which there is much
debate within the sub-field of philosophy called epistemology (``study of
knowledge''). The philosopher Alvin Plantinga argues persuasively that the
argument does not beg the question, but being able to assess that argument
takes patient years of study in the field of epistemology (not to mention a careful
engagement with Plantinga's work). As this example illustrates, the issue of
whether an argument begs the question requires us to draw on our general
knowledge of the world. This is the mark of an informal, rather than formal,
fallacy.

\section{False dichotomy}

Suppose I were to argue as follows:
Raising taxes on the wealthy will either hurt the economy or it will help it.
But it won't help the economy. Therefore it will hurt the economy.
The standard form of this argument is:
1. Either raising taxes on the wealthy will hurt the economy or it will help
it.
2. Raising taxes on the wealthy won't help the economy.
3. Therefore, raising taxes on the wealthy will hurt the economy.

This argument contains a fallacy called a ``false dichotomy.'' A false dichotomy
is simply a disjunction that does not exhaust all of the possible options. In this
case, the problematic disjunction is the first premise: either raising the taxes on
the wealthy will hurt the economy or it will help it. But these aren't the only
options. Another option is that raising taxes on the wealthy will have no effect
on the economy. Notice that the argument above has the form of a disjunctive
syllogism:
\begin{enumerate}
\item $A \lor B$
\item $\lnot A$
\item $\therefore B$
\end{enumerate}
However, since the first premise presents two options as if they were the only
two options, when in fact they aren't, the first premise is false and the argument
fails. Notice that the form of the argument is perfectly good—the argument is
valid. The problem is that this argument isn't sound because the first premise of
the argument commits the false dichotomy fallacy. False dichotomies are
commonly encountered in the context of a disjunctive syllogism or constructive
dilemma (see chapter 2).
In a speech made on April 5, 2004, President Bush made the following remarks
about the causes of the Iraq war:
Saddam Hussein once again defied the demands of the world. And so I
had a choice: Do I take the word of a madman, do I trust a person who
had used weapons of mass destruction on his own people, plus people in
the neighborhood, or do I take the steps necessary to defend the
country? Given that choice, I will defend America every time.
The false dichotomy here is the claim that:
Either I trust the word of a madman or I defend America (by going to war
against Saddam Hussein's regime).
The problem is that these aren't the only options. Other options include
ongoing diplomacy and economic sanctions. Thus, even if it true that Bush
shouldn't have trusted the word of Hussein, it doesn't follow that the only other

option is going to war against Hussein's regime. (Furthermore, it isn't clear in
what sense this was needed to defend America.) That is a false dichotomy.
As with all the previous informal fallacies we've considered, the false dichotomy
fallacy requires an understanding of the concepts involved. Thus, we have to
use our understanding of world in order to assess whether a false dichotomy
fallacy is being committed or not.

\section{Equivocation}

Consider the following argument:
Children are a headache. Aspirin will make headaches go away.
Therefore, aspirin will make children go away.
This is a silly argument, but it illustrates the fallacy of equivocation. The problem
is that the word ``headache'' is used equivocally—that is, in two different senses.
In the first premise, ``headache'' is used figuratively, whereas in the second
premise ``headache'' is used literally. The argument is only successful if the
meaning of ``headache'' is the same in both premises. But it isn't and this is
what makes this argument an instance of the fallacy of equivocation.
Here's another example:
Taking a logic class helps you learn how to argue. But there is already
too much hostility in the world today, and the fewer arguments the better.
Therefore, you shouldn't take a logic class.
In this example, the word ``argue'' and ``argument'' are used equivocally.
Hopefully, at this point in the text, you recognize the difference. (If not, go back
and reread section 1.1.)
The fallacy of equivocation is not always so easy to spot. Here is a trickier
example:
The existence of laws depends on the existence of intelligent beings like
humans who create the laws. However, some laws existed before there
were any humans (e.g., laws of physics). Therefore, there must be some
non-human, intelligent being that created these laws of nature.

The term ``law'' is used equivocally here. In the first premise it is used to refer to
societal laws, such as criminal law; in the second premise it is used to refer to
laws of nature. Although we use the term ``law'' to apply to both cases, they are
importantly different. Societal laws, such as the criminal law of a society, are
enforced by people and there are punishments for breaking the laws. Natural
laws, such as laws of physics, cannot be broken and thus there are no
punishments for breaking them. (Does it make sense to scold the electron for
not doing what the law says it will do?)
As with every informal fallacy we have examined in this section, equivocation
can only be identified by understanding the meanings of the words involved. In
fact, the definition of the fallacy of equivocation refers to this very fact: the same
word is being used in two different senses (i.e., with two different meanings). So,
unlike formal fallacies, identifying the fallacy of equivocation requires that we
draw on our understanding of the meaning of words and of our understanding
of the world, generally.

\section{Slippery slope fallacies}
Slippery slope fallacies depend on the concept of vagueness. When a concept
or claim is vague, it means that we don't know precisely what claim is being
made, or what the boundaries of the concept are. The classic example used to
illustrate vagueness is the ``sorites paradox.'' The term ``sorites'' is the Greek
term for ``heap'' and the paradox comes from ancient Greek philosophy. Here is
the paradox. I will give you two claims that each sound very plausible, but in
fact lead to a paradox. Here are the two claims:
1. One grain of sand is not a heap of sand.
2. If I start with something that is not a heap of sand, then adding one
grain of sand to that will not create a heap of sand.

For example, two grains of sand is not a heap, thus (by the second claim) neither
is three grains of sand. But since three grains of sand is not a heap then (by the
second claim again) neither is four grains of sand. You can probably see where
this is going. By continuing to add one grain of sand over and over, I will
eventually end up with something that is clearly a heap of sand, but that won't
be counted as a heap of sand if we accept both claims 1 and 2 above.

Philosophers continue to argue and debate about how to resolve the sorites
paradox, but the point for us is just to illustrate the concept of vagueness. The
concept ``heap'' is a vague concept in this example. But so are so many other
concepts, such a color concepts (red, yellow, green, etc.), moral concepts (right,
wrong, good, bad), and just about any other concept you can think of. The one
domain that seems to be unaffected by vagueness is mathematical and logical
concepts. There are two fallacies related to vagueness: the causal slippery slope
and the conceptual slippery slope. We'll cover the conceptual slippery slope
first since it relates most closely to the concept of vagueness I've explained
above.

\subsection{Conceptual slippery slope}
It may be true that there is no essential difference between 499 grains of sand
and 500 grains of sand. But even if that is so, it doesn't follow that there is no
difference between 1 grain of sand and 5 billion grains of sand. In general, just
because we cannot draw a distinction between A and B, and we cannot draw a
distinction between B and C, it doesn't mean we cannot draw a distinction
between A and C. Here is an example of a conceptual slippery slope fallacy.
It is illegal for anyone under 21 to drink alcohol. But there is no
difference between someone who is 21 and someone who is 20 years 11
months old. So there is nothing wrong with someone who is 20 years and
11 months old drinking. But since there is no real distinction between
being one month older and one month younger, there shouldn't be
anything wrong with drinking at any age. Therefore, there is nothing
wrong with allowing a 10 year old to drink alcohol.
Imagine the life of an individual in stages of 1 month intervals. Even if it is true
that there is no distinction in kind between any one of those stages, it doesn't
follow that there isn't a distinction to be drawn at the extremes of either end.
Clearly there is a difference between a 5 year old and a 25 year old—a
distinction in kind that is relevant to whether they should be allowed to drink
alcohol. The conceptual slippery slope fallacy assumes that because we cannot
draw a distinction between adjacent stages, we cannot draw a distinction at all
between any stages. One clear way of illustrating this is with color. Think of a
color spectrum from purple to red to orange to yellow to green to blue. Each
color grades into the next without there being any distinguishable boundaries

between the colors—a continuous spectrum. Even if it is true that for any two
adjacent hues on the color wheel, we cannot distinguish between the two, it
doesn't follow from this that there is no distinction to be drawn between any two
portions of the color wheel, because then we'd be committed to saying that
there is no distinguishable difference between purple and yellow! The example
of the color spectrum illustrates the general point that just because the
boundaries between very similar things on a spectrum are vague, it doesn't
follow that there are no differences between any two things on that spectrum.
Whether or not one will identify an argument as committing a conceptual
slippery slope fallacy, depends on the other things one believes about the world.
Thus, whether or not a conceptual slippery slope fallacy has been committed will
often be a matter of some debate. It will itself be vague. Here is a good
example that illustrates this point.
People are found not guilty by reason of insanity when they cannot avoid
breaking the law. But people who are brought up in certain deprived
social circumstances are not much more able than the legally insane to
avoid breaking the law. So we should not find such individuals guilty any
more than those who are legally insane.
Whether there is conceptual slippery slope fallacy here depends on what you
think about a host of other things, including individual responsibility, free will,
the psychological and social effects of deprived social circumstances such as
poverty, lack of opportunity, abuse, etc. Some people may think that there are
big differences between those who are legally insane and those who grow up in
deprived social circumstances. Others may not think the differences are so
great. The issues here are subtle, sensitive, and complex, which is why it is
difficult to determine whether there is any fallacy here or not. If the differences
between those who are insane and those who are the product of deprived social
circumstances turn out to be like the differences between one shade of yellow
and an adjacent shade of yellow, then there is no fallacy here. But if the
differences turn out to be analogous to those between yellow and green (i.e.,
with many distinguishable stages of difference between) then there would
indeed be a conceptual slippery slope fallacy here. The difficulty of
distinguishing instances of the conceptual slippery slope fallacy, and the fact
that distinguishing it requires us to draw on our knowledge about the world,
shows that the conceptual slippery slope fallacy is an informal fallacy.


\subsection{Causal slippery slope fallacy}
The causal slippery slope fallacy is committed when one event is said to lead to
some other (usually disastrous) event via a chain of intermediary events. If you
have ever seen Direct TV's ``get rid of cable'' commercials, you will know exactly
what I'm talking about. (If you don't know what I'm talking about you should
Google it right now and find out. They're quite funny.) Here is an example of a
causal slippery slope fallacy (it is adapted from one of the Direct TV
commercials):
If you use cable, your cable will probably go on the fritz. If your cable is
on the fritz, you will probably get frustrated. When you get frustrated you
will probably hit the table. When you hit the table, your young daughter
will probably imitate you. When your daughter imitates you, she will
probably get thrown out of school. When she gets thrown out of school,
she will probably meet undesirables. When she meets undesirables, she
will probably marry undesirables. When she marries undesirables, you
will probably have a grandson with a dog collar. Therefore, if you use
cable, you will probably have a grandson with dog collar.
This example is silly and absurd, yes. But it illustrates the causal slippery slope
fallacy. Slippery slope fallacies are always made up of a series of conjunctions of
probabilistic conditional statements that link the first event to the last event. A
causal slippery slope fallacy is committed when one assumes that just because
each individual conditional statement is probable, the conditional that links the
first event to the last event is also probable. Even if we grant that each ``link'' in
the chain is individually probable, it doesn't follow that the whole chain (or the
conditional that links the first event to the last event) is probable. Suppose, for
the sake of the argument, we assign probabilities to each ``link'' or conditional
statement, like this. (I have italicized the consequents of the conditionals and
assigned high conditional probabilities to them. The high probability is for the
sake of the argument; I don't actually think these things are as probable as I've
assumed here.)
\begin{itemize}
\item If you use cable, then your cable will probably go on the fritz (.9)
\item If your cable is on the fritz, then you will probably get angry (.9)
\item If you get angry, then you will probably hit the table (.9)
\item If you hit the table, your daughter will probably imitate you (.8)
\item If your daughter imitates you, she will probably be kicked out of school (.8)
\item If she is kicked out of school, she will probably meet undesirables (.9)
\item If she meets undesirables, she will probably marry undesirables (.8)
\item If she marries undesirables, you will probably have a grandson with a dog collar (.8)
\end{itemize}
However, even if we grant the probabilities of each link in the chain is high (80-
90\% probable), the conclusion doesn't even reach a probability higher than
chance. Recall that in order to figure the probability of a conjunction, we must
multiply the probability of each conjunct:
$(.9) \times (.9) \times (.9) \times (.8) \times (.8) \times (.9) \times (.8) \times (.8) = .27$
That means the probability of the conclusion (i.e., that if you use cable, you will
have a grandson with a dog collar) is only 27\%, despite the fact that each
conditional has a relatively high probability! The causal slippery slope fallacy is
actually a formal probabilistic fallacy and so could have been discussed in
chapter 3 with the other formal probabilistic fallacies. What makes it a formal
rather than informal fallacy is that we can identify it without even having to know
what the sentences of the argument mean. I could just have easily written out a
nonsense argument comprised of series of probabilistic conditional statements.
But I would still have been able to identify the causal slippery slope fallacy
because I would have seen that there was a series of probabilistic conditional
statements leading to a claim that the conclusion of the series was also probable.
That is enough to tell me that there is a causal slippery slope fallacy, even if I
don't really understand the meanings of the conditional statements.
It is helpful to contrast the causal slippery slope fallacy with the valid form of
inference, hypothetical syllogism. Recall that a hypothetical syllogism has the
following kind of form:
\begin{itemize}
\item $A \onlyif B$
\item $B \onlyif C$
\item $C \onlyif D$
\item $D \onlyif E$
\item $\therefore A \onlyif E$
\end{itemize}
The only difference between this and the causal slippery slope fallacy is that
whereas in the hypothetical syllogism, the link between each component is
certain, in a causal slippery slope fallacy, the link between each event is
probabilistic. It is the fact that each link is probabilistic that accounts for the
fallacy. One way of putting this is point is that probability is not transitive. Just
because A makes B probable and B makes C probable and C makes X probable,
it doesn't follow that A makes X probable. In contrast, when the links are certain
rather than probable, then if A always leads to B and B always leads to C and C
always leads to X, then it has to be the case that A always leads to X.

\section{Fallacies of relevance}
What all fallacies of relevance have in common is that they make an argument or
response to an argument that is irrelevant. Fallacies of relevance can be
compelling psychologically, but it is important to distinguish between rhetorical
techniques that are psychologically compelling, on the one hand, and rationally
compelling arguments, on the other. What makes something a fallacy is that it
fails to be rationally compelling, once we have carefully considered it. That said,
arguments that fail to be rationally compelling may still be psychologically or
emotionally compelling. The first fallacy of relevance that we will consider, the
ad hominem fallacy, is an excellent example a fallacy that can be psychologically
compelling.

\subsection{Ad hominem}

``Ad hominem'' is a Latin phrase that can be translated into English as the phrase, ``against the man.'' In an ad hominem fallacy, instead of responding to (or attacking) the argument a person has made, one attacks the person him or herself. In short, one attacks the person making the argument rather than the argument itself. Here is an anecdote that reveals an ad hominem fallacy (and that has actually occurred in my ethics class before). A philosopher named Peter Singer had made an argument that it is
morally wrong to spend money on luxuries for oneself rather than give all
of your money that you don't strictly need away to charity. The argument
is actually an argument from analogy (whose details I discussed in section
3.3), but the essence of the argument is that there are every day in this
world children who die preventable deaths, and there are charities who

could save the lives of these children if they are funded by individuals
from wealthy countries like our own. Since there are things that we all
regularly buy that we don't need (e.g., Starbuck's lattes, beer, movie
tickets, or extra clothes or shoes we don't really need), if we continue to
purchase those things rather than using that money to save the lives of
children, then we are essentially contributing to the deaths of those
children if we choose to continue to live our lifestyle of buying things we
don't need, rather than donating the money to a charity that will save
lives of children in need. In response to Singer's argument, one student
in the class asked: ``Does Peter Singer give his money to charity? Does
he do what he says we are all morally required to do?''
The implication of this student's question (which I confirmed by following up
with her) was that if Peter Singer himself doesn't donate all his extra money to
charities, then his argument isn't any good and can be dismissed. But that
would be to commit an ad hominem fallacy. Instead of responding to the
argument that Singer had made, this student attacked Singer himself. That is,
they wanted to know how Singer lived and whether he was a hypocrite or not.
Was he the kind of person who would tell us all that we had to live a certain way
but fail to live that way himself? But all of this is irrelevant to assessing Singer's
argument. Suppose that Singer didn't donate his excess money to charity and
instead spent it on luxurious things for himself. Still, the argument that Singer
has given can be assessed on its own merits. Even if it were true that Peter
Singer was a total hypocrite, his argument may nevertheless be rationally
compelling. And it is the quality of the argument that we are interested in, not
Peter Singer's personal life and whether or not he is hypocritical. Whether
Singer is or isn't a hypocrite, is irrelevant to whether the argument he has put
forward is strong or weak, valid or invalid. The argument stands on its own and
it is that argument rather than Peter Singer himself that we need to assess.
Nonetheless, there is something psychologically compelling about the question:
Does Peter Singer practice what he preaches? I think what makes this question
seem compelling is that humans are very interested in finding ``cheaters'' or
hypocrites---those who say one thing and then do another. Evolutionarily, our
concern with cheaters makes sense because cheaters can't be trusted and it is
essential for us (as a group) to be able to pick out those who can't be trusted.
That said, whether or not a person giving an argument is a hypocrite is irrelevant
to whether that person's argument is good or bad. So there may be
psychological reasons why humans are prone to find certain kinds of ad

hominem fallacies psychologically compelling, even though ad hominem
fallacies are not rationally compelling.
Not every instance in which someone attacks a person's character is an ad
hominem fallacy. Suppose a witness is on the stand testifying against a
defendant in a court of law. When the witness is cross examined by the defense
lawyer, the defense lawyer tries to go for the witness's credibility, perhaps by
digging up things about the witness's past. For example, the defense lawyer
may find out that the witness cheated on her taxes five years ago or that the
witness failed to pay her parking tickets. The reason this isn't an ad hominem
fallacy is that in this case the lawyer is trying to establish whether what the
witness is saying is true or false and in order to determine that we have to know
whether the witness is trustworthy. These facts about the witness's past may be
relevant to determining whether we can trust the witness's word. In this case,
the witness is making claims that are either true or false rather than giving an
argument. In contrast, when we are assessing someone's argument, the
argument stands on its own in a way the witness's testimony doesn't. In
assessing an argument, we want to know whether the argument is strong or
weak and we can evaluate the argument using the logical techniques surveyed
in this text. In contrast, when a witness is giving testimony, they aren't trying to
argue anything. Rather, they are simply making a claim about what did or didn't
happen. So although it may seem that a lawyer is committing an ad hominem
fallacy in bringing up things about the witness's past, these things are actually
relevant to establishing the witness's credibility. In contrast, when considering
an argument that has been given, we don't have to establish the arguer's
credibility because we can assess the argument they have given on its own
merits. The arguer's personal life is irrelevant.

\subsection{Straw Man}

Suppose that my opponent has argued for a position, call it position A, and in
response to his argument, I give a rationally compelling argument against
position B, which is related to position A, but is much less plausible (and thus
much easier to refute). What I have just done is attacked a straw man—a
position that ``looks like'' the target position, but is actually not that position.
When one attacks a straw man, one commits the straw man fallacy. The straw
man fallacy misrepresents one's opponent's argument and is thus a kind of
irrelevance. Here is an example.

Two candidates for political office in Colorado, Tom and Fred, are having
an exchange in a debate in which Tom has laid out his plan for putting
more money into health care and education and Fred has laid out his plan
which includes earmarking more state money for building more prisons
which will create more jobs and, thus, strengthen Colorado's economy.
Fred responds to Tom's argument that we need to increase funding to
health care and education as follows: ``I am surprised, Tom, that you are
willing to put our state's economic future at risk by sinking money into
these programs that do not help to create jobs. You see, folks, Tom's
plan will risk sending our economy into a tailspin, risking harm to
thousands of Coloradans. On the other hand, my plan supports a healthy
and strong Colorado and would never bet our state's economic security
on idealistic notions that simply don't work when the rubber meets the
road.''
Fred has committed the straw man fallacy. Just because Tom wants to increase
funding to health care and education does not mean he does not want to help
the economy. Furthermore, increasing funding to health care and education
does not entail that fewer jobs will be created. Fred has attacked a position that
is not the position that Tom holds, but is in fact a much less plausible, easier to
refute position. However, it would be silly for any political candidate to run on a
platform that included ``harming the economy.'' Presumably no political
candidate would run on such a platform. Nonetheless, this exact kind of straw
man is ubiquitous in political discourse in our country.
Here is another example.
Nancy has just argued that we should provide middle schoolers with sex
education classes, including how to use contraceptives so that they can
practice safe sex should they end up in the situation where they are
having sex. Fran responds: ``proponents of sex education try to
encourage our children to a sex-with-no-strings-attached mentality, which
is harmful to our children and to our society.''
Fran has committed the straw man (or straw woman) fallacy by misrepresenting
Nancy's position. Nancy's position is not that we should encourage children to
have sex, but that we should make sure that they are fully informed about sex so
that if they do have sex, they go into it at least a little less blindly and are able to
make better decision regarding sex.

As with other fallacies of relevance, straw man fallacies can be compelling on
some level, even though they are irrelevant. It may be that part of the reason
we are taken in by straw man fallacies is that humans are prone to ``demonize''
the ``other''—including those who hold a moral or political position different
from our own. It is easy to think bad things about those with whom we do not
regularly interact. And it is easy to forget that people who are different than us
are still people just like us in all the important respects. Many years ago,
atheists were commonly thought of as highly immoral people and stories about
the horrible things that atheists did in secret circulated widely. People believed
that these strange ``others'' were capable of the most horrible savagery. After
all, they may have reasoned, if you don't believe there is a God holding us
accountable, why be moral? The Jewish philosopher, Baruch Spinoza, was an
atheist who lived in the Netherlands in the 17th century. He was accused of all
sorts of things that were commonly believed about atheists. But he was in fact
as upstanding and moral as any person you could imagine. The people who
knew Spinoza knew better, but how could so many people be so wrong about
Spinoza? I suspect that part of the reason is that since at that time there were
very few atheists (or at least very few people actually admitted to it), very few
people ever knowingly encountered an atheist. Because of this, the stories
about atheists could proliferate without being put in check by the facts. I
suspect the same kind of phenomenon explains why certain kinds of straw man
fallacies proliferate. If you are a conservative and mostly only interact with other
conservatives, you might be prone to holding lots of false beliefs about liberals.
And so maybe you are less prone to notice straw man fallacies targeted at
liberals because the false beliefs you hold about them incline you to see the
straw man fallacies as true.

\section{Tu quoque}
``Tu quoque'' is a Latin phrase that can be translated into English as ``you too''
or ``you, also.'' The tu quoque fallacy is a way of avoiding answering a criticism
by bringing up a criticism of your opponent rather than answer the criticism. For
example, suppose that two political candidates, A and B, are discussing their
policies and A brings up a criticism of B's policy. In response, B brings up her
own criticism of A's policy rather than respond to A's criticism of her policy. B
has here committed the tu quoque fallacy. The fallacy is best understood as a
way of avoiding having to answer a tough criticism that one may not have a
good answer to. This kind of thing happens all the time in political discourse.

Tu quoque, as I have presented it, is fallacious when the criticism one raises is
simply in order to avoid having to answer a difficult objection to one's argument
or view. However, there are circumstances in which a tu quoque kind of
response is not fallacious. If the criticism that A brings toward B is a criticism
that equally applies not only to A's position but to any position, then B is right to
point this fact out. For example, suppose that A criticizes B for taking money
from special interest groups. In this case, B would be totally right (and there
would be no tu quoque fallacy committed) to respond that not only does A take
money from special interest groups, but every political candidate running for
office does. That is just a fact of life in American politics today. So A really has
no criticism at all to B since everyone does what B is doing and it is in many
ways unavoidable. Thus, B could (and should) respond with a ``you too'' rebuttal
and in this case that rebuttal is not a tu quoque fallacy.

\subsection{Genetic fallacy}
The genetic fallacy occurs when one argues (or, more commonly, implies) that
the origin of something (e.g., a theory, idea, policy, etc.) is a reason for rejecting
(or accepting) it. For example, suppose that Jack is arguing that we should
allow physician assisted suicide and Jill responds that that idea first was used in
Nazi Germany. Jill has just committed a genetic fallacy because she is implying
that because the idea is associated with Nazi Germany, there must be
something wrong with the idea itself. What she should have done instead is
explain what, exactly, is wrong with the idea rather than simply assuming that
there must be something wrong with it since it has a negative origin. The origin
of an idea has nothing inherently to do with its truth or plausibility. Suppose
that Hitler constructed a mathematical proof in his early adulthood (he didn't,
but just suppose). The validity of that mathematical proof stands on its own; the
fact that Hitler was a horrible person has nothing to do with whether the proof is
good. Likewise with any other idea: ideas must be assessed on their own merits
and the origin of an idea is neither a merit nor demerit of the idea.
Although genetic fallacies are most often committed when one associates an
idea with a negative origin, it can also go the other way: one can imply that
because the idea has a positive origin, the idea must be true or more plausible.
For example, suppose that Jill argues that the Golden Rule is a good way to live
one's life because the Golden Rule originated with Jesus in the Sermon on the
Mount (it didn't, actually, even though Jesus does state a version of the Golden
Rule). Jill has committed the genetic fallacy in assuming that the (presumed)

fact that Jesus is the origin of the Golden Rule has anything to do with whether
the Golden Rule is a good idea.
I'll end with an example from William James's seminal work, The Varieties of
Religious Experience. In that book (originally a set of lectures), James considers
the idea that if religious experiences could be explained in terms of neurological
causes, then the legitimacy of the religious experience is undermined. James,
being a materialist who thinks that all mental states are physical states—
ultimately a matter of complex brain chemistry, says that the fact that any
religious experience has a physical cause does not undermine that veracity of
that experience. Although he doesn't use the term explicitly, James claims that
the claim that the physical origin of some experience undermines the veracity of
that experience is a genetic fallacy. Origin is irrelevant for assessing the veracity
of an experience, James thinks. In fact, he thinks that religious dogmatists who
take the origin of the Bible to be the word of God are making exactly the same
mistake as those who think that a physical explanation of a religious experience
would undermine its veracity. We must assess ideas for their merits, James
thinks, not their origins.

\subsection{Appeal to consequences}
The appeal to consequences fallacy is like the reverse of the genetic fallacy:
whereas the genetic fallacy consists in the mistake of trying to assess the truth or
reasonableness of an idea based on the origin of the idea, the appeal to
consequences fallacy consists in the mistake of trying to assess the truth or
reasonableness of an idea based on the (typically negative) consequences of
accepting that idea. For example, suppose that the results of a study revealed
that there are IQ differences between different races (this is a fictitious example,
there is no such study that I know of). In debating the results of this study, one
researcher claims that if we were to accept these results, it would lead to
increased racism in our society, which is not tolerable. Therefore, these results
must not be right since if they were accepted, it would lead to increased racism.
The researcher who responded in this way has committed the appeal to
consequences fallacy. Again, we must assess the study on its own merits. If
there is something wrong with the study, some flaw in its design, for example,
then that would be a relevant criticism of the study. However, the fact that the
results of the study, if widely circulated, would have a negative effect on society
is not a reason for rejecting these results as false. The consequences of some
idea (good or bad) are irrelevant to the truth or reasonableness of that idea.

Notice that the researchers, being convinced of the negative consequences of
the study on society, might rationally choose not to publish the study (for fear of
the negative consequences). This is totally fine and is not a fallacy. The fallacy
consists not in choosing not to publish something that could have adverse
consequences, but in claiming that the results themselves are undermined by
the negative consequences they could have. The fact is, sometimes truth can
have negative consequences and falsehoods can have positive consequences.
This just goes to show that the consequences of an idea are irrelevant to the
truth or reasonableness of an idea.

\subsection{Appeal to authority}

In a society like ours, we have to rely on authorities to get on in life. For example, the things I believe about electrons are not things that I have ever verified for myself. Rather, I have to rely on the testimony and authority of physicists to tell me what electrons are like. Likewise, when there is something wrong with my car, I have to rely on a mechanic (since I lack that expertise) to tell me what is wrong with it. Such is modern life. So there is nothing wrong with needing to rely on authority figures in certain fields (people with the relevant expertise in that field)---it is inescapable. The fallacy comes when we invoke someone whose expertise is not relevant to the issue for which we are invoking it.

For example, suppose that a group of doctors sign a petition to prohibit abortions, claiming that abortions are morally wrong. If Bob cites that fact that these doctors are against abortion, therefore abortion must be morally wrong, then Bob has committed the appeal to authority fallacy. The problem is that doctors are not authorities on what is morally right or wrong. Even if they are authorities on how the body works and how to perform certain procedures (such as abortion), it doesn't follow that they are authorities on whether or not these procedures ought to be performed---they are not experts on the ethical status of these procedures. It would be just as much an appeal to consequences fallacy if Melissa were to
argue that since some other group of doctors supported abortion, that shows that it must be morally acceptable. In either case, since doctors are not authorities on moral issues, their opinions on a moral issue like abortion is irrelevant. In general, an appeal to authority fallacy occurs when someone takes what an individual says as evidence for some claim, when that individual has no particular expertise in the relevant domain (even if they do have expertise in some other, unrelated, domain).
